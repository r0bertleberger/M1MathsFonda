\documentclass[a4paper,12pt,reqno]{amsart}
\usepackage{lille}
%==============================================================
\lilleset{
  formation=Master 1\iere année,
  parcours=parcours Mathématiques,
  annee=2024--2025,
  module=Analyse,
  couleur=teal
}
%==============================================================
% -------------- Autres bibliothèques
\usepackage{dsfont} % pour \mathds{1}

% -------------- Les abréviations pour M1AN
% les ensembles standard
\def\R{\mathbb{R}}
\def\C{\mathbb{C}}
\def\N{\mathbb{N}}
\def\Z{\mathbb{Z}}
\def\Q{\mathbb{Q}}
% les parenthèses et leurs dérivées
\newcommand{\p}[1]{\left(#1\right)}% les parenthèses automatiques
\newcommand{\bp}[1]{\bigl(#1\bigr)}% les parenthèses 'big'
\newcommand{\Bp}[1]{\Bigl(#1\Bigr)}% les parenthèses 'Big'
\newcommand{\Bbp}[1]{\biggl(#1\biggr)}% les parenthèses 'bigg'
\newcommand{\BBp}[1]{\Biggl(#1\Biggr)}% les parenthèses 'Bigg'
\newcommand*{\suite}[2][n]{\left( #2 \right)_{#1}}% exemple \suite{u_n}, \suite[k>0]{u_k}
\newcommand*{\suiteN}[1]{\suite[n\in\mathbb{N}]{#1}}% exemple \suiteN{u_n}
\newcommand*{\abs}[1]{\left\lvert{\ifx\hfuzz#1\hfuzz \,\cdot\,\else#1\fi}\right\rvert}% |.|
\newcommand*{\norm}[1]{\left\lVert{\ifx\hfuzz#1\hfuzz \,\cdot\,\else#1\fi}\right\rVert}% norme
\newcommand*{\norminf}[1]{\norm{#1}_{\infty}}% la norme sup
\newcommand*{\vertiii}[1]{{\left\vert\kern-0.25ex\left\vert\kern-0.25ex\left\vert
  {\ifx\hfuzz#1\hfuzz \,\cdot\,\else#1\fi}
    \right\vert\kern-0.25ex\right\vert\kern-0.25ex\right\vert}}
\let\normop\vertiii% exemple \normop{T}
\newcommand*{\scalprod}[3][]{#1\langle{#2}\kern1pt #1|{#3}#1\rangle}% exemple \scalprod[\big]{A}{B}
\newcommand*{\ensemble}[3][]{#1\{ #2 \;#1|\; #3 #1\}}% exemple \ensemble[\big]{x^2}{x \in \R}
% indication et remarque
\newcommand{\indication}[1]{\emph{Indication : #1}}
\newcommand{\remarque}[1]{\emph{Remarque : #1}}
% la fonction indicatrice
\usepackage{dsfont}
\newcommand*{\ind}{\mathds{1}}% exemple \ind_{[0,1]}
% autre symboles
\DeclareMathOperator{\im}{Im}% exemple \im A
\let\ker\relax\DeclareMathOperator{\ker}{Ker}% exemple \ker A
\DeclareMathOperator{\vect}{vect}% exemple \vect\ensemble{x_n}{n\in\N}
\DeclareMathOperator{\dist}{dist}% exemple \dist(x,M)
\DeclareMathOperator{\id}{Id}% exemple T-\lambda\id
\DeclareMathOperator{\supp}{supp}% le support, exemple \supp \ind_{A} = A
\renewcommand{\Re}{\mathfrak{Re}}% la partie réelle
\newcommand{\dd}{\kern1pt\mathrm{d}}% le «d» du dx, dt, ...

% pour surligner
\sisolutions{
  \usepackage{soul}
  \colorlet{hl}{yellow!35!white}
  \sethlcolor{hl}
  \renewcommand{\hl}[1]{\relax\ifmmode\colorbox{hl}{\ensuremath{#1}}\else\texthl{#1}\fi}
}


%%%%%%
\newcommand{\dt}{\mathrm dt}
\newcommand{\dx}{\mathrm dx}
\newcommand{\du}{\mathrm du}
\newcommand{\dy}{\mathrm dy}
\newcommand{\dmu}{\mathrm d\mu}
\newcommand{\Nfrac}[1]{\dstyle\nicefrac{\varepsilon}{#1}}
\newcommand{\inté}{\mathrm{int}\ }
\newcommand{\TQ}{\ |\ }
\newcommand\YUGE{\fontsize{38.4}{48}\selectfont}
\newcommand{\dstyle}{\displaystyle}
\newcommand{\tend}{\underset{n\to+\infty}{\longrightarrow}}
\newcommand{\ttend}{\underset{\longrightarrow}{\longrightarrow}}
\newcommand{\Lf}{L^2([-\pi,\pi])}
\newcommand{\intr}{\dstyle\int_{-\infty}^{+\infty}}
\newcommand{\ex}[1]{\mathrm{e}^{#1}}
\renewcommand{\ker}{\mathrm{ker}}
\newcommand{\Id}{\mathrm{Id}}
\newcommand{\Par}[1]{\left(#1\right)}
\newcommand{\Lim}{\underset{n\to\infty}{\mathrm{lim}}}
\newcommand{\intI}{\dstyle\int_{-\infty}^{+\infty}}
\newcommand{\ps}[1]{\left\langle #1 \right\rangle}
\renewcommand{\abs}[1]{\left| #1 \right|}
\newcommand{\ens}[1]{\left\{#1\right\}}
\lilleset{titre={DM 3, Raphaël Casanova}}

\begin{document}


\begin{exo}\emph{(Espace de Schwartz)}

  On rappelle que $\mathcal{S}(\mathbb{R})$ désigne l'espace de Schwartz, constitué des fonctions $f : \mathbb{R} \to \mathbb{R}$ infiniment dérivables et telles que, pour tout $k, m \in \mathbb{N}$,
  \[
  \sup_{x \in \mathbb{R}} \abs{x^m f^{(k)}(x)} < +\infty,
  \]
  où $f^{(k)}$ désigne la $k$-ième dérivée de $f$.
  \begin{enumerate}
      \item Soit $f \in \mathcal{S}(\mathbb{R})$ et $p$ est une fonction polynomiale.
      \begin{enumerate}
         \item Montrer que $pf \in \mathcal{S}(\mathbb{R})$.
         \item Est-ce que $p\ast f$ est bien définie ? Si oui, est-ce que $p\ast f \in \mathcal{S}(\mathbb{R})$ ?
       \end{enumerate}
      \item Soient $f, g \in \mathcal{S}(\mathbb{R})$. On suppose que $f \ast g = 0$.
      \begin{enumerate}
          \item Peut-on en déduire que $f = 0$ ou $g = 0$ ?
          \item Qu'en est-il si $f = g$ ?
      \end{enumerate}
      \item Soit $g \in \mathcal{S}(\mathbb{R})$. Montrer qu'il existe une unique fonction $f \in \mathcal{S}(\mathbb{R})$ telle que
      \[
        f^{(12)} + f^{(8)} + f = g.
      \]
  \end{enumerate}
\end{exo}

\hrule


a)(i) Soit $f\in \mathcal S$, montrons que pour $k,m,\ell\in\N,\ x\mapsto x^m\Par{x^kf(x)^{(\ell)}}$ est bornée, on calcule 
$$\abs{x^m\Par{x^kf(x)}^{(\ell)}}=\abs{x^m\sum_{i=0}^\ell \binom{\ell}{i}\frac{k!}{\Par{k-i}!}x^{k-i}f^{(\ell-i)}(x)}=\abs{\sum_{i=0}^\ell \binom{\ell}{i}\frac{k!}{\Par{k-i}!}x^{k-i+m}f^{(\ell-i)}(x)}$$
et par hypothèse, il existe $M\in \R$ tel que,
$$\forall x\in\R,\ x^{k-i+m}f^{(\ell-i)}(x)<M$$
on a donc 
$$\abs{x^m\Par{x^kf(x)}^{(\ell)}}<\sum_{i=0}^\ell \binom{\ell}{i}\frac{k!}{\Par{k-i}!}M<\infty$$
et puisque l'espace de Schwartz est un espace vectoriel, par linéarité on constante que pour tout $p\in\R[X]$, $pf\in \mathcal S$.


a)(ii) Soit $x\in\R$ fixé, on pose 
$$\tilde p:t\longmapsto p(x-t)$$
d'après la formule du binôme de Newton, c'est aussi un polynôme en $t$ (car $x$ est fixe) donc d'après la question précédente, $\tilde pf\in \mathcal S$ donc, par exemple, 
$$\exists c>0\ :\ \forall t\in\R,\ \abs{\Par{1+t^2}\tilde p(t)f(t)}<c$$
autrement dit 
$$\forall t\in\R,\ \abs{p(x-t)f(t)}<\frac{c}{1+t^2}\in L^1$$
donc 
$$p*f(x)=\intr p(x-t)f(t)\dt<\infty$$
et $p*f$ est bien définie.


Par contre, $p*f\notin\mathcal S$ car si on considère $p=1$ et $f:t\mapsto \ex{-t^2/2}$, alors on a 
$$\forall x\in\R,\ p*f(x)=\intr\ex{-t^2/2}\dt=\sqrt{2\pi}$$
donc en prenant $k=0$ et $m=1$ dans la définition de l'espace de Schwartz, on a 
$$\underset{x\in\R}{\mathrm{sup}}\abs{x^m \Par{p*f(x)}^{(k)}}=+\infty$$
donc, puisque $f\in\mathcal S$, le contre-exemple est valide et $p*f\notin\mathcal S.$\\


b)(i) Soit 
$$f : x \mapsto \mathcal{F}^{-1} \bigg( \exp\bigg( -\frac{1}{x(1-x)} \bigg) \chi_{]0,1[}(x) \bigg)$$
Cet objet est bien défini puisque la transformée de Fourier est bijective de $\mathcal S\longrightarrow\mathcal S$ et que la fonction considérée, $x\mapsto \mathrm\exp\bigg( -\frac{1}{x(1-x)} \bigg) \chi_{]0,1[}(x) \bigg)$ est $C^\infty$ à support compact donc est dans $\mathcal S$. On pose de même 
$$g : x \mapsto \mathcal{F}^{-1} \bigg( \exp\bigg( -\frac{1}{x(1-x)} \bigg) \chi_{]-1,1[}(x) \bigg)$$
qui est dans $\mathcal S$ au même titre que $f$. D'après le cours, on a alors
$$f*g=\hat f.\hat g$$
donc on a, (normalement on aurait un "presque-partout" mais ici $f$ et $g$ sont continues donc c'est un "partout")
$$f*g(x)=\bigg( \exp\bigg( -\frac{1}{x(1-x)} \bigg) \chi_{]0,1[}(x) \bigg)\bigg( \exp\bigg( -\frac{1}{x(1-x)} \bigg) \chi_{]-1,0[}(x) \bigg)=0$$
donc $f*g$ est nulle presque-partout et on a trouvé $f,g\in\mathcal S$ telles que $f*g=0$ sans que $f=0$ ou $g=0$, donc non on ne peut pas déduire que "$f*g=0$" que $f=0$ ou $g=0$.

b)(ii) Si $f=g$, alors on aurait 
$$f*f=\hat f^2=0$$
donc, puisque on est dans $\mathcal S$ on peut appliquer la formule d'inversion de Fourier, qui est ici aussi valide "partout" au lieu de "presque-partout" par contiuité, et ainsi $f=0$.\\


c) On va procéder par analyse-synthèse :\\
$\star$ Analyse : soit $f$ une solution, alors, on applique la transformée de Fourier et on obtient 
$$\forall y\in\R,\ \Par{2i\pi y}^{12}\hat f(y)+\Par{2i\pi y}^{8}\hat f(y)+\hat f(y)=\hat g(y)$$
donc nécessairement
$$\forall y\in\R,\ \hat f(y)=\frac{\hat g(y)}{\Par{2i\pi y}^{12}+\Par{2i\pi y}^{8}+1}.$$


$\star$ Montrons que l'expression de $\hat f$ trouvée convient : 
C'est le produit de la fonction $\hat g$ qui est $C^\infty$ et d'une fonction rationnelle sans pôle réel, donc $C^\infty$, donc $\hat f$ est aussi de classe $C^\infty$, on va aussi montrer que c'est une fonction à décroissance rapide, on va pour cela calculer partiellement ses dérivées, on va plus précisément montrer, par récurrence que 
$$\forall n\in\N^*,\ \exists \ens{P_i,\ i\in[\![1,n]\!]}\in\R[X]\ |\ \forall y\in\R,\ \hat f^{(n)}(y)=\frac{\sum_{k=0}^n \hat g^{(k)}(y)P_k(y)}{\Par{\Par{2i\pi y }^{12}+\Par{2i\pi y }^{8}+1}^{2n}}$$

Pour simplifier les notation, on va noter le dénominateur $u$, ainsi pour le cas $n=1$, c'est la formule de la dérivée d'un quotient et on a 
$$\forall y\in\R,\ \hat f'(y)=\frac{\hat g'(y)v(y)-\hat g(y)v'(y)}{v(y)^2}$$
(il est claire que $v'$ est un polynôme) et pour le cas $n\to n+1$, on va dériver
\begin{align*}
	\hat f^{(n+1)}(y)&=\Par{\frac{\sum_{k=0}^n \hat g^{(k)}(y)P_k(y)}{v(y)^{2n}}}'\\
	&=\frac{\Par{\sum_{k=0}^n \hat g^{(k)}(y)P_k(y)}'v(y)-\sum_{k=1}^n \hat g^{(k)}(y)P_k(y)v'(y)}{v(y)^{2(n+1)}}\\
	&=\frac{\Par{\sum_{k=0}^n \hat g^{(k+1)}(y)P_k(y)+\hat g^{(k)}(y)P'_k(y)}v(y)-\sum_{k=1}^n \hat g^{(k)}(y)P_k(y)v'(y)}{v(y)^{2(n+1)}}\\
	\hat f^{(n+1)}(y)&=\frac{\sum_{k=0}^n \hat g^{(k+1)}(y)P_k(y)v(y)-\hat g^{(k)}(y)\Par{P_k(y)v'(y)-P'_k(y)v(y)}}{v(y)^{2(n+1)}}\\
\end{align*}
et je n'ai pas la place (en largeur) pour développer complètement cette expression, on va donc juste "constater" que les $P_k, k\in[\![0,n+1]\!]$ existent.
Et puisque $\hat g$ est dans $\mathcal S$, on peut majorer tous les $\hat g^{(k)}P_k$ par un même $c>0$, donc finalement 
$$\forall y\in\R,\ f^{(n)}(y)<\frac{nc}{{\Par{2i\pi y}^{12}+\Par{2i\pi y}^{8}+1}}\longrightarrow 0$$
ce qui montre que l'expression de $f$ trouvée est bien dans $\mathcal S$, on peut alors appliquer la formule d'inversion et 
$$\forall x\in\R,\ f(x)=\mathcal F^{-1}\Par{\frac{\hat g(y)}{\Par{2i\pi y}^{12}+\Par{2i\pi y}^{8}+1}}(x).$$

\end{document}
