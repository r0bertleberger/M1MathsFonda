\documentclass[a4paper,11pt, twoside]{article}
\title{}
\author{Raphaël Casanova}

\usepackage{packageTest}


\begin{document}

\pagestyle{empty}
\begin{center}
{\color{white} test}\\
\vspace{5cm}
{\bf \Huge {\YUGE A}NALYSE}\\[1em]
Raphaël Casanova\\
\href{mailto:raphael.casanova@centrale.centralelille.fr}{raphael.casanova@centrale.centralelille.fr}\\[2em]
\emph{D'après le cours d'Emmanuel Fricain à l'université de Lille}\\
\vspace{10cm}
\includegraphics[width=1\textwidth]{lille.pdf}
\end{center}


\newpage


\pagestyle{pageGarde}


\tableofcontents


\newpage


Ce document est une transciption en \LaTeX\ plus ou moins fidèle au cours d'analyse d'Emmanuel Fricain en première année de master, en 2024-2025.\\
En terme de notation, j'ai préféré utiliser les notations en {\bf gras} pour les ensembles : 
$$\N,\mathbf Z,\mathbf Q,\R,\mathbf C$$ plutôt que celles en : $$\mathbb N,\mathbb Z, \mathbb Q,\mathbb R,\mathbb C$$ pour coller à la norme ISO 80000-2.\\
La notation 
$$f_n\ttend f$$
indique que la suite de fonction $\left(f_n\right)$ converge uniformément vers $f$.\\
Dans le présent document, $\K=\R$ ou $\mathbf C$, $(K,d)$ est un espace métrique compact et l'on munit l'espace 
$$C(K,\K)=\left\{f:K\longrightarrow\K\text{ continues }\right\}$$
de la norme 
$$||f||_{\infty}:=\mathrm{sup}\left\{f(x)|,\ x\in K\right\}.$$


\newpage


\pagestyle{TestPage}

\NouvellePart{Espace des fonctions continues sur un compact}


On commence par (re)voir quelques notions de topologies qui vont nous servir au cours de ce chapitre :

\NouvelleSubsection{3 notions importantes}


$\bullet\quad$ densité : si $(E,d)$ est un espace métrique, $A\subset E$ est dite \emph{dense dans }$E$ si, et seulement si 
$$\forall x\in E,\ \exists \left(a_n\right)\in A\ \mathrm{t.q.}\ d(a_n,x)\tend 0.$$

$\bullet\quad$ compacité : l'espace métrique $(E,d)$ est \emph{compact} si, et seulement si toute suite dans $E$ admet une sous-suite convergente.\\
C'est équivalent à : pour tout recouvrement de $E$ par une collection quelconque d'ouverts $\left(V_i\right)_{i\in I}$, il existe un sous-recouvrement fini, \emph{i.e.} 
$$E=\bigcup_{k=1}^nV_k.$$

$\bullet\quad$ complétude :  l'espace métrique $(E,d)$ est \emph{complet} si toute suite de Cauchy à valeur dans $E$ converge dans $E$.\\

Dans le contexte d'espaces de fonctions,\\
la densité s'identifie avec une approximation par des fonctions régulières et la compacité / complétude s'identifie avec l'existence de limites et de valeurs d'adhérences de suites de fonctions.\\


\setcounter{CompteurRemarque}{0}
\begin{RQ}
  on montre que $\left(C(K,\K),||\cdot||_{\infty}\right)$ est un espace de Banach, $i.e;$ un espace vectoriel normé complet.
\end{RQ} 


\NouvelleSubsection{Théorème de Dini}


On sait, depuis la \emph{spé} que la convergence uniforme implique la convergence simple, et qu'il n'y a \emph{à priori} pas de réciproque (un contre-exemple est $f_n :t\in ]0,1[ \longmapsto t^n$, qui converge simplement vers la fonction nulle, mais dont la norme infinie est constante égale à 1).\\
Le théorème suivant nous donne un critère de réciproque : 


\begin{thC}[Dini]
  soit $\left(f_n\right)\in C(K,\R)$, avec $K$ un compact, on suppose\\
  $(\bullet)\quad f_n$ converge simplement vers $f:K\longrightarrow\R$,\\
  $(\bullet\bullet)\quad f$ est continue,\\
  $(\bullet\bullet\bullet)\quad \forall x\in K,\ \left(f_n(x)\right)$ est croissante,\\
  alors $f_n$ converge uniformément vers $f$.
\end{thC}


\begin{Proof}
  Soit $\varepsilon >0$, pour tout $n\geqslant 1$, on pose 
  $$\O_n:=\left\{x\in K \ |\ f_n(x)>f(x)-\varepsilon\right\},$$
  chaque $\O_n$ est un ouvert car c'est la pré-image d'un ouvert de $\R$ par une application continue.\\
  On a aussi 
  $$K=\bigcup_{n\geqslant 1}\O_n$$
  $K$ étant compact, on peut re-numéroter les $\O_n$ de façon à avoir 
  $$K=\bigcup_{i=1}^N\O_{N_i}.$$
  Quitte à réarranger la suite $(n_i)$, on la suppose croissante. Puisque chaque $\left(f_n(x)\right)$ est croissante, on a aussi les inclusions 
  $$\forall i\in \llbracket 1,N\rrbracket,\ \O_{n_i}\subset\O_{n_N}$$
  donc 
  $$K=\O_{m_n}.$$
  Autrement dit,
  $$\exists N\in\N \ |\ \forall x\in K,\ f_n(x)> f(x) - \varepsilon.$$
  Puisque $f_n(x)$ tend par valeurs inférieurs vers $f(x)$, on a aussi l'inégalité 
  $$\forall x\in K,\ f(x)-f_n(x)>0.$$
  On peut donc écrire 
  $$\exists N\in\N \ |\ \forall x\in K, |f_n(x) - f(x)| < \varepsilon.$$
  Et comme la suite $\left(f_n(x)\right)$ est croissante et converge vers $f(x)$, on a finalement
  $$\forall n\geqslant N,\ \forall x\in K,\ |f_n(x)-f(x)|< \varepsilon,$$
  et $f_n$ converge uniformément vers $f$.
\end{Proof}


$\triangleright$\emph{ex :} si l'on pose 
$$\left\{\begin{array}{rcl}
P_0(t)&=&0\\
P_{n+1}(t)&=&P_n(t)+\dstyle\frac12\big(t-P^2_n(t)\big)
\end{array}\right.$$
alors $P_n$ est une suite de polynômes convergeant uniformément vers $t\in[0,1]\longmapsto \sqrt t.$\\
En effet, on montre par récurrence que 
$$\forall t\in[0,1],\ \forall n\in\N,\ 0\leqslant P_n(t)\leqslant \sqrt t$$
ce qui nous montre que la suite $P_n(t)$ est croissante et bornée, donc converge (vers $f$).\\
On passe à la limite dans la relation de récurrence 
$$f(t)=f(t)+\dstyle\frac12(t-f^2(t))$$
donc (puisqe $f\geqslant0$)
$$f(t)=\sqrt t.$$
Le théorème de Dini s'applique, et $P_n$ converge uniformément sur $[0,1]$ vers $t\mapsto\sqrt t.$


\NouvelleSubsection{Théorème(s) de Stone-Weierstra\ss}


On commence par une parenthèse algébrique : soit $\A$ un ensemble, muni des lois $(+,.,\times)$.\\
On dit que $\A$ est une $\emph{algèbre}$ si, et seulement si\\
$(\bullet)\quad (\A,+,.)$ est un espace vectoriel,\\
$(\bullet\bullet)\quad \times : \A\times\A\longrightarrow\A$ est bilinéaire.

$\A$ est \emph{unitaire} si il contient une élement neutre pour $+$.\\

Une partie $\mathcal B$ de l'algèbre $\A$ est une \emph{sous-algèbre de}$\A$ si, et seulement si\\
$(\bullet)\quad (\mathcal B,+,.)$ est un sous-espace vectoriel de $(\A,+,.)$,\\
$(\bullet\bullet)\quad \mathcal B$ est stable pour la loi $\times$.\\


La question à laquelle répond ce paragraphe est comme suit : si $\A$ est une sous-algèbre unitaire de $C(K,\K)$, à quelle condition est-ce-que $\A$ est dense dans $C(K,\K)$ ?\\[1em]

\setcounter{CompteurRemarque}{0}
\begin{RQ}
  si $\A$ est dense dans $C(K,\K)$, alors $\A$ sépare les points, \emph{i.e.}
  $$\forall (x,y)\in K,\ x\neq y,\ \exists f\in\A\ |\ f(x)\neq f(y).$$
\end{RQ}


\begin{Proof}
  soit $x\neq y$ dans $K$ et $g:z\longmapsto d(x,z)$, on a alors 
  $$g(x)=0\text{ et }g(y)>0$$
  $g$ est une distance donc est 1-lipschitzienne donc est continue. Soit $\varepsilon:=g(y)$, par densité de $\A$ il existe $f\in \A$ telle que 
  $$||f-g||_{\infty}<\dstyle\nicefrac{\varepsilon}{2}$$
  On a 
  $$|f(x)|=|f(x)-g(x)|\leqslant||f-g||_{\infty}<\dstyle\nicefrac{\varepsilon}{2}$$
  On a aussi (inégalité triangulaire)
  $$|g(y)|=|(g(y)-f(y))+f(y)|\leqslant|g(y)-f(y)|+|f(y)|$$
  d'où
  $$|f(y)|\geqslant |g(y)|-|g(y)-f(y)|>\varepsilon-\dstyle\nicefrac{\varepsilon}{2}=\dstyle\nicefrac{\varepsilon}{2}$$
  donc 
  $$|f(y)|>|f(x)|$$
  et $\A$ sépare les points.
\end{Proof}


Avant de passer aux théorèmes de Stone-Weierstra\ss, on montre quelques propriétés générales sur les sous-algèbres unitaires de $C(K,\K)$.


\begin{prop}
  Soit $\A$ une sous-algèbre unitaire de $C(K,\K)$, soient $f,g\in\A$, alors\\
  $(\mathit 1)\quad |f|\in\overline{\A}$,\\
  $(\mathit 2)\quad \mathrm{sup}(f,g)$ et $\mathrm{inf}(f,g)\in\overline{\A}$.
\end{prop}


\begin{Proof}[Démonstration du $(\mathit1)$]
  Si $f=0$, alors $f=|f|$ et le résultat est vrai, on peut donc supposer $f\neq 0$ et l'on pose 
  $$g:=\frac{f}{||f|_{\infty}}\in\A$$
  on a alors 
  $$0\geqslant g^2\geqslant 1$$
  on peut donc composer $g^2$ par la suite des $P_n$ définie précedemment, donc pour $\varepsilon>0$, on a 
  $$\exists N\in\N\ |\ \forall n\geqslant N,\ \underset{x\in K}{\mathrm{sup}}\left|P_n(g^2(x))-|g(x)|\right|\leqslant\varepsilon.$$
  $\A$ étant stable par $+$ et $\times$, $x\longmapsto P_n(g^2(x))\in\A$, donc l'inégalité nous indique que $|g|\in\overline{\A}$.\\
  $\overline{\A}$ étant aussi stable par multiplication par un scalaire, on en déduit que $|f|\in\overline{\A}$.
\end{Proof}


\begin{Proof}[Démonstration du $(\mathit1)$]
  C'est une conséquence immédiate du premier point, en écrivant nos fonctions sous les formes suivantes : 
  $$\mathrm{sup}(f,g)=\frac12\left(f+g+|f-g|\right)$$
  et
  $$\mathrm{inf}(f,g)=\frac12\left(f+g-|f-g|\right).$$
\end{Proof}


\begin{thC}[Stones-Weierstra\ss, cas réel]
  Soit $(K,d)$ un espace métrique compact, $\A$ une sous-algèbre unitaire de $C(K,\R)$ qui sépare les points,\\
  alors $\A$ est dense dans $C(K,\R).$
\end{thC}


avant de démontrer ce théorème, on aura besoin des trois lemmes suivants :


\begin{propC}[lemme 1]
  Soit $\A$ une sous-algèbre unitaire de $C(K,\R)$ séparant les points,\\
  alors $\forall (x,y)\in K$ distincts, $\forall (\alpha,\beta)\in\R$, il existe $f\in\A$ telle que 
  $$f(x)=\alpha\text{ et }f(y)=\beta.$$
\end{propC}


\begin{Proof}
  puisque $\A$ sépare les points, il existe $g\in\A$ telle que
  $$g(x)\neq g(y).$$
  Soit le système d'équations linéaires
  $$(S)\left\{\begin{array}{rcl}
    \lambda g(x)+\mu&=&\alpha\\
    \lambda g(y)+\mu&=&\beta
  \end{array}\right.$$
  Ce système a pour déterminant
  $$\det S=g(x)-g(y)\neq 0$$
  il est donc inversible et admet une solution.

  On vérifie que l'application 
  $$t\longmapsto \lambda g(t)+\mu$$
  convient et est dans $\A$.
\end{Proof}


\begin{propC}[lemme 2]
  Soit $\A$ une sous-algèbre unitaire de $C(K,\R)$ séparant les points, soient $\varphi\in C(K,\R)$ et $\varepsilon>0$,\\
  alors $\forall x\in K,\ \exists f_x\in\overline\A$ telle que 
  $$\left\{\begin{array}{l}
    f_x(x)=x\\
    \forall x\in K,\ f_x(z)>\varphi(x)-\varepsilon.
  \end{array}\right.$$
\end{propC}


\begin{Proof}
  Soit $x\in K$, on sait, d'après le lemme précédent que pour $y\in K$, il existe $f^y\in \A$ telle que 
  $$f^y(x)=\varphi(x)\text{ et }f^y(y)=\varphi(y).$$
  (on considère ici que $\varphi(x)\neq\varphi(y)$, si ce n'est pas le cas, on prend $f^y=\varphi(x)$)

  Par continuité de $f^y$ et de $\varphi$, il existe des voisinages $V^y$ de $y$ tels que 
  $$\forall z\in V^y,\ f^y(z)>\varphi(z)-\varepsilon.$$
  La famille $(V^y)_{y\in K}$ est un recouvrement par ouverts du compact $K$ donc il existe une numérotation des $y$ telle que 
  $$K=\bigcup_{i=1}^p V^{y_i}.$$
  Soit 
  $$f_x:=\mathrm{sup}\left\{f^{y_1},\cdots,f^{y_p}\right\}$$
  $f_x\in\overline\A$ et par définition des $f^y$, on a $f_x(x)=\varphi(x)$. De plus,
  $$\forall z\in K,\exists i\in\llbracket 1,p\rrbracket\ |\ z\in V^{y_p}$$
  donc
  $$f_x(z)\geqslant f^{y_i}(z)>\varphi(x)-\varepsilon.$$
\end{Proof}


\begin{propC}[lemme 3]
  Soit $\A$ une sous-algèbre unitaire de $C(K,\R)$ séparant les points, soit $\varphi\in\overline{\A}$ et $\varepsilon>0$,\\
  alors il existe $f\in\overline{\A}$ telle que $||f-\varphi||_{\infty}<\varepsilon.$
\end{propC}


\begin{Proof}
  Soit $x\in K$ et $f_x$ telle que décrite dans le lemme 2. Puisque $f_x$ et $\varphi$ sont continues, il existe un voisinage $V_x$ de $x$ tel que 
  $$\forall x\in V_x,\ f_x(z)<\varphi(z)+\varepsilon.$$
  La famille $(V_x)_{x\in K}$ est un recouvrement par ouverts du compact $K$, donc il existe une numérotation des $x$ telle que 
  $$K=\bigcup_{i=1}^m V_{x_i}.$$
  Soit 
  $$f:=\mathrm{int}\left\{f_{x_1},\cdots,f_{x_m}\right\}$$
  $f\in\overline{\overline{\A}}=\overline{\A}$ et comme dans la démonstration du lemme précédent, 
  $$\forall z\in K,\ f(z)<\varphi(z)+\varepsilon\qquad (i).$$
  Par ailleurs, tous les $f_x$ vérifient aussi
  $$\forall z\in K,\forall x_i\ |\ f_{x_i}(z)>\varphi(z)-\varepsilon$$
  chacun des $f_{x_i}$ est supérieur à $\varphi-\varepsilon$, par passage à l'inf on a 
  $$\forall z\in K,\ f(z)>\varphi-\varepsilon\qquad (ii).$$
  En combinant $(i)$ et $(ii)$, on obtient l'encadrement suivant :
  $$\forall z\in K,\ \varphi(z)-\varepsilon<f(z)<\varphi(z)+\varepsilon$$
  autrement dit 
  $$||f-\varphi||_{\infty}<\varepsilon.$$
\end{Proof}


On peut enfin prouver le théorème :


\begin{Proof}
  Soient $\varphi\in C(K,\R)$ et $\varepsilon>0$, d'après le lemme 3, il existe $f\in \overline{\A}$ telle que 
  $$||f-\varphi||<\nicefrac{\varepsilon}{2}.$$
  Puisque $f\in\overline{\A}$, il existe $g\in\A$ telle que 
  $$||fèg||<\nicefrac{\varepsilon}{2}$$
  donc
  \begin{flalign*}
    ||\varphi-g||&=||(\varphi-f)+(f-g)||\\
    &\leqslant ||\varphi-f||+||f-g||\\
    &\leqslant \varepsilon
  \end{flalign*}
  et $\A$ est bien dense dans $C(K,\R)$.
\end{Proof}


\begin{propC}(corollaire]
  Soient $a<b\in\R$, 
  
  alors $\R[X]$ est dense dans $C([a,b],\R)$.
\end{propC}
 

\begin{Proof}
  Vérifier que $\R[X]$ est une sous-algèbre unitaire de $C([a,b],\R)$ qui sépare les points (le polynôme identité convient pour la séparation).
\end{Proof}


On peut même expliciter un (il n'y a pas unicité de l'approximation) polynôme convenable, soit $f\in C([0,1],\R)$, alors 
$$B_n(f)(X)=\sum_{i=0}^nf\left(\frac{k}{n}\right)\binom{n}{k}X^i(1-X)^{n-i}$$
est une suite de polynômes convergeant uniformément sur $[0,1]$ vers $f$.\\[1em]


\dnote{ce résultat est faux sur tout $\R$ entier, puisque si il existe $P_n$ une suite de polynômes convergeant uniformément vers $f:\R\longrightarrow \R$ sur $\R$, alors $f$ est polynômiale.}


\begin{Proof}
  Puisque $P_n$ CVU vers $f$, il existe $N$ tel que 
  $$\forall n\geqslant N,\ ||P_N-f||_{\infty}\leqslant 1$$
  donc, par inégalité triangulaire,
  $$\forall n\geqslant N,\ ||P_N-P_n||_{\infty}=||(P_N-f)+(f-P_n)||_{\infty}\leqslant 2$$
  Les seuls polynômes bornés sont des constantes, donc 
  $$\forall n\geqslant N,\ \exists \lambda_n\ |\ Pn-P_N=\lambda_n.$$
  On évalue cette expression en 0 et on trouve que 
  $$\lambda_n \tend f(0) - P_N(0)=:\lambda_{\infty}.$$
  Donc $$P_n\tend P_N + \lambda_{\infty}$$
  et par unicité de la limite, 
  $$f=P_N+\lambda_{\infty}.$$
\end{Proof}


\dnote{ce résultat est aussi faux sur $\C$ si la sous-algèbre considérée n'est pas stable par conjugaison.}


\begin{Proof}
  On considère l'application $f:z\longmapsto \overline z$,\\
  $\C[X]$ est bien une sous-algèbre unitaire de $C(\mathbf U,\C)$ qui sépare les points, donc si le théorème de Stones-Weierstra\ss\ est vrai, alors $\C[X]$ est dense dans $C(\mathbf U,\C)$, alors il existe $P_n\in\C[X]$ convergeant uniformément vers $f$.

  On a 
  $$\forall n\in\N,\ \int_0^{2\pi}\mathrm e^{int}\mathrm e^{it}\mathrm dt=0$$
  donc par linéarité de l'intégrale,
  $$\forall n\in\N,\ \int_0^{2\pi}P_n\left(\mathrm e^{it}\right)\mathrm e^{it}\mathrm dt=0$$
  par convergeance uniforme des $P_n$ vers $f$, on a aussi
  $$\int_0^{2\pi}f\left(\mathrm e^{it}\right)\mathrm e^{it}\mathrm dt=0$$
  or on peut calculer
  $$\int_0^{2\pi}f\left(\mathrm e^{it}\right)\mathrm e^{it}\mathrm dt=\int_0^{2\pi}\mathrm e^{-it}\mathrm e^{it}\mathrm dt=\int_0^{2\pi}1\mathrm dt=2\pi.$$
  c'et absurde donc le théorème est effectivement faux sur $\C$.
\end{Proof}

\begin{Def}
  Une partie $P$ de $\C$ est dite \emph{stable par conjugaison} si, et seulement si
  $$ x\in E\Rightarrow \overline x\in E.$$
\end{Def}

\begin{thC}[Stones-Weierstra\ss, cas complexe]
  Soit $(K,d)$ un espace métrique compact, $\A$ une sous-algèbre unitaire de $C(K,\C)$ qui sépare les points et est stable par conjugaison,\\
  alors $\A$ est dense dans $C(K,\C).$
\end{thC}

\begin{Proof}
  Soit $f\in C(K,\C)$, on pose 
  $$\A_{\R}:=A\cap C(K,\R),$$
  c'est une sous-algèbre unitaire de $C(K,\R)$ et, grâce à la stabilité par conjugaison, 
  $$\Re\ f=\frac{f+\overline f}2\text{ et }\Im\ f=\frac{f-\overline f}2$$
  sont tous deux dans $\A_{\R}$.

  Vérifions que $\A_{\R}$ sépare les points, soit $(x,y)\in \C$ distincts, puisque $\A$ sépare les points, il existe $\varphi\in \A$ telle que $\varphi(x)\neq \varphi(y)$, donc soit
  $$\Re \varphi(x)\neq \Re \varphi(y)$$
  soit 
  $$\Im \varphi(x)\neq \Im \varphi(y)$$
  et $\A_{\R}$ sépare aussi les points.
  On applique donc le théorème de Stone-Weierstra\ss, cas réel à $\Re\ f$ et $\Im\ f$, on fixe $\varepsilon>0$ et il existe $(g,h)\in \A_{\R}$ tels que 
  $$\left\{\begin{array}{l}
    ||g-\Re\ f||_{\infty}<\nicefrac{\varepsilon}{2}\\
    ||h-\Im\ f||_{\infty}<\nicefrac{\varepsilon}{2}.
  \end{array}\right.$$
  On pose $$\varphi:=g+ih\in \A$$
  et on a alors
  \begin{align*}
    ||f-\psi||_{\infty}&=||\Re\ f+i\Im\ f-g-ih||_{\infty}\\
    &\leqslant ||g-\Re\ f||_{\infty}+||h-\Im\ f||_{\infty}\\
    &\leqslant \varepsilon\\
  \end{align*}
  et $\psi$ CVU vers $f$, donc $\A$ est dense dans $C(K,\C)$.
\end{Proof}


\begin{propC}[corollaire, Féjer]
  L'ensemble
  $$\left\{z\longmapsto z ^n,\ n\in\mathbf Z\right\}$$
  est dense dans $C(\mathbf U,\C)$
\end{propC}

\begin{Proof}
  On vérifie que le théorème de Stone-Weierstra\ss, cas complexe s'applique.
\end{Proof}

\NouvelleSubsection{Polynômes trigonométriques}

Un polynôme trigonométrique est une application de la forme
$$P:t\longmapsto\sum_{k\in\llbracket -N,N\rrbracket}c_k\mathrm e^{ikt}$$
où les $c_k$ sont des coefficents complexes.

En passant à la forme trigonométrique de $\mathrm e^{ikt}$, on peut ré-écrire $P$ comme
\begin{flalign*}
  P(t)&=\sum_{k\in\llbracket -N,N\rrbracket}c_k\left[\cos(kt)+i\sin(kt)\right]\\
  &=\sum_{k\in\llbracket -N,N\rrbracket} a_k\cos(kt)+b_k \sin(kt).
\end{flalign*}

\begin{prop}
  Soit $f:\R\longrightarrow\C$ continue et $2\pi$-périodique,

  alors il existe une suite de polynômes trigonométriques convergeant uniformément vers $f$.
\end{prop}

\begin{Proof}
  il suffit de constater que 
  $$C_{2\pi}(\R)\simeq C(\mathbf U)$$
  via l'application
  $$\varphi : f\in C_{2\pi}(\R)\longmapsto \tilde f(t):=f(\mathrm e^{it})\in C(\mathbf U).$$
\end{Proof}


\NouvelleSubsection{Espaces séparables}

\begin{Def}
  $(E,d)$ est dit séparable si, et seulement si il existe $A\subset E$ dénombrable et dense dans $E$.
\end{Def}

On rappelle que $A$ est dénombrable si, et seulement si $A$ est fini ou $A\simeq \N$.

$\triangleright$\emph{ex : }$\R$ est séparable car $\overline{\mathbf Q}=\R$ et $\mathbf Q\simeq \N^2\simeq \N$.

On montre (en TD) que tout espace vectoriel de dimension finie est séparable. On a même la propriété suivante, qui est une version un peu plus forte :


\begin{prop}
  Soit $(E,d)$ un espace vectoriel normé, on suppose qu'il existe une famille $(e_n)\in E$ telle que 
  $$\overline{\mathrm{Vect}(e_i,\ i\in \mathbf N)}=E$$
  alors $E$ est séparable.
\end{prop}


\begin{Proof}
  Idée de la preuve : il s'agira de montrer que 
  $$\left\{\sum{i\in I}\lambda_i e_i,\ \lambda_i\in\mathbf Q,\ I\subset\N\text{ finie}\right\}$$
  est dense et dénombrable.
\end{Proof}


\begin{prop}
  Soit $(E,d)$ et $(O_i)_{i\in I}$ une famille quelconque d'ouverts deux à deux disjoints, 

  alors si $E$ est séparable, $I$ est dénombrable.
\end{prop}


$\triangleright$\emph{ ex : }soit $1\leqslant p<\infty$, on rappelle que 
$$\ell^p:=\left\{x:\N\longrightarrow\C\ |\ \sum_{n\in\N}|x(n)|^p<\infty\right\}$$
muni de la norme
$$||\cdot||_p:x\in\ell^p\longmapsto\left(\sum_{n\in\N}|x(n)|^p\right)^{\nicefrac{1}{p}}$$
est un espace (vectoriel normé) séparable.


\begin{Proof}
  Soit $e_n=\mathbf 1_{\{n\}}=(0,\cdots,0,1,0,\cdots,0)$ où le $1$ est en $n$-ème position et soit $x_n\in\ell^p$, alors 
  $$\norm{u-\sum_{i=1}^Nx_ie_i}_p=\left(\sum_{i=N+1}^{\infty}|x_n|^p\right)^{\nicefrac{1}{p}}\underset{N\to+\infty}{\longrightarrow 0}$$
  ce qui montre que (puisque chaque $e_n$ est dans $\ell^p$
  $$\overline{\mathrm{Vect}(e_i,\ i\in \N)}=\ell^p$$
  donc $\ell^p$ est bien séparable.
\end{Proof}


\begin{lemme}
  Soit $(K,d)$ compact,

  alors $K$ est séparable.
\end{lemme}

\begin{Proof}
  Soit $n\geqslant 1$, on a 
  $$K=\bigcup_{x\in K}B(x,\nicefrac{1}{n})$$
  par compacité de $K$, on peut numéroter ces boules donc 
  $$K=\bigcup_{i=1}^{N_n}B(x_j^n\nicefrac{1}{n})$$
  et l'ensemble 
  $$\mathcal D:=\left\{x_j^n,\ 1\leqslant j\leqslant N_n,\ n\in\N\right\}$$
  est dénombrable, montrons qu'il est dense,

  soit $x\in K$, alors 
  $$\forall n\in\N,\ x\in\bigcup_{i=1}^{N_n}B(x_i^n, \nicefrac{1}{n})$$
  autrement dit, 
  $$\forall n\in\N,\ \exists i_n\in\llbracket 1,N_n\rrbracket\ |\ x\in B(x_{i_n}^n,\nicefrac{1}{n})$$
  puisque le rayon de la boule tend vers $0$ quand $n\to+\infty$, on peut écrire 
  $$x_{i_n}^n\tend x$$
  ce qui montre que $\mathcal D$ est bien dense dans $K$, et $K$ est séparable.
\end{Proof}


\begin{lemme}
  Soit $(K,d)$ compact,

  alors $C(K,\R)$ est séparable.
\end{lemme}

\begin{Proof}
  $K$ est compact, donc d'après le lemme précédent, il existe $\{a_n\}$ une suite dense dans $K$. Pour $n\geqslant 1$, on pose 
  $$f_n:t\longmapsto d(a_n,t),$$
  les applications $f_n$ sont continues et on pose $f_0:=1$.

  On considère l'ensemble $\A$ définit comme suit :
  $$\A:=\mathrm{Vect}\left(\prod_{i\in I}f_i,\ I\text{ finie}\right)$$
  c'est une sous-algèbre unitaire et on va montrer que $\A$ sépare les points.

  Soient $(t_1,t_2)\in K$ distincts, alors $\varepsilon:=d(t_1,t_2)>0$. Par densité des $\{a_n\}$ dans $K$, il existe $n\in\N$ tel que 
  $$d(t_1,a_n)<\nicefrac{\varepsilon}{2}$$
  donc 
  $$f_n(t_1)=d(a_n,t_1)<\nicefrac{\varepsilon}{2}$$
  et de plus,
  \begin{align*}
    f_n(t_2)&=d(t_2,a_n)\\
    &\geqslant d(t_2,t_1)-d(t_1,a_n)\\
    &>\varepsilon-\nicefrac{\varepsilon}{2}\\
    f_n(t_2)&>\nicefrac{\varepsilon}{2}
  \end{align*}
  donc $\A$ sépare les points, ce qui nous permet d'appliquer le théorème de Stone-Weierstra\ss, donc $\A$ est dense dans $C(K,\R)$, donc $C(K,\R)$ est séparable.
\end{Proof}


\NouvelleSubsection{Théorème d'Ascoli}


Le fil directeur de cette partie est la question suivante : peut-on caractériser les partie compacte de $C(K)$ ?


\begin{Def}
  Soit $\mathcal F\in C(K)$,

  $\mathcal F$ est dite \emph{équicontinue en }$x$ si, et seulement si
  $$\forall \varepsilon>0, \exists\delta>0,\ \forall f\in\mathcal F, \forall y\in B(x,\delta)\cap K,\ |f(x)-f(y)|<\varepsilon.$$
  $\mathcal F$ est dite \emph{équicontinue} si, et seulement si $\mathcal F$ est équicontinue en tout point de $K$.
\end{Def}

$\triangleright$\emph{ex : }pour $\ell>0$, l'ensemble des fonctions $\ell$-lipschitzienne forme une famille équicontinue.


\begin{thC}[Ascoli, v1]
  Soient $(K,d)$ un compact et $\mathcal F=\{f_n\}$ une famille dans $C(K)$, si,

  $(\bullet)\ \forall x\in K,\ \big(f_n(x)\big)$ est une suite bornée de $K$,\\
  $(\bullet\bullet)$ $\mathcal F$ est équicontinue,

  alors il existe une sous-suite des $f_n$ qui converge uniformément dans $C(K).$
\end{thC}


\begin{Proof}
  Puisque $K$ est un compact, il est séparable et il existe une famille $\D=\{a_n\}$ dense dans $K$. 
  
  Par hypothèse, la suite $\left(f_n(a_0)\right)$ est bornée, donc on peut en extraire une sous-suite convergente, donc il existe $\varphi_0:\N\longrightarrow\N$ strictement croissante telle que $\left(f_{\varphi_0(n)}(a_0)\right)_n$ converge.

  De même, la suite $\left(f_{\varphi_0(n)}(a_1)\right)_n$ est bornée donc admet un extractrice $\varphi_1$ telle que $\left(f_{\varphi_0\circ\varphi_1(n)}(a_1)\right)_n$ converge.

  Par récurrence, on va construire les itérations succssives de $\varphi_n$ comme suit, autrement dit 
  $$\forall k\in\N,\ \left(f_{\varphi_0\circ\cdots\circ\varphi_k(n)}(a_k)\right)_n\text{ converge}$$
  Soit $\varphi:\N\longrightarrow\N$ telle que 
  $$\forall n\in\N,\ \varphi(n)=\varphi_0\circ\cdots\circ\varphi_n(n).$$
    Vérifions, par récurrence, que $\varphi$ est strictement croissante. 
    
    $\star$ pour $n=1$, on a (par croissante stricte de $\varphi_1$ et $\varphi_0)\ \varphi(1)=\varphi_1(\varphi_0(0))>\varphi(0).$

    $\star$ Soit $n\in\N$,
    $$\varphi(n+1)=\varphi_{n+1}(\varphi_0\circ\cdots\circ\varphi_n(n+1))$$
    et puisque $\varphi_0\circ\cdots\circ\varphi_n$ est strictement croissante, $\varphi_0\circ\cdots\circ\varphi_n(n+1)>\varphi_0\circ\cdots\circ\varphi_n(n)$ et en composant par $\varphi_{n+1}$, on trouve bien
    $$\varphi(n+1)>\varphi(n)$$
    ce qui conclut la récurrence.

    Donc $\left(f_{\varphi(n)}\right)$ est une sous-suite de $f_n$ telle que 
    $$\forall k\in\N,\ \left(f_{\varphi(n)}(a_k)\right)\text{ converge}$$
    puisque 
    $$\forall n\geqslant k,\ \left(f_{\varphi(n)}(a_k)\right)=\left(f_{\varphi_{k+1}\circ\cdots\circ\varphi_n(\varphi_1\circ\cdots\circ\varphi_k(n))}(a_k)\right)$$
    où la suite $\varphi_{k+1}\circ\cdots\circ\varphi_n$ est strictement croissante.

    Soit $\varepsilon>0$, par équicontinuité des $f_n$, pour tout $z\in K$, il existe $\delta_z>0$ tel que 
    $$\forall n\in \N,\ \forall u\in B(z,\delta_z),\ \left|f_{\varphi(n)}(u)-f_{\varphi(n)}(z)\right|\leqslant \varepsilon$$
    par inégalité triangulaire,
    $$\forall (u,v)\in B(z,\delta_z), \left|f_{\varphi(n)}(u)-f_{\varphi(n)}(v)\right|\leqslant \varepsilon\qquad (i).$$
    On peut écrire $K$ comme 
    $$K=\bigcup_{z\in K}B(z,\delta_z)$$
    par compacité de $K$, on a 
    $$K=\bigcup_{j=1}^NB(z_j,\delta{z_j}).$$
    Par densité de $\D$ dans $K$, pour tout $j\in\llbracket 1,n\rrbracket$, il existe un $a_{n_j}\in\D$ tel que 
    $$a_n\in B(z_j,\delta_{z_j})$$
    où $a_n$ est tel que $\left(f_{\varphi(n)}(a_{n_j})\right)_{n\geqslant 0}$ converge donc est de Cauchy, donc $\exists N_j$ tel que 
    $$\forall p,q\geqslant N_j,\ \left|f_{\varphi(q)}(a_{n_j})-f_{\varphi(p)}(a_{n_j})\right|\leqslant\varepsilon\qquad (ii).$$
    Soit $x\in K$, $\exists z_j\in K$ tel que $x\in B(z_j,\delta_{z_j})$ et $\exists a_{n_j}\in\D$ tel que $a_{n_j}\in B(z_j,\delta_{z_j})$, donc les hypothèses de l'inégalité $(i)$ sont valides et
    $$\forall n\geqslant 1, \left|f_{\varphi(n)}(x)-f_{\varphi(n)}(a_{n_j})\right|\leqslant 2\varepsilon$$
    pour $p,q\geqslant\mathrm{max}\{N_j,\ j\in\llbracket 1,N\rrbracket\}$
    \begin{align*}
      \left|f_{\varphi(q)}(x)-f_{\varphi(p)}(x)\right|&\leqslant \underbrace{\left|f_{\varphi(q)}(x)-f_{\varphi(q)}(a_{n_j})\right|}_{\leqslant 2\varepsilon\text{ d'après }(i)}+\underbrace{\left|f_{\varphi(q)}(a_{n_j})-f_{\varphi(p)}(a_{n_j})\right|}_{\leqslant \varepsilon\text{ d'après }(ii)}+\underbrace{\left|f_{\varphi(q)}(a_{n_j})-f_{\varphi(p)}(x)\right|}_{\leqslant 2\varepsilon\text{ d'après }(i)}\\
      &\leqslant 2\varepsilon+\varepsilon+2\varepsilon\\
      \left|f_{\varphi(q)}(x)-f_{\varphi(p)}(x)\right|&\leqslant 5\varepsilon
    \end{align*}
    par passage au $\mathrm{\sup}$, $||f_{\varphi(q)}-f_{\varphi(p)}||\leqslant 5\varepsilon$, donc la suite $f_{\varphi(n)}$ est de Cauchy à valeurs dans le complet $C(K)$, donc est convergente.
\end{Proof}

\begin{Def}
  Une partie $A\subset X$ de l'espace topologique $X$ est \emph{relativement compacte} si, et seulement si elle est inclue dans une partie compacte de $X$.

  Si $X$ est séparé, alors $A$ est relativement compacte si, et seulement si $\overline A$ est compacte.
\end{Def}


\begin{thC}[Ascoli, v2]
  Soit $(K,d)$ compact,

  $(\mathit 1)\quad$ les partie compactes de $C(K)$ sont exactement les parties fermées, bornées et équicontinues de $C(K)$,

  $(\mathit 2)\quad$ les parties relativement compactes de $C(K)$ sont exactement les parties bornées et équicontinues.
\end{thC}


\begin{Proof}
  $\star$ Soit $\mathcal F\subset C(K)$ compacte fermée et bornée, montrons que cette famille est équicontinue. 
  
  Soit $\varepsilon>0$,
  \begin{align*}
    K&=\bigcup_{f\in\mathcal F}B(f,\varepsilon)\\
    &=\bigcup{i=1}^NB(f_n,\varepsilon)
  \end{align*}
  Pour $i\in\llbracket 1,N\rrbracket$, pour $z\in K$, par continuité de $f_i$, $\exists \delta_{z,i}>0$ tel que 
  $$\forall u \in B(z,\delta_{z,i}),\ \left|f_i'u)-f_i(z)\right|<\varepsilon\qquad (i)$$
  On pose 
  $$\delta_z:=\mathrm{min}\{\delta_{z,i},\ i\in\llbracket 1,N\rrbracket\}>0.$$
  Soit $f\in\mathcal F$, donc 
  $$\exists i\in \llbracket 1,N\rrbracket\text{ t.q. }||f-f_i||_{\infty}<\varepsilon\qquad (ii)$$
  soit $i\in \llbracket 1,N\rrbracket$, $u\in B(z,\delta_z)\subset B(z,\delta_{z,i})$, on a 
  \begin{align*}
    |f(u)-f(z)|&\underbrace{\leqslant|f(u)-f_i(u)|}_{\leqslant\varepsilon\text{ d'après }(i)}+\underbrace{|f_i(u)-f_i(z)|}_{\leqslant\varepsilon\text{ d'après }(2)}+\underbrace{|f_i(z)-f(z)|}_{\leqslant\varepsilon\text{ d'après }(i)}\\
  &\leqslant 3\varepsilon
  \end{align*}
  donc $\mathcal  F$ est équicontinue.\\

  $\star$ Réciproquement, si on prend $F\subset C(K)$ fermée bornée équicontinue, montrons que $\mathcal F$ est compacte.
  
  Soit $(f_n)\in \mathcal F$, c'est une famille bornée et équicontinue donc d'après le théorème d'Ascoli, elle admet une sous-suite convergente, qui appartient à $\mathcal F$ car c'est un fermé, donc $F$ vérifie la propriété de Weierstra\ss, c'est un compact.
\end{Proof}


\begin{Proof}
  Se déduit de la proposition précédente en considérant la fermeture de $\mathcal F$.
\end{Proof}



\begin{thC}[Ascoli, v3]
  Soit $(E,d)$ séparable, $(F,\delta)$ métrique et $f_n\in C(E,F)$, si

  $(\bullet)\ \forall x\in E,\ \left\{f_n(x)\right\}_{n\geqslant 0}$ est relativement compacte dans $F$,\\
  $(\bullet\bullet)\ \left\{f_n\right\}$ est une famille équicontinue,

  alors $f_n$ admet une sous-suite qui converge uniformément sur tout compact de $E$ vers une application continue.
\end{thC}


Le théorème suivant découle du théorème d'Ascoli, et est quant à lui censé avoir plein d'applications :

\begin{thC}[Ascoli-Arzèla-Peano]
  soit $t_0\in\R,\ x_0\in\R^n, (a,r)>0$, on pose 
  $$K:=[t_0-a,t_0+a]\times \overline{B(x_0,r)}$$
  (qui est compact car fermé borné de $R^{n+1}$)\\
  on prend $f:K\longrightarrow \R^n$ continue, on pose
  $$M:=\mathrm{sup}_{(x,t)\in K}||f(x,t)||\text{ et }c:=\mathrm{min}(a,\nicefrac{a}{r}).$$
  Enfin on considère le problème de Cauchy suivant 
  $$\left\{\begin{array}{rcl}
    x'(t)&=&f(t,x(t))\\
    x(t_0)&=&x_0
  \end{array}\right.$$
  alors le problème admet une solution $x:[t_0-a,t_0+a]\longrightarrow \overline{B(x_0,r)}$
\end{thC}

\begin{Proof}
  Le principe va être de discrétiser $[t_0,t_0+c]$ puis de construire selon la méthode d'Euleur une suite de fonction qui vérifie cette discrétisation.

  Puis on appliquera le théorème d'Ascoli pour montrer que cette suite de fonction CVU vers une fonction, qui sera solution du problème.\\

  $\star$ On considère une subdivision de $[t_0,t_0+c]$, de pas $h:=\nicefrac{c}{n+1}$, donc $\forall i\in\llbracket 0,n\rrbracket,\ t_i=t_0+i\frac{c}{n+1}$. On a alors (méthode d'Euler)
  $$x(t_{i+1})=x(t_i)+\int_{t_i}^{t_{i+1}}f(t,x(t))\mathrm dt.$$
  On construit par récurrence les points $x_i$ tels que 
  $$x_{i+1}=x_i+\frac{c}{n+1}f(t,x_i)$$
  On montre alors 
  $$||x_i-x_0||\leqslant i\frac{cM}{n+1}.$$
  Pour $t\in[t_i,t_{i+1}]$, on pose 
  $$X_n(t):=a_it+b_i$$
  où les $a_i$ et $b_i$ sont tels que 
  $$\left\{\begin{array}{ccl}
    X_n(t_i)&=&x_i\\
    X_n(t_{i+1})&=&x_{i+1}
  \end{array}\right.$$
  Ainsi on a 
  $$\left\{\begin{array}{ccl}
    \displaystyle a_i&=&\frac{x_{i+1}-x_i}{t_{i+1}-t_i}\\
    b_i&=&x_i-a_i t_i
  \end{array}\right.$$
  Chaque $x_n$ est continue sur son $[t_0,t_0+c]$, et de pour tout $t\in [t_i,t_{i+1}]$, $x_n$ y est dérivable et 
  $$X'_n(t)=a_i=\frac{\nicefrac{c}{n+1}f(t_i,x_i)}{\nicefrac{c}{n+1}}=f(t_i,x_i)$$
  donc 
  $$\mathrm{sup}_{t\in]t_i,t_{i+1}[}||X'_n(t)||\leqslant M$$
  donc (inégalité des accroissements finis) les $(x_n)$ forment une famille $M$-lipschitzienne donc équicontinue.

  On a aussi
  \begin{align*}
    \forall n\in\N,\ t\in [t_0,t_0+c],\ ||X_n(t)-x_0||&=||X_n(t)-X_n(t_0)||\\
    &\leqslant M|t-t_0|\\
    &\leqslant \\
    ||X_n(t)-x_0||&\leqslant r\\
  \end{align*}
  Donc $X_n(t)\in\overline{B(x_0,r)}$ qui est un fermé bornée de $\R^n$, donc compact. Ainsi, $\forall t\in [t_0,t_0+c]$, $\left\{X_n(t)\right\}_{n\geqslant 0}$ est relativement compact dans $\R^n$.

  D'après le théorème d'Ascoli (quitte à considérer une sous-suite, ce qu'on ne fait pas pour alléger les notations), $(X_n)$ CVU vers $X:[t_0,t_0+c]\longrightarrow\overline{B(x_0,r)}$, où $X$ est continue.

  Soit $t\in[t_i,t_{i+1}]$,
  \begin{align*}
    ||X_n(t)-f(t,X_n(t))||&=||f(ti_,x_i)-f(t,X_n(t))||\\
    &\leqslant \omega \frac{(M+1)c}{n+1}
  \end{align*}
  où 
  $$\omega :=\mathrm{sup}\left\{||(t,x)-t(u,y),\ |t-u|<\delta, ||x-y||<\delta\right\}<\infty$$
  avec $\delta>0$
  Donc 
  $$X_n(t)-x_0-\int_{t_0}^tf(u,X_n(u))\mathrm du\leqslant \omega_fc\frac{M+1}{n+1}\tend 0$$
  donc 
  $$X(t)=x_0+\int_{t_0}^tf(u,X_n(u)\mathrm du$$
  autrement dit, $X$ est solution.
\end{Proof}


\newpage


\setcounter{prop}{0}


\NouvellePart{Théorèmes fondamentaux de l'analyse fonctionnelle}


\NouvelleSubsection{Théorème de Baire}

On rappelle le théorème suivant, déjà vu en topologie de L3 : 

\begin{thC}[des fermés emboités :]
  Soit $(E,d)$ un espace métrique complet, $F_n$ une suite de fermés non-vide de $E$, si
  $$\forall n\in\N,\ F_{n+1}\subset F_n$$
  et
  $$\mathrm{diam}\ F_n=\mathrm{sup}\left\{d(a,b),\ (a,b)\in\ F_n\right\}\tend 0,$$
  alors $\bigcap_{n\geqslant 0}F_n$ est un singleton.
\end{thC}

\begin{Proof}
  Pour $n\in\N$, soit $x(n)\in F_n$, pour $p<q$, on a $x_q\in F_q\subset F_p$ donc 
  $$d(x_p,x_q)\leqslant \mathrm{diam}\ F_q\leqslant \mathrm{diam}\ F_p\longrightarrow 0$$
  et les $x_n$ sont une suite de Cauchy dans $E$ qui est complet, donc $x_n\tend x\in E$.

  Pour tout $m\geqslant n$, $x_n\in F_m\subset F_n$ donc ($F_n$ étant fermé) $x\in F_n$ donc 
  $$x\in \bigcap_{n\geqslant 0}F_n.$$
  De plus, pour $(a,b)\in \bigcap F_n$, 
  $$d(a,b)\leqslant \mathrm{diam}\ F_n\tend 0$$
  donc $\bigcap_{n\geqslant 0} F_n$ est un singleton, ce qui conclut.
\end{Proof}

\begin{thC}[de Baire, v1]
  Soit $(E,d)$ métrique complet, soit $(O_n)$ une suite d'ouverts denses dans $E$,

  alors 
  $\bigcap_{n\geqslant 0}O_n$ est dense dans $E$.
\end{thC}

on a une autre version, où on parle de fermés :

\begin{thC}[de Baire, v2]
  Soit $(E,d)$ métrique complet, soit $(F_n)$ une suite de fermés d'intérieurs non-vides dans $E$,

  alors 
  $\bigcup_{n\geqslant 0}F_n$ est d'intérieur vide.
\end{thC}


\begin{Proof}
  Soit $O_n$ une suite d'ouverts, tous denses, soit $\Omega$ un ouvert de $E$, il s'agit de montrer que 
  $$\Omega\cap\bigcap_{n\geqslant 0}O_n\neq \emptyset.$$
  $O_0$ est un ouvert dense dans $E$, donc 
  $$\Omega\cap O_0$$
  est un ouvert non-vide et l'on peut prendre $x_0\in E,\ r_0\in ]0,1[$ tels que
  $$\overline{B(x_0,r_0)}\subset \Omega\cap O_0.$$
  $O_1$ est un ouvert dense dans $E$ donc 
  $$O_1\cap B(x_0,r_0)$$
  est un ouvert non-vide et l'on peut prendre $x_1\in E,\ r_1\in]0,\nicefrac12[$ tels que
  $$\overline{B(x_1,r_1)}\subset O_1\cap B(x_0,r_0)\subset \Omega\cap _0\cap O_1.$$
  On va construire par récurrence les $B_n=B(x_n,r_n)$ tels que
  $$\overline{B_{n+1}}\subset B_n\text{, }r_n\leqslant \nicefrac{1}{2^n}\text{ et }\overline{B_n}\subset \Omega\cap O_n.$$
  $\left(\overline{B_n}\right)$ est une suite décroissante de fermés dont le diamètre tend vers 0, donc le théorème des fermés emboîtés s'applique et $\bigcap_{n\geqslant 0}\overline{B_n}\neq\emptyset$ et puisque $\overline{B_n}\subset \Omega\cap O_n$, alors
  $$\Omega\cap\bigcap_{n\geqslant 0}O_n\neq\emptyset,$$
  autrement dit, $\bigcap_{n\geqslant 0}O_n$ est dense dans $E$.
\end{Proof}


On passe maintenant à la v2, \emph{i.e.} la formulation avec des fermés. Il va suffire pour ça de "passer" des fermés à des ouverts, puis d'appliquer la première version du théorème.\\


\begin{Proof}
  Soit $F_n$ une suite de fermés d'intérieurs non vides, on pose 
  $$O_n:=E\backslash F_n$$
  qui est une suite d'ouverts et 
  $$\overline{O_n}=\overline{E\backslash F_n}=E\backslash \mathrm{int}\ {F_n}=E$$
  on peut donc appliquer la première version du théorème de Baire et $\bigcap_{n\geqslant 0}O_n$ est dense dans $E$ donc
  $$\mathrm{int}\ \bigcup_{n\geqslant 0}F_n=\mathrm{int}\ E\backslash \bigcap_{n\geqslant 0}O_n=E\backslash \overline{\bigcap_{n\geqslant 0}O_n}=\emptyset.$$
\end{Proof}


\setcounter{CompteurRemarque}{0}
\begin{RQ}
  si l'on écrit $\mathbf Q$ (qui est dénombrable) sous la forme 
  $$\mathbf Q=\left(r_n\right)_{n\in\mathbf N}$$
  et que l'on choisit comme collection d'ouverts les 
  $$O_n:=\R\backslash \{r_n\}$$
  qui sont tous des ouverts denses dans $\R$, on a alors
  \begin{align*}
    \bigcap_{n\geqslant 0}O_n&=\bigcap_{n\geqslant 0}R\backslash \{r_n\}\\
    &=\R\backslash \bigcup_{n\geqslant 0}\{r_n\}\\
    &=\R\backslash \mathbf Q
  \end{align*}
  qui n'est pas un ouvert.
\end{RQ}


\begin{RQ} 
  le théorème n'est pas valide pour une réunion non-dénombrable, si l'on pose par exemple
  $$O_x:=\R\backslash \{x\}$$
  alors 
  $$\bigcap_{x\in_R}O_x=\emptyset$$
  qui n'est pas dense dans $\R$.
\end{RQ}

\begin{RQ}
  petit rappel de topologie, pour $E$ un espace topologique et $X\subset E$, on a 
  $$\mathrm{int}\ E\backslash X = E\backslash \overline{X}.$$
\end{RQ}


\begin{Proof}
  Soit $x\in\mathrm{int}\ E\backslash X$, donc il existe un voisinage $V$ de $x$ tel que 
  $$x\in V\subset E\backslash X$$
  donc $V\cap X=\emptyset$, autrement dit $x\notin \overline X$, donc $x\in E\backslash \overline{X}$.\\
  Réciproquement, soit $x\in E\backslash \overline{X}$, puisque $\overline X$ est un fermé, $E\backslash\overline X$ est un ouvert donc il existe un voisinage $U$ de $x$ tel que 
  $$x\in V\subset E\backslash\overline X\subset E\backslash X$$
  donc $x$ est intérieur à $X\subset E\backslash X$, ce qui conclut.
\end{Proof}


\begin{propC}[un corollaire]
  Soit $(E,d)$ métrique complet et $\left(F_n\right)$ une suite de fermées tels que 
  $$\bigcup_{n\geqslant 0}F_n=E$$
  alors $\Omega:=\bigcup_{n\geqslant 0}\mathrm{int}\ F_n$ est un ouvert dense dans $E$, donc à fortiori il existe au moins un $\mathrm{int}\ {F_{n_0}}\neq\emptyset$.
\end{propC}

\begin{Proof}
  On a l'équivalence suivante 
  $$\Omega\text{ dense }\Leftrightarrow E\backslash \Omega\text{ d'intérieur vide}$$
  on remarque que 
  \begin{align*}
    E\backslash\Omega&=\bigcup{k\geqslant 0}F_n\backslash\bigcup_{n\geqslant 0}\mathrm{int}\ F_n\\
    &\subset \bigcup_{k\geqslant 0}F_k\backslash\mathrm{int}\ F_k.
  \end{align*}
  Et $\partial F_k=F_k\backslash\mathrm{int}\ F_k$ est un fermé d'intérieur vide, donc le théorème de Baire s'applique et $\bigcup{k\geqslant 0}\partial F_k$ est d'intérieur vide, donc $E\backslash\Omega$ est aussi d'intérieur vide, donc d'après l'équivalence du début, $\Omega$ est bien dense dans $E$.
\end{Proof}

$\triangleright$\emph{ ex :} soient $(E,d)$ complet, $f_n:E\longrightarrow \R$ une suite de fonctions continues et $f$ la limite simple des $f_n$, alors 
$$\mathrm{cont}\ f:=\left\{x\in E\ |\ f\text{ est continue en }x\right\}$$
est dense dans $E$.


\begin{Proof}
  Soit $n,k\in \N\times\N^*$, on définit
  \begin{align*}
    F_n^k:&=\left\{x\in E\text{ t.q. }\forall p,\geqslant n,\ |f_p(x)-f_q(x)<\nicefrac{1}{k}\right\}\\
    &=\bigcap_{p,q\geqslant n}\left(f_p-f_q\right)^{-1}\left(\left[-\nicefrac{1}{k},\nicefrac{1}{k}\right]\right)
  \end{align*}
  Chacun des $F_n^k$ est fermé en tant que pré-image d'un fermé par une application continue.
  
  De plus, pour $k\in\N^*$ et $x\in E$, $f_n(x)$ converge donc est de Cauchy donc appartient à au moins un $F_n^k$, ainsi $E=\bigcup_{n\geqslant 0}F_n^k$.

  D'après le corollaire du théorème de Baire, $O_k:=\bigcup_{n\geqslant 0}\mathrm{int}\ F_n^k$ est un ouvert dense dans $E$ donc, en y ré-applicant le théorème de Baire, 
  $$\Omega:=\bigcap_{k\geqslant 0}\bigcup_{n\geqslant 0}\mathrm{int}\ F_n^k$$
  est un dense dans $E$, montrons que $\Omega\subset\mathrm{cont}\ f$.

  Soit $x_0\in\Omega$ et $\varepsilon>0$. Soit $k\geqslant 1$ tel que $\nicefrac{1}{k}\leqslant \varepsilon$.

  $x_0\in \Omega$ donc en particulier $x_0\in O_k=\bigcup_{n\geqslant 0}\inté F_n^k$, donc il existe $n_0$ tel que $x_0\in\inté F_{n_0}^k$ et 
  $$\exists r>0\tq B(x_0,r)\subset\inté F_n^k\subset F_n^k$$
  donc
  $$\forall x\in B(x_0,r),\ \forall p\geqslant n_0,\ |f_p(x)-f_{n_0}(x)|\leqslant \nicefrac{1}{k}\geqslant \varepsilon.$$
  Puisque $p\geqslant n_0$, on peut le faire tendre vers $+\infty$ et 
  $$\forall x\in B(x_0,r),\ |f(x)-f_{n_0}(x)|\leqslant \varepsilon.$$
  Par inégalité triangulaire, pour tout $x\in B(x_0,r)$, on a 
  \begin{align*}
    |f(x)-f(x_0)|&=|\big(f(x)-f_{n_0}(x)\big)+\big(f_{n_0}(x_0)-f(x_0)\big)+\big(f_{n_0}(x)-f_{n_0}(x_0)\big)|\\ 
    &\leqslant 2\varepsilon + |f_{n_0}(x)-f_{n_0}(x_0)|
  \end{align*}
  $f_{n_0}$ est continue en $x_0$ donc il existe $\tilde{r}$ tel que 
  $$\forall x\in B(x_0,\tilde{r}),\ |f_{n_0}(x)-f_{n_0}(x_0)|\leqslant \varepsilon$$
  donc, en posant $R:=\mathrm{min}(r, \tilde r)$, on a 
  $$\forall x\in N(x_0,R),\ |f(x)-f(x_0)|\leqslant 3\varepsilon$$
  donc $f$ est continue en $x_0$, ainsi $\Omega$ est bien dense dans $E$, donc $\mathrm{cont}\ f\supset\Omega$ est dense dans $E$
\end{Proof}


\setcounter{CompteurRemarque}{0}

\begin{RQ}
  si $f:\R\longrightarrow \R$ est dérivable, alors $f'$ est continue sur un ensemble dense, puisque la suite de fonction 
  $$f_n(x)=\frac{1}{n}\left(f\left(x-\nicefrac{1}{n}\right)-f(x)\right)$$
  est une suite de fonctions continues qui CV vers $f'$, ainsi l'exemple s'applique.
\end{RQ}


\NouvelleSubsection{Quelques rappels sur les applications linéaires continues}


$X$ et $Y$ sont deux espaces vectoriels normés, $T:X\longrightarrow Y$ est une application linéaire.\\[1em]

$T$ est continue si, et seulement si $\exists c>0$ tel que
$$\forall x\in X,\ ||Tx||_Y\leqslant c||x||_X.$$

On note 
$$\mathcal L(X,Y):=\left\{T:X\longrightarrow Y\text{ linéaire continue}\right\}$$
Pour $T\in\mathcal L(X,Y)$, on note
$$||T||:=\mathrm{inf}\left\{c>0\ |\ \forall x\in X,\ ||Tx||\leqslant c||x||\right\}$$
on montre que cette définition est équivalente à
$$\begin{array}{rcl}
||T||&=&\mathrm{sup}\left\{||Tx||,\ x\in X\ |\ ||x||\leqslant 1\right\}\\
&=&\mathrm{sup}\left\{||Tx||,\ x\in X\ |\ ||x||= 1\right\}\\
&=&\mathrm{sup}\left\{\dstyle\frac{||Tx||}{||x||}, x\in X\backslash\{0\}\right\}
\end{array}$$
on montre aussi que cette norme est \emph{sous-multiplicative}, \emph{i.e.}
$$\forall T\in\mathcal L(Y,Z),\ S\in\mathcal L(X,Y),\ ||TS||\leqslant ||T||\times ||S||$$
on généralise pour les itérations successives de T,
$$\forall n\in\N,\ ||T^n||=||\underbrace{T\circ T\cdots\circ T}_{n\text{ fois}}||\leqslant ||T||^n.$$

\begin{prop}
  Soient $(X,Y)$ deux espaces vectoriels normés, où $Y$ est un Banach,\\
  alors $\mathcal L(X,Y)$ est un Banach (\emph{i.e.} un e.v.n. complet).
\end{prop}

\begin{Proof}
  Soit $(T_n)$ une suite de Cauchy dans $\mathcal L(X,Y)$, et soit $\varepsilon>0$,
  $$\exists n_0\in\N\text{ t.q. }\forall p,q\geqslant n_0,\ ||T_p-T_p||\leqslant \varepsilon$$
  soient $x\in X,\ p,q\geqslant n_0$
  $$||T_px-T_qx||=||\left(T_p-T_q\right) x||\leqslant ||T_p-T_q||\ ||x||\leqslant \varepsilon ||x||\qquad (i)$$
  donc $(T_n(x))$ est une suite de Cauchy à valeurs dans $Y$, donc elle est convergente et on peut poser
  $$T(x):=\underset{n\to +\infty}{\mathrm{lim}}T_n(x).$$
  Montrons que $T$ est bien linéaire, soient $(u,v)\in X,\ \lambda\in\K$,
  \begin{flalign*}
    T(\lambda u+v)&=\underset{n\to +\infty}{\mathrm{lim}}T_n(\lambda u+v)\\
    &=\underset{n\to +\infty}{\mathrm{lim}}\lambda T_n(u)+\underset{n\to +\infty}{\mathrm{lim}}T_n(v)\\
    &=\lambda\underset{n\to +\infty}{\mathrm{lim}} T_n(u)+\underset{n\to +\infty}{\mathrm{lim}}T_n(v)\\
    T(\lambda u+v)&=\lambda T(u)+T(v)
  \end{flalign*}
  donc $T$ est bien linéaire et pour montrer la continuité de $T$, on fait tendre $p\to +\infty$ dans $(i)$, on obtient alors
  $$||T-T_q||\ ||x||\leqslant \varepsilon||x||$$
  donc $T-T_p$ est continue, et puisque $T_q$ est continue, donc $T$ l'est et $T\in\mathcal L(X,Y)$.
\end{Proof}


\begin{Def}
  Soit $X,Y$ deux espaces vectoriels, un \emph{opérateur compact} est une application continue (un \emph{opérateur} continu donc) $T:X\longrightarrow Y$ tel que pour tout $A\in X$ borné, $T(A)$ est une partie relativement compacte de $Y$.

  Si la topologie sur $X$ est la topologie métrique habituelle, alors $T$ est compact si, et seulement si $T(B(0,1))$ est une partie relativement compacte.\\
\end{Def}


$\triangleright$\emph{ ex : }soient $-\infty<a<b<+\infty$ et $K:[a,b]\times[a,b]\longrightarrow \mathbf C$ continue. Pour $f\in C([a,b],\mathbf C)$ et $x\in[a,b]$, on pose 
$$T_K(f)(x):=\int_a^bK(x,y)f(y)\mathrm dy.$$
Vérifions que $T_K:C([a,b],\mathbf C)\longrightarrow C([a,b],\mathbf C)$

$(x,y)\longmapsto K(x,y)f(y)$ est continue et l'on intègre des fonctions continues sur un segment, donc $T_K(f)$ est bien continue. Par linéarité de l'intégrale, $T_K$ est aussi linéaire.

Soit $x\in [a,b]$,
$$|T_K(f)(x)|\leqslant ||K||_{\infty}(b-a)^2||f||_{\infty}=c||f||_{\infty}$$
avec $||f||_{\infty}<\infty$ car $f$ est continue sur un segment, donc $T_K$ est bien une application linéaire continue.

Montrons que $T_K$ est compact, \emph{i.e.} montrons que 
$$T_K(\overline{B(0,1)})$$
est relativement compact dans $C([a,b],\C)$.

On a
$$T_K(\overline{B(0,1)})=\left\{T_K(f),\ f\in C([a,b],\C)\text{ t.q.} ||f||_{\infty}\leqslant 1\right\}.$$
Montrons que la famille 
$$\left\{T_K(f),\ f\in\overline{B(0,1)}\right\}$$
est équicontinue, soit $\varepsilon>0$ et $x\in[a,b]$, alors pour tout $z\in[a,b]$
$$T_K(f)(x)-T_K(f)(z)=\int_a^b\left[K(x,y)-K(z,y)\right]f(y)\mathrm dy\qquad (i)$$
$K$ est continue sur le compact $[a,b]\times[a,b]$ donc $K$ est équicontinue et il existe $\delta>0$ tel que
$$\forall (x,z)\in[a,b]\times[a,b],\ |x-z|\leqslant\delta\Longrightarrow \forall y\in [a,b],\ |K(x,y)-K(z,y)|\leqslant \varepsilon$$
soit $y\in B(x,\delta)$, d'après $(i)$ on a 
\begin{flalign*}
  |T_K(f)(x)-T_K(f)(z)|&\leqslant \varepsilon \int_a^b|f(y)\mathrm dy\\
  &\leqslant \varepsilon ||f||_{\infty}(b-a)\\
  &\leqslant \varepsilon (b-a)
\end{flalign*}
on va donc change de $\delta$, on va prendre le $\tilde\delta$ définie à partir de $\nicefrac{\varepsilon}{b-a}$ et on a maintenant
$$\forall |x-a|\leqslant \tilde\delta,\ |T_K(f)(x)-T_K(f)(z)|\leqslant \varepsilon$$
donc la famille est bien équicontinue et on peut appliquer le théorème d'Ascoli, et 
$$T_K(\overline{B(0,1)})$$
est relativement compacte.

\begin{prop}
  Soient $X$ un espace vectoriel normé, $Y$ un espace de Banach et $E\subset X$ un sous-espace dense, soit $T\in \mathcal L(E,Y)$,

  alors il existe un unique $\tilde T\in \mathcal L(X,Y)$ tel que 
  $$\forall x\in E,\ \tilde Tx=Tx$$
  et plus, ces applications sont de normes égales, \emph{i.e.}
  $$||\tilde T||_{\mathcal L(X,Y)}=||T||_{\mathcal L(E,Y)}.$$
\end{prop}


\begin{Proof}
  La preuve est en trois partie; la construction de $\hat T$, s'assurer que cette application est continue puis s'assurer qu'elle est unique.

  $\star$ Pour la construction de $\hat T$, on considère $x\in X\backslash E$. Par densité il existe $e:\N\longrightarrow E$ telle que 
  $$e_n\tend e$$
  et pour $x\in E$, on considère la suite constante, donc pour tout $x\in X$, il existe une suite $e_n$ à valeur dans $E$ convergeant vers $x$.

  Montrons que $Te_n$ converge dans $Y$, on a 
  \begin{align*}
    ||Te_n-Te_m||&=||T(e_n-e_m)||\\
    &\leqslant ||T||\ ||e_n-e_m||
  \end{align*}
  la suite $e_n$ étant convergente, elle est de Cauchy donc $Te_n$ est aussi de Cauchy, et $Y$ étant un espace de Banach, cette suite est convergente et on peut poser, 
  $$\hat Tx:=\underset{n\to+\infty}{\mathrm{lim}}Te_n.$$
  Vérifions que $\hat T$ est bien définie, $\emph{i.e.}$ que la définition n'est pas fonction du choix de la suite convergeant vers $x\in X$. Soient $x\in X$ et $e_n,e_n'$ deux suites convergeants vers $x$
  $$||Te_n-Te_n'||\leqslant||t||\ ||e_n-e_n'||\tend0$$
  donc 
  $$Te_n-Te_n'\tend 0$$
  et $\hat Tx$ est correctement définie.

  $\star$ Pour la linéarité, soient $(x,y)\in X$, $\lambda\in\K$. On considère $y_n\longrightarrow y$ et $x_n\longrightarrow x$, alors
  \begin{align*}
    \hat T(\lambda x+y)&=\underset{n\to+\infty}{\mathrm{lim}}T(\lambda x_n+y_n)\\
    &=\lambda\underset{n\to+\infty}{\mathrm{lim}}Tx_n+\underset{n\to+\infty}{\mathrm{lim}}Ty_n\\
    \hat T(\lambda x+y)&=\lambda\hat Tx+\hat Ty
  \end{align*}
  et $\hat T$ est bien une application linéaire.

  $\star$ Pour la continuité de $\hat T$, soient $x\in X$ et $x_n\longrightarrow x$, on sait, puisque $T$ est continue, que 
  $$\forall n\in\N,\ ||Te_n||\leqslant ||T||\ ||e_n||$$
  par passage à la limite simple
  $$||\hat Tx||\leqslant ||T||\ ||x||$$
  donc $\hat T$ est continue, et on a même l'inégalité des normes $||\hat T||\leqslant ||T||$.

  $\star$ Quant aux normes, on a $\hat T\big|_E=T$ donc pour $e\in E$,
  $$||Te||=||\hat Te||\leqslant ||\hat T||\ ||e||$$
  donc $||T||\leqslant||\hat T||$ ainsi on a l'égalité des normes recherchée.

  $\star$ Pour l'unicité, on suppose qu'il existe $U$ distinct de $\hat T$ qui convient, alors $\forall x\in X$, il existe $x_n\longrightarrow x$ une suite d'élements  de $E$, alors
  \begin{align*}
    U(x)&=\underset{n\to+\infty}{\mathrm{lim}} Ue_n\\
    &=\underset{n\to+\infty}{\mathrm{lim}}Te_n\\
    &=\underset{n\to+\infty}{\mathrm{lim}}\hat Te_n\\
    Ux&=\hat Tx
  \end{align*}
  ainsi $U=\hat T$, donc $\hat T$ est effectivement unique.
\end{Proof}


\begin{thC}[Banach-Steinhaus]
  Soient $X$ un espace de Banach, $Y$ un espace vectoriel normé et $\left\{T_i,\ i\in I\right\}$ une suite quelconque d'application linéaires $X\longrightarrow Y$,
  
  alors exactement l'une des assertions suivantes est vraie\\
  
  $(i))$ $\underset{i\in I}{\mathrm{sup}}||T_i||<\infty$\\[1em]
  $(ii) $ $\big\{x\in X\text{ t.q. } \underset{i\in I}{\mathrm{sup}}||T_ix||=+\infty\big\}$ est dense dans $X$.
\end{thC}

\begin{Proof}
  On suppose que l'assertion $(ii)$ est fausse, donc que 
  $$\left\{x\in X\text{ t.q. } \mathrm{sup}_{i\in I}||T_ix||=+\infty\right\}$$ 
  n'est pas dense dans $X$, autrement dit son complémentaire, noté $F$ est d'intérieur non vide.

  Soit $n\geqslant 0$, on pose 
  \begin{align*}
    F_n:&=\left\{x\in X\text{ t.q. }\mathrm{sup}_{i\in I}||T_ix||\leqslant n\right\}\\
    &=\bigcap_{i\in I}T_i^{-1}\left(\overline{B(0,1)}\right)
  \end{align*}
  Les $F_n$ sont tous fermés car ils sont des intersection de pré-images de fermés par l'application continue $T_i$, donc le théorème de Baire s'applique et 
  $$\exists n_0\geqslant 0, x_0\in X, r>0\text{ t.q. }\overline{B(x_0,r)}\subset F_{n_0}$$
  pour $x\in X$ tel que $||x||\leqslant r$, on écrit
  $$x=\frac12\left((x+x_0)+(x-x_0)\right)$$
  donc, puisque $T_i$ est linéaire
  $$T_ix=\frac12T(x_0+x)-\frac12T(x_0-x)$$
  et puisque $x_0+r$ et $x_0-x$ sont tous deux dans $\overline{B(x_0,r)}$, 
  $$\left\{\begin{array}{l}
    T_i(x_0+x)\in F_{n_0}\\
    T_i(x_0-x)\in F_{n_0}
  \end{array}\right.$$
  d'où la majoration suivante 
  $$||T_ix||\leqslant n_0.$$
  Dans le cas général, pour $u\in X$, on pose $x:=r\frac{u}{||u||}$ (on suppose $u\neq 0$), alors $||x||\leqslant r$ et on est dans le cadre du cas précédent, 
  \begin{align*}
    ||T_iu||&=\norm{T_i\left(\frac{||u||}{r}x\right)}\\
    &=\frac{||u||}{r}\norm{T_i\left(x\right)}\\
    &\leqslant \frac{||u||}{r} n_0
  \end{align*}
  en faisant varier $||u||$ sur le disque unité, on a (passage au sup)
  $$||T_i||\leqslant \frac{n_0}{r}$$
  ainsi 
  $$\mathrm{sup}_{i\in I}||T_i||<\infty$$
  et l'on est bien dans le cas $\mathit 1$.
\end{Proof}

$\triangleright$\emph{ ex : }application au séries de Fourier, pour $f:[-\pi,\pi]\longrightarrow\mathbf C$ intégrable, on définit les coefficients de Fourier comme suit :
$$\hat f(k):=\frac{1}{2\pi}\int_{-\pi}^{\pi}f(t)\ex{-ikt}\mathrm dt$$
et la somme partielle de Fourier comme 
$$S_n(f)(t):=\sum_{k=-n}^n\hat f(k)\ex(ikt)$$

On va voir dans quel cas on a, pour $f\in C_{2\pi}(\R,\mathbf C)$,
$$S_nf\tend f\text{ CVS}$$

On montre d'abord que 
$$\left\{f\in C([-\pi,\pi])\ |\ \underset{n\geqslant 0}{\mathrm{sup}}|S_nf(0)|=+\infty\right\}$$
est dense dans $C([-\pi,\pi])$, muni de la norme $||\cdot||_{\infty}$.

\begin{Proof}
  Soit, pour $n\in \N$, 
  $$\ell_n: 
  \begin{array}{|rcl}
    \mathcal C[-\pi,\pi]&\longrightarrow&\R\\
    f&\longmapsto&S_nf(0)\\
  \end{array}$$
  $\ell_n$ est une application linéaire.

  $\star$ Montrons que $\ell_n$ est continue, soit $f\in\mathcal C[-\pi,\pi]$, on calcule 
  \begin{align*}
    \ell_n(f)&=\sum_{k=-n}^n\frac{1}{2_i}\int_{-\pi}^{\pi}f(t)\ex{-ikt}\dt\\
    \ell_n(f)&=\frac{1}{2\pi}\int_{-\pi}^{\pi}f(t)\sum_{k=-n}^n\ex{-ikt}\dt
  \end{align*}
  On définit 
  $$D_n(t):=\sum_{k=-n}^n\ex{-ikt}$$
  le noyau de Dirichlet, que l'on peut calculer
  $$D_n(t)=\left\{\begin{array}{rcr}
    &\dstyle\frac{\sin(\nicefrac{2n+1}{2}t)}{\sin(\nicefrac{t}{2})}&\text{ si }t\neq 0\\[1em]
    & 2n+1&\text{ si }t=0  
  \end{array}\right.$$
  Avec ces notations, on a 
  \begin{align*}
    |\ell_n(f)|&=\frac{1}{2\pi}\left|\int_{-\pi}^{\pi}f(t)D_n(t)\dt\right|\\
    &\leqslant ||f||_{\infty}\frac{1}{2\pi}\left|\int_{-\pi}^{\pi}D_n(t)\dt\right|\\
    |\ell_n(f)|&\leqslant ||f||_{\infty}\frac{1}{2\pi}\int_{-\pi}^{\pi}\left|D_n(t)\right|\dt\qquad (i)
  \end{align*}
  donc $\ell_n$ est continue.

  $\star$ Montrons que $\underset{n\geqslant 0}{\mathrm{sup}}\ ||\ell_n||=+\infty$

  On montre d'abord que $||\ell_n||=\dstyle\frac{1}{2\pi}\int_{-\pi}^{\pi}|D_n|\dt$. 
  
  $\star\star$ L'inégalité $(i)$ nous donne la moitié de l'égalité
  $$||\ell_n||\leqslant\frac{1}{2\pi}\int_{-\pi}^{\pi}\left|D_n(t)\right|\dt.$$
  $\star\star$ On va chercher à montrer l'autre sens de cette inégalité, soit $\varepsilon>0$ et 
  $$f_\varepsilon:t\longmapsto \frac{D_n(t)}{|D_n(t)|+\varepsilon}$$
  $\sin$ étant $2\pi$-périodique, $f_\varepsilon$ l'est aussi donc $f_\varepsilon\in\mathcal C[-\pi,\pi]$ et 
  $$\ell_n(f_\varepsilon)=\frac{1}{2\pi}\int_{-\pi}^{\pi}\frac{D_n^2(t)}{|D_n(t)|+\varepsilon}\dt$$
  De plus, 
  $$\underset{\varepsilon\to 0^+}{\mathrm{lim}}\frac{D_n^2(t)}{|D_n(t)|+\varepsilon}=|D_n(t)|$$
  et on peut dominer l'intégrande par $|D_n|$, qui est dans $\mathcal L_1$, donc on peut appliquer le théorème de convergence dominée et 
  $$\underset{\varepsilon\to 0^+}{\mathrm{lim}}\frac{1}{2\pi}\int_{-\pi}^{\pi}\frac{D_n^2(t)}{|D_n(t)|+\varepsilon}\dt=\frac{1}{2\pi}\int_{-\pi}^{\pi}|D_n(t)|\dt.$$
  De plus, $\forall \varepsilon>0,\ ||f_\varepsilon||_\infty\leqslant 1$ donc $||f_\varepsilon||_1\leqslant 1$ et $||\ell_n||\geqslant |\ell_n(f_\varepsilon)|$, donc par passage à la limite, 
  $$||\ell_n||\geqslant \frac{1}{2\pi}\int_{-\pi}^{\pi}|D_n(t)|\dt.$$
  Ainsi on a les deux majorations / minorations, donc 
  $$||\ell_n||= \frac{1}{2\pi}\int_{-\pi}^{\pi}|D_n(t)|\dt.$$
  On peut maintenant passer au calcul de $||\ell_n||$;
  \begin{align*}
    ||\ell_n||&=\frac{1}{\pi}\int_{-\pi}^{\pi}|D_n(t)|\dt\\
    &=\frac{1}{\pi}\int_{-\pi}^{\pi}\left|\frac{\sin\left(t\nicefrac{(2n+1)}{2}\right)}{\sin\left(\nicefrac{t}{2}\right)}\right|\dt\\
    &\geqslant\frac{1}{\pi}\int_{-\pi}^{\pi}\frac{\left|\sin\left(t\nicefrac{(2n+1)}{2}\right)\right|}{|t|}\dt\\
    &\geqslant\frac{2}{\pi}\int_0^{\pi}\frac{\left|\sin\left(t\nicefrac{(2n+1)}{2}\right)\right|}{|t|}\dt\\
    &\geqslant \frac{2}{\pi}\int_0^{\nicefrac{2n+1}{2}}\frac{|\sin u|}{\nicefrac{2u}{2n+1}}\frac{2\mathrm du}{2n+1}\\
    ||\ell_n||&\geqslant \frac{2}{\pi}\int_0^{\nicefrac{2n+1}{2}}\frac{|\sin u|}{u}\mathrm du\\
  \end{align*}
  Donc $||\ell_n||\tend \dstyle\int_{0}^{\infty}\frac{|\sin u|}{u}\mathrm du=+\infty$, donc $\underset{n\geqslant 0}{\mathrm{sup}}\ ||\ell_n||=+\infty$.
\end{Proof}


\begin{thC}[de majoration automatique]
  Soient $(X,Y)$ des espaces de Banach et $T\in\mathcal L(X,Y)$, si $T$ est surjective,

  alors $\forall \varepsilon>0$, $\exists c>0$ tel que $\forall y\in Y,\ \exists x\in X$ tel que
  $$y=Tx\text{ et }||x||\leqslant \varepsilon||y||=\varepsilon||Tx|$$
\end{thC}


\begin{Proof}
  A FAIRE
\end{Proof}


\begin{thC}[d'isomorphisme de Banach]
  Soient $X,Y$ est espaces de Banach et $T\in \mathcal L(X,Y)$, si $T$ est bijective, 

  alors $T^{-1}\in \mathcal L(Y,X)$.
\end{thC}

\begin{Proof}
  Soit $\varepsilon>0$, on sait d'après le théorème de majoration automatique q'il existe $c>0$ tel que $\forall y\in Y,\ \exists x\in X$ vérifiant $Tx=y$ et $||x||\leqslant \varepsilon ||Tx||$. On écrit alors $x=T^{-1}y$ et l'inégalité devient alors 
  $$||T^{-1}y||\leqslant \varepsilon ||y||$$
  ceci étant vrai pour tout $y\in Y$, $T^{-1}$ est continue.
\end{Proof}


\begin{thC}[de l'application ouverte]
  Soient $X,Y$ des espaces de Banach et $T\in\mathcal L(X,y)$, si $T$ est surjective,

  alors $T$ est ouverte, \emph{i.e.} $\forall O$ ouvert, $T(O)$ est un ouvert.
\end{thC}


\begin{Proof}
  On applique le théorème de majoration automatique, donc il existe $c>0$ tel que 
  $$B(0,1)\subset T\left(B(0,c)\right),$$
  ce qui conclut.
\end{Proof}


\begin{thC}
  Soient $X$ un espace vectoriel normé et $||\cdot||_1$, $||\cdot||_2$ deux normes sur $X$. On suppose que $X$ est complet pour les deux normes et qu'il existe $c>0$ tel que pour tout $x\in X$, $||x||_2\leqslant c||x||_1$, 

  alors $||\cdot||_1$ et $||\cdot||_2$ sont équivalentes.
\end{thC}


\begin{Proof}
  A FAIRE
\end{Proof}

\NouvelleSubsection{Théorème du graphe fermé}


\begin{prop}
  Soient $X,Y$ deux Banach, $T\in\mathcal L(X,Y)$, on suppose que pour $x_n\in X$, $y_n\in Y$ telles que $x_n\tend x\in X$, $y_n\tend y\in Y$, alors $Tx=y$,

  alors $T$ est continue
\end{prop}

\begin{Proof}
  A FAIRE
\end{Proof}


$\triangleright$\emph{ex :} application en analyse complexe, soit 
$$H^2:=\left\{f\in\mathrm{Hol}\ D(0,1),\ f(z)=\sum_{n\in\mathbf Z}a_n z^n \tq \sum_{n\in\mathbf Z}|a_n|^2=||f||_2^2<\infty\right\}$$
l'espace de Hardy,

alors $H^2$ est un espace de Banach.


\begin{Proof}
  A FAIRE
\end{Proof}


\NouvelleSubsection{Théorème de Hann-Banach}

Soit $E$ un espace vectoriel normé, on note $E^*$ ou $E'$ l'ensemble des formes linéaires continues
$$E^*\left\{\varphi:E\longrightarrow \R\text{ linéaire continue }\right\},$$ 
muni de la norme d'opérateur 
$$||\varphi||=\underset{x\in E\{0\}}{\mathrm{sup}}\frac{|\varphi(x)|}{||x||}.$$
On montre que c'est un espace de Banach, \emph{i.e.} un espace vectoriel normé complet.

$\triangleright$\emph{ex : }si $\dim E<\infty$, en considérant $(e_1,\cdots, e_n)$ une base de $E$, alors $\forall \varphi\in E^*$,
$$\varphi(x)=\sum_{i=1}^n\varphi(e_i)x_i$$
où $x=(x_1,\cdots,x_n)$.

$\triangleright$\emph{ex : }pour les espaces $\ell^p$ avec $p<\infty$, on note $q$ le conjugué de $p$, on se donne $a=\left\{a_n\right\}\in\ell^q$, on pose 
$$\varphi_a:\begin{array}{|rcl}
  \ell^p&\longrightarrow&\C\\
  u&\longmapsto\sum_{n\in\N}u_na_n 
\end{array}$$
$\varphi_a$ est bien définie (inégalité de Hölder) et est continue, avec 
$$\varphi_a||\leqslant ||a||_q.$$

On rappelle la propriété suivante : 


\begin{prop}
  Soit $\varphi:E\longrightarrow \C$ linéaire,
  alors $\varphi\in E^*\Leftrightarrow\ker\ \varphi$ est fermé dans $E$
\end{prop}


\begin{Proof}
  A FAIRE
\end{Proof}


Soit $E$ un $\mathbf K$-espace vectoriel normé et $F\subset E$ un sous-espace vectoriel, on prend $\varphi:F\longrightarrow \mathbf K$ linéaire continue, on va se demander si il existe un prolongement de $\varphi$ sur $E$ de même norme que $\varphi$. Plus formellement, on cherche $f:E\longrightarrow \mathbf K$ telle que 
$$f\big|_F=\varphi\text{ et }||f|_E=||\varphi||_F.$$

On fait d'abord une parenthèse en théorie des ensembles, on se donne $(\mathcal E,\preccurlyeq)$ un \emph{ensemble ordonné}, \emph{i.e.} muni d'une relation d'ordre.

$A\subset \mathcal P(\mathcal E)$ est \emph{totalement ordonné} si, et seulement si 
$$\forall (x,y)\in A, x\preccurlyeq y\text{ ou }y\preccurlyeq x.$$

On dit que $\preccurlyeq$ est \emph{inductif} si toute partie de $\mathcal E$ totalement ordonnée admet un majorant.

On dit que $z\in \mathcal E$ est un \emph{élément maximal} si, et seulement si 
$$\forall x\in \mathcal E, z\preccurlyeq x\Rightarrow z=x.$$


$\triangleright$\emph{ex :} soit $\O$ un ensemble de cardinal supérieur à 2, pour $A,B\in\mathcal P(\O)$, on pose
$$A\preccurlyeq B\Leftrightarrow A\subset B,$$
alors $(\mathcal P(\O),\preccurlyeq)$ est un ensemble ordonné non totalement ordonné donc $\O$ est un élément maximal.\\[1em]

On a le résultat / axiome suivant (en fait, c'est équivalent à l'axiome du choix);


\begin{thC}[lemme de Zorn]
  Tout ensemble non vide, ordonné et inductif possède un élément maximal.
\end{thC}


\begin{thC}[de Hann-Banach]
  Soit $E$ un $\R$-espace vectoriel et $F\subset E$ un sous-espace vectoriel, soit $\varphi\in F^*$, 

  alors il existe $\tilde\varphi\in E^*$ telle que 
  $$\tilde\varphi\big|_F=\varphi\text{ et }||\varphi||_{F^*}=||\tilde\varphi||_{E^*}$$
\end{thC}


\begin{Proof}
  A FAIRE
\end{Proof}

\end{document}
