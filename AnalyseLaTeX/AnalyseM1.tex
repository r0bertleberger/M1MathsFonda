\documentclass[a4paper,11pt, twoside]{article}
\title{}
\author{Raphaël Casanova}

\usepackage{packageTest}


\begin{document}

\pagestyle{empty}
\begin{center}
{\color{white} test}\\
\vspace{5cm}
{\bf \Huge {\YUGE A}NALYSE}\\[1em]
{\large
Raphaël Casanova\\
\href{mailto:raphael.casanova@centrale.centralelille.fr}{raphael.casanova@centrale.centralelille.fr}\\[2em]
\emph{D'après le cours d'Emmanuel Fricain à l'université de Lille}}\\
\vspace{10cm}
\includegraphics[width=1\textwidth]{lille.pdf}
\end{center}


\newpage


\pagestyle{pageGarde}


\tableofcontents


\newpage


Ce document est une transciption en \LaTeX\ plus ou moins fidèle au cours d'analyse d'Emmanuel Fricain en première année de Master Mathématiques, en 2024-2025.\\[3em]

En terme de notation, j'ai préféré utiliser les notations en {\bf gras} pour les ensembles : 
$$\N,\mathbf Z,\mathbf Q,\R,\mathbf C$$ plutôt que celles en : $$\mathbb N,\mathbb Z, \mathbb Q,\mathbb R,\mathbb C$$ pour coller à la norme ISO 80000-2.\\

La notation $f:E\hookrightarrow F$ indique que l'application $f$ est injective de $E$ dans $F$, la notation $f:E\twoheadrightarrow F$ indique que l'application $f$ est surjective de $E$ dans $F$ et la notation $f_n\tend f$ indique que la suite $(f_n)$ converge uniformément vers $f$.\\ 
La norme opérateur sera notée $||\cdot||_{\mathrm{op}}$ ou $\norm{\cdot}$ si il n'y a pas de confusion possible avec les autres normes (dans la littérature, on rencontre parfois la notation $|||\cdot|||$).\\

Pour $x:N\longrightarrow E$ une suite, la notation $(x_n)_{n\geqslant 0}$ insiste sur l'aspect \og suite\fg\ tandis que la notation $\ens{x_n}_{n\geqslant 0}$ insiste sur l'aspect \og ensemble de points\fg.\\

Dans le présent document, $\K=\R$ ou $\mathbf C$, $(K,d)$ est un espace métrique compact.\\

Dans le contexte d'une suite $(f_n)$ d'applications de $\Par{E,\norm{\cdot}_\infty}\longrightarrow\K$, l'abréviation CVS signifie \og converge simplement\fg\ \emph{i.e.} 
$$\forall x\in E,\ f_n(x)\tend f(x)$$
et l'abréviation CVU signifie \og converge uniformément\fg\ \emph{i.e.}
$$\norm{f_n-f}_\infty\tend 0.$$


Le symbole $\square$ signifie \emph{C.Q.F.D.}, le symbole $(\bullet)$ indique plusieurs hypothèses à vérifier, le symbole $\star$ indique une étape de la preuve et le symbole $\dagger$ indique une disjonction de cas.




\pagestyle{TestPage}

\NouvellePart{Espace des fonctions continues sur un compact}


On commence par (re)voir quelques notions de topologies qui vont nous servir au cours de ce chapitre :

\PremiereNouvelleSection{3 notions importantes}


$\bullet\quad$ densité : si $(E,d)$ est un espace métrique, $A\subset E$ est dite \emph{dense dans }$E$ si, et seulement si 
$$\forall x\in E,\ \exists \left(a_n\right)\in A\ \mathrm{t.q.}\ d(a_n,x)\tend 0.$$

$\bullet\quad$ compacité : l'espace métrique $(E,d)$ est \emph{compact} si, et seulement si toute suite dans $E$ admet une sous-suite convergente.\\
C'est équivalent à : pour tout recouvrement de $E$ par une collection quelconque d'ouverts $\left(V_i\right)_{i\in I}$, il existe un sous-recouvrement fini, \emph{i.e.} 
$$E\subset\bigcup_{k=1}^nV_k.$$

$\bullet\quad$ complétude :  l'espace métrique $(E,d)$ est \emph{complet} si toute suite de Cauchy à valeur dans $E$ converge dans $E$.\\

Dans le contexte d'espaces de fonctions,\\
la densité s'identifie avec une approximation par des fonctions régulières et la compacité / complétude s'identifie avec l'existence de limites et de valeurs d'adhérences de suites de fonctions.\\


\setcounter{CompteurRemarque}{0}
\begin{RQ}
  on montre que $\left(C(K,\K),||\cdot||_{\infty}\right)$ est un espace de Banach, $i.e;$ un espace vectoriel normé complet.
\end{RQ} 


\begin{Proof}
  À FAIRE
\end{Proof}


Dans la suite du document, on considerera un compact $K$ et on se placera dans $\big(C(K,\K),\norm{\cdot}_\infty\big)$.

\NouvelleSection{Théorème de Dini}


On sait, depuis la \emph{spé} que la convergence uniforme implique la convergence simple, et qu'il n'y a \emph{à priori} pas de réciproque (un contre-exemple est $f_n :t\in ]0,1[ \longmapsto t^n$, qui converge simplement vers la fonction nulle, mais dont la norme infinie est constante égale à 1).\\
Le théorème suivant nous donne un critère de réciproque :\\


\begin{thC}{Dini}
  soit $\left(f_n\right)\in C(K,\R)$, avec $K$ un compact, on suppose\\
  $(\bullet)\quad f_n$ converge simplement vers $f:K\longrightarrow\R$,\\
  $(\bullet\bullet)\quad f$ est continue,\\
  $(\bullet\bullet\bullet)\quad \forall x\in K,\ \left(f_n(x)\right)_{n\geqslant 0}$ est croissante,\\

  alors $f_n$ converge uniformément vers $f$.
\end{thC}


\begin{Proof}
  Soit $\varepsilon >0$, pour tout $n\geqslant 1$, on pose 
  $$\O_n:=\left\{x\in K \ |\ f_n(x)>f(x)-\varepsilon\right\},$$
  chaque $\O_n$ est un ouvert car c'est la pré-image d'un ouvert de $\R$ par une application continue.\\
  On a aussi 
  $$K=\bigcup_{n\geqslant 1}\O_n$$
  $K$ étant compact, on peut re-numéroter les $\O_n$ de façon à avoir 
  $$K=\bigcup_{i=1}^M\O_{n_i}.$$
  Quitte à réarranger la suite $(n_i)$, on la suppose croissante. Puisque chaque $\left(f_n(x)\right)$ est croissante, on a aussi les inclusions 
  $$\forall i\in \llbracket 1,M-1\rrbracket,\ \O_{n_i}\subset\O_{n_{i+1}}$$
  donc 
  $$K=\O_{m_n}.$$
  Autrement dit,
  $$\exists N\in\N \ |\ \forall x\in K,\ f_n(x)> f(x) - \varepsilon.$$
  Puisque $f_n(x)$ tend par valeurs inférieurs vers $f(x)$, on a aussi l'inégalité 
  $$\forall x\in K,\ f(x)-f_n(x)>0.$$
  On peut donc écrire 
  $$\exists N\in\N \ |\ \forall x\in K, |f_n(x) - f(x)| < \varepsilon.$$
  Et comme la suite $\left(f_n(x)\right)$ est croissante et converge vers $f(x)$, on a finalement
  $$\forall n\geqslant N,\ \forall x\in K,\ |f_n(x)-f(x)|< \varepsilon,$$
  et $f_n$ converge uniformément vers $f$.
\end{Proof}


$\triangleright$\emph{ex :} si l'on pose 
$$\left\{\begin{array}{rcl}
P_0(t)&=&0\\
P_{n+1}(t)&=&P_n(t)+\dstyle\frac12\big(t-P^2_n(t)\big)
\end{array}\right.$$
alors $P_n$ est une suite de polynômes convergeant uniformément vers $t\in[0,1]\longmapsto \sqrt t.$


\begin{Proof}
  On montre par récurrence que 
  $$\forall t\in[0,1],\ \forall n\in\N,\ 0\leqslant P_n(t)\leqslant \sqrt t$$
  ce qui nous montre que la suite $P_n(t)$ est croissante et bornée, donc converge (vers $f$).\\
  On passe à la limite dans la relation de récurrence 
  $$f(t)=f(t)+\dstyle\frac12(t-f^2(t))$$
  donc (puisqe $f\geqslant0$)
  $$f(t)=\sqrt t.$$
  Le théorème de Dini s'applique, et $P_n$ converge uniformément sur $[0,1]$ vers $t\mapsto\sqrt t.$
\end{Proof}




\NouvelleSection{Théorème(s) de Stone-Weierstra\ss}


On commence par une parenthèse algébrique : soit $\A$ un ensemble, muni des lois $(+,.,\times)$.


\begin{Def}
  On dit que $\A$ est une $\emph{algèbre}$ si, et seulement si\\
  $(\bullet)\quad (\A,+,.)$ est un espace vectoriel,\\
  $(\bullet\bullet)\quad \times : \A\times\A\longrightarrow\A$ est bilinéaire.
\end{Def}


$\A$ est \emph{unitaire} si il contient une élement neutre pour $+$.


\begin{Def}
  Une partie $\mathcal B$ de l'algèbre $\A$ est une \emph{sous-algèbre de} $\A$ si, et seulement si\\
  $(\bullet)\quad (\mathcal B,+,.)$ est un sous-espace vectoriel de $(\A,+,.)$,\\
  $(\bullet\bullet)\quad \mathcal B$ est stable pour la loi $\times$.\\
\end{Def}

La question à laquelle répond ce paragraphe est comme suit : si $\A$ est une sous-algèbre unitaire de $C(K,\K)$, à quelle condition est-ce-que $\A$ est dense dans $C(K,\K)$ ?\\[1em]


\begin{RQ}
  si $\A$ est dense dans $C(K,\K)$, alors $\A$ sépare les points, \emph{i.e.}
  $$\forall (x,y)\in K,\ x\neq y,\ \exists f\in\A\ |\ f(x)\neq f(y).$$
\end{RQ}


\begin{Proof}
  soit $x\neq y$ dans $K$ et $g:z\longmapsto d(x,z)$, on a alors 
  $$g(x)=0\text{ et }g(y)>0$$
  $g$ est une distance donc est 1-lipschitzienne donc est continue. Soit $\varepsilon:=g(y)$, par densité de $\A$ il existe $f\in \A$ telle que 
  $$||f-g||_{\infty}<\dstyle\nicefrac{\varepsilon}{2}$$
  On a 
  $$|f(x)|=|f(x)-g(x)|\leqslant||f-g||_{\infty}<\dstyle\nicefrac{\varepsilon}{2}$$
  On a aussi (inégalité triangulaire)
  $$|g(y)|=|(g(y)-f(y))+f(y)|\leqslant|g(y)-f(y)|+|f(y)|$$
  d'où
  $$|f(y)|\geqslant |g(y)|-|g(y)-f(y)|>\varepsilon-\dstyle\nicefrac{\varepsilon}{2}=\dstyle\nicefrac{\varepsilon}{2}$$
  donc 
  $$|f(y)|>|f(x)|$$
  et $\A$ sépare les points.
\end{Proof}


Avant de passer aux théorèmes de Stone-Weierstra\ss, on montre quelques propriétés générales sur les sous-algèbres unitaires de $C(K,\K)$.


\begin{prop}
  Soit $\A$ une sous-algèbre unitaire de $C(K,\K)$, soient $f,g\in\A$,\\
  
  alors\\
  $(\mathit 1)\quad |f|\in\overline{\A}$,\\[1em]
  $(\mathit 2)\quad \mathrm{sup}(f,g)$ et $\mathrm{inf}(f,g)\in\overline{\A}$.
\end{prop}


\begin{ProofC}{Démonstration du $(\mathit1)$}
  Si $f=0$, alors $f=|f|$ et le résultat est vrai, on peut donc supposer $f\neq 0$ et l'on pose 
  $$g:=\frac{f}{||f||_{\infty}}\in\A$$
  on a alors 
  $$0\leqslant g^2\leqslant 1$$
  on peut donc composer $g^2$ par la suite des $P_n$ définie précedemment, donc pour $\varepsilon>0$, on a 
  $$\exists N\in\N\ |\ \forall n\geqslant N,\ \underset{x\in K}{\mathrm{sup}}\left|P_n(g^2(x))-|g(x)|\right|\leqslant\varepsilon.$$
  $\A$ étant stable par $+$ et $\times$, $x\longmapsto P_n(g^2(x))\in\A$, donc l'inégalité nous indique que $|g|\in\overline{\A}$.\\
  $\overline{\A}$ étant aussi stable par multiplication par un scalaire, on en déduit que $|f|\in\overline{\A}$.
\end{ProofC}


\begin{ProofC}{Démonstration du $(\mathit2$)}
  C'est une conséquence immédiate du premier point, en écrivant nos fonctions sous les formes suivantes : 
  $$\mathrm{sup}(f,g)=\frac12\left(f+g+|f-g|\right)$$
  et
  $$\mathrm{inf}(f,g)=\frac12\left(f+g-|f-g|\right).$$
\end{ProofC}


\begin{thC}{Stone-Weierstra\ss, cas réel}
  Soit $(K,d)$ un espace métrique compact, $\A$ une sous-algèbre unitaire de $C(K,\R)$ qui sépare les points,\\

  alors $\A$ est dense dans $C(K,\R).$
\end{thC}


avant de démontrer ce théorème, on aura besoin des trois lemmes suivants :


\begin{lemme}
  Soit $\A$ une sous-algèbre unitaire de $C(K,\R)$ séparant les points,\\

  alors $\forall (x,y)\in K$ distincts, $\forall (\alpha,\beta)\in\R$, il existe $f\in\A$ telle que 
  $$f(x)=\alpha\text{ et }f(y)=\beta.$$
\end{lemme}


\begin{Proof}
  puisque $\A$ sépare les points, il existe $g\in\A$ telle que
  $$g(x)\neq g(y).$$
  Soit le système d'équations linéaires
  $$(S)\left\{\begin{array}{rcl}
    \lambda g(x)+\mu&=&\alpha\\
    \lambda g(y)+\mu&=&\beta
  \end{array}\right.$$
  Ce système a pour déterminant
  $$\det S=g(x)-g(y)\neq 0$$
  il est donc inversible et admet une solution.

  On vérifie que l'application 
  $$t\longmapsto \lambda g(t)+\mu$$
  convient et est dans $\A$.
\end{Proof}


\begin{lemme}
  Soit $\A$ une sous-algèbre unitaire de $C(K,\R)$ séparant les points, soient $\varphi\in C(K,\R)$ et $\varepsilon>0$,\\

  alors $\forall x\in K,\ \exists f_x\in\overline\A$ telle que 
  $$\left\{\begin{array}{l}
    f_x(x)=x\\
    \forall x\in K,\ f_x(z)>\varphi(x)-\varepsilon.
  \end{array}\right.$$
\end{lemme}


\begin{Proof}
  Soit $x\in K$, on sait, d'après le lemme précédent que pour $y\in K$, il existe $f^y\in \A$ telle que 
  $$f^y(x)=\varphi(x)\text{ et }f^y(y)=\varphi(y).$$
  (on considère ici que $\varphi(x)\neq\varphi(y)$, si ce n'est pas le cas, on prend $f^y=\varphi(x)$)
  ||f-g||
  Par continuité de $f^y$ et de $\varphi$, il existe des voisinages $V^y$ de $y$ tels que 
  $$\forall z\in V^y,\ f^y(z)>\varphi(z)-\varepsilon.$$
  La famille $(V^y)_{y\in K}$ est un recouvrement par ouverts du compact $K$ donc il existe une numérotation des $y$ telle que 
  $$K=\bigcup_{i=1}^p V^{y_i}.$$
  Soit 
  $$f_x:=\mathrm{sup}\left\{f^{y_1},\cdots,f^{y_p}\right\}$$
  $f_x\in\overline\A$ et par définition des $f^y$, on a $f_x(x)=\varphi(x)$. De plus,
  $$\forall z\in K,\exists i\in\llbracket 1,p\rrbracket\ |\ z\in V^{y_p}$$
  donc
  $$f_x(z)\geqslant f^{y_i}(z)>\varphi(x)-\varepsilon.$$
\end{Proof}


\begin{lemme}
  Soit $\A$ une sous-algèbre unitaire de $C(K,\R)$ séparant les points, soit $\varphi\in\overline{\A}$ et $\varepsilon>0$,\\

  alors il existe $f\in\overline{\A}$ telle que $||f-\varphi||_{\infty}<\varepsilon.$
\end{lemme}


\begin{Proof}
  Soit $x\in K$ et $f_x$ telle que décrite dans le lemme 2. Puisque $f_x$ et $\varphi$ sont continues, il existe un voisinage $V_x$ de $x$ tel que 
  $$\forall x\in V_x,\ f_x(z)<\varphi(z)+\varepsilon.$$
  La famille $(V_x)_{x\in K}$ est un recouvrement par ouverts du compact $K$, donc il existe une numérotation des $x$ telle que 
  $$K=\bigcup_{i=1}^m V_{x_i}.$$
  Soit 
  $$f:=\mathrm{int}\left\{f_{x_1},\cdots,f_{x_m}\right\}$$
  $f\in\overline{\overline{\A}}=\overline{\A}$ et comme dans la démonstration du lemme précédent, 
  $$\forall z\in K,\ f(z)<\varphi(z)+\varepsilon\qquad (i).$$
  Par ailleurs, tous les $f_x$ vérifient aussi
  $$\forall z\in K,\forall x_i\ |\ f_{x_i}(z)>\varphi(z)-\varepsilon$$
  chacun des $f_{x_i}$ est supérieur à $\varphi-\varepsilon$, par passage à l'inf on a 
  $$\forall z\in K,\ f(z)>\varphi-\varepsilon\qquad (ii).$$
  En combinant $(i)$ et $(ii)$, on obtient l'encadrement suivant :
  $$\forall z\in K,\ \varphi(z)-\varepsilon<f(z)<\varphi(z)+\varepsilon$$
  autrement dit 
  $$||f-\varphi||_{\infty}<\varepsilon.$$
\end{Proof}


On peut enfin prouver le théorème :


\begin{Proof}
  Soient $\varphi\in C(K,\R)$ et $\varepsilon>0$, d'après le lemme 3, il existe $f\in \overline{\A}$ telle que 
  $$||f-\varphi||_{\infty}<\nicefrac{\varepsilon}{2}.$$
  Puisque $f\in\overline{\A}$, il existe $g\in\A$ telle que 
  $$||f-g||_{\infty}<\nicefrac{\varepsilon}{2}$$
  donc
  \begin{flalign*}
    ||f-g||_{\infty}&=||(\varphi-f)+(f-g)||_{\infty}\\
    &\leqslant ||\varphi-f||_{\infty}+||f-g||_{\infty}\\
    ||f-g||_{\infty}&\leqslant \varepsilon
  \end{flalign*}
  et $\A$ est bien dense dans $C(K,\R)$.
\end{Proof}


\begin{corollaire}
  Soient $a<b\in\R$,\\
  
  alors $\R[X]$ est dense dans $C([a,b],\R)$.
\end{corollaire}
 

\begin{Proof}
  Vérifier que $\R[X]$ est une sous-algèbre unitaire de $C([a,b],\R)$ qui sépare les points (le polynôme identité convient pour la séparation).
\end{Proof}


On peut même expliciter un (il n'y a pas unicité de l'approximation) polynôme convenable, soit $f\in C([0,1],\R)$, alors 
$$B_n(f)(X)=\sum_{i=0}^nf\left(\frac{k}{n}\right)\binom{n}{k}X^i(1-X)^{n-i}$$
est une suite de polynômes convergeant uniformément sur $[0,1]$ vers $f$.\\[1em]


\dnote{ce résultat est faux sur tout $\R$ entier, puisque si il existe $P_n$ une suite de polynômes convergeant uniformément vers $f:\R\longrightarrow \R$ sur $\R$, alors $f$ est polynômiale.}


\begin{Proof}
  Puisque $P_n$ CVU vers $f$, il existe $N$ tel que 
  $$\forall n\geqslant N,\ ||P_N-f||_{\infty}\leqslant 1$$
  donc, par inégalité triangulaire,
  $$\forall n\geqslant N,\ ||P_N-P_n||_{\infty}=||(P_N-f)+(f-P_n)||_{\infty}\leqslant 2$$
  Les seuls polynômes bornés sont des constantes, donc 
  $$\forall n\geqslant N,\ \exists \lambda_n\ |\ Pn-P_N=\lambda_n.$$
  On évalue cette expression en 0 et on trouve que 
  $$\lambda_n \tend f(0) - P_N(0)=:\lambda_{\infty}.$$
  Donc $$P_n\tend P_N + \lambda_{\infty}$$
  et par unicité de la limite, 
  $$f=P_N+\lambda_{\infty}.$$
\end{Proof}


\dnote{ce résultat est aussi faux sur $\C$ si la sous-algèbre considérée n'est pas stable par conjugaison.}


\begin{Proof}
  On considère l'application $f:z\longmapsto \overline z$,\\
  $\C[X]$ est bien une sous-algèbre unitaire de $C(\mathbf U,\C)$ qui sépare les points, donc si le théorème de Stone-Weierstra\ss\ est vrai, alors $\C[X]$ est dense dans $C(\mathbf U,\C)$, alors il existe $P_n\in\C[X]$ convergeant uniformément vers $f$.

  On a 
  $$\forall n\in\N,\ \int_0^{2\pi}\mathrm e^{int}\mathrm e^{it}\mathrm dt=0$$
  donc par linéarité de l'intégrale,
  $$\forall n\in\N,\ \int_0^{2\pi}P_n\left(\mathrm e^{it}\right)\mathrm e^{it}\mathrm dt=0$$
  par convergeance uniforme des $P_n$ vers $f$, on a aussi
  $$\int_0^{2\pi}f\left(\mathrm e^{it}\right)\mathrm e^{it}\mathrm dt=0$$
  or on peut calculer
  $$\int_0^{2\pi}f\left(\mathrm e^{it}\right)\mathrm e^{it}\mathrm dt=\int_0^{2\pi}\mathrm e^{-it}\mathrm e^{it}\mathrm dt=\int_0^{2\pi}1\mathrm dt=2\pi.$$
  c'et absurde donc le théorème est effectivement faux sur $\C$.
\end{Proof}

\begin{Def}
  Une partie $P$ de $\C$ est dite \emph{stable par conjugaison} si, et seulement si
  $$ x\in E\Rightarrow \overline x\in E.$$
\end{Def}

\begin{thC}{Stone-Weierstra\ss, cas complexe}
  Soit $(K,d)$ un espace métrique compact, $\A$ une sous-algèbre unitaire de $C(K,\C)$ qui sépare les points et est stable par conjugaison,\\

  alors $\A$ est dense dans $C(K,\C).$
\end{thC}

\begin{Proof}
  Soit $f\in C(K,\C)$, on pose 
  $$\A_{\R}:=A\cap C(K,\R),$$
  c'est une sous-algèbre unitaire de $C(K,\R)$ et, grâce à la stabilité par conjugaison, 
  $$\Re\ f=\frac{f+\overline f}2\text{ et }\Im\ f=\frac{f-\overline f}2$$
  sont tous deux dans $\A_{\R}$.

  Vérifions que $\A_{\R}$ sépare les points, soit $(x,y)\in \C$ distincts, puisque $\A$ sépare les points, il existe $\varphi\in \A$ telle que $\varphi(x)\neq \varphi(y)$, donc soit
  $$\Re \varphi(x)\neq \Re \varphi(y)$$
  soit 
  $$\Im \varphi(x)\neq \Im \varphi(y)$$
  et $\A_{\R}$ sépare aussi les points.
  On applique donc le théorème de Stone-Weierstra\ss, cas réel à $\Re\ f$ et $\Im\ f$, on fixe $\varepsilon>0$ et il existe $(g,h)\in \A_{\R}$ tels que 
  $$\left\{\begin{array}{l}
    ||g-\Re\ f||_{\infty}<\nicefrac{\varepsilon}{2}\\
    ||h-\Im\ f||_{\infty}<\nicefrac{\varepsilon}{2}.
  \end{array}\right.$$
  On pose $$\psi:=g+ih\in \A$$
  et on a alors
  \begin{align*}
    ||f-\psi||_{\infty}&=||\Re\ f+i\Im\ f-g-ih||_{\infty}\\
    &\leqslant ||g-\Re\ f||_{\infty}+||h-\Im\ f||_{\infty}\\
    ||f-\psi||_{\infty}&\leqslant \varepsilon\\
  \end{align*}
  et $\psi$ CVU vers $f$, donc $\A$ est dense dans $C(K,\C)$.
\end{Proof}


\begin{corollaireC}{Féjer}
  L'ensemble
  $$\left\{z\longmapsto z ^n,\ n\in\mathbf Z\right\}$$
  est dense dans $C(\mathbf U,\C)$
\end{corollaireC}

\begin{Proof}
  On vérifie que le théorème de Stone-Weierstra\ss, cas complexe s'applique.
\end{Proof}


Un polynôme trigonométrique est une application de la forme
$$P:t\longmapsto\sum_{k =-N}^Nc_k\mathrm e^{ikt}$$
où les $c_k$ sont des coefficients complexes.

En passant à la forme trigonométrique de $\mathrm e^{ikt}$, on peut ré-écrire $P$ comme
\begin{flalign*}
  P(t)&=\sum_{k =-N}^Nc_k\left[\cos(kt)+i\sin(kt)\right]\\
  P(t)&=\sum_{k =-N}^N a_k\cos(kt)+b_k \sin(kt).
\end{flalign*}

\begin{corollaire}
  Soit $f:\R\longrightarrow\C$ continue et $2\pi$-périodique,\\

  alors il existe une suite de polynômes trigonométriques convergeant uniformément vers $f$.
\end{corollaire}

\begin{Proof}
  il suffit de constater que 
  $$C_{2\pi}(\R)\simeq C(\mathbf U)$$
  via l'application
  $$\varphi : f\in C_{2\pi}(\R)\longmapsto \tilde f(t):=f(\mathrm e^{it})\in C(\mathbf U).$$
\end{Proof}



\NouvelleSection{Espaces séparables}

\begin{Def}
  $(E,d)$ est dit séparable si, et seulement si il existe $A\subset E$ dénombrable et dense dans $E$.
\end{Def}

On rappelle que $A$ est dénombrable si, et seulement si $A$ est fini ou $A\simeq \N$.

$\triangleright$\emph{ex : }$\R$ est séparable car $\overline{\mathbf Q}=\R$ et $\mathbf Q\simeq \N^2\simeq \N$.

On montre (en TD) que tout espace vectoriel de dimension finie est séparable. On a même la propriété suivante, qui est une version un peu plus forte :


\begin{prop}
  Soit $(E,d)$ un espace vectoriel normé, on suppose qu'il existe une famille $(e_n)\in E$ telle que 
  $$\overline{\mathrm{Vect}\ens{e_i,\ i\in \mathbf N}}=E$$
  où $\mathrm{Vect}\ens{e_i,\ i\in \mathbf N}$ est l'ensemble des combinaisons linéaires finies des $(e_n)$,\\

  alors $E$ est séparable.
\end{prop}


\begin{Proof}
  Idée de la preuve : il s'agira de montrer que 
  $$\left\{\sum_{i\in I}\lambda_i e_i,\ \lambda_i\in\mathbf Q,\ I\subset\N\text{ finie}\right\}$$
  est dense et dénombrable.
\end{Proof}


\begin{prop}
  Soit $(E,d)$ un espace métrique et $(O_i)_{i\in I}$ une famille quelconque d'ouverts deux à deux disjoints, on suppose que $E$ est séparable,\\

  alors $I$ est dénombrable.
\end{prop}


$\triangleright$\emph{ ex : }soit $1\leqslant p<\infty$, on rappelle que 
$$\ell^p(\N):=\left\{x:\N\longrightarrow\C\ |\ \sum_{n\in\N}|x(n)|^p<\infty\right\}$$
muni de la norme
$$||\cdot||_p:x\in\ell^p\longmapsto\left(\sum_{n\in\N}|x(n)|^p\right)^{\nicefrac{1}{p}}$$
est un espace (vectoriel normé) séparable.


\begin{Proof}
  Soit $e_n=\mathbf 1_{\{n\}}=(0,\cdots,0,1,0,\cdots,0)$ où le $1$ est en $n$-ème position et soit $x_n\in\ell^p$, alors 
  $$\norm{u-\sum_{i=1}^Nx_ie_i}_p=\left(\sum_{i=N+1}^{\infty}|x_n|^p\right)^{\nicefrac{1}{p}}\underset{N\to+\infty}{\longrightarrow 0}$$
  ce qui montre que (puisque chaque $e_n$ est dans $\ell^p$),
  $$\overline{\mathrm{Vect}\ens{e_i,\ i\in \N}}=\ell^p$$
  donc $\ell^p$ est bien séparable.
\end{Proof}


\begin{lemme}
  Soit $(K,d)$ compact,\\

  alors $K$ est séparable.
\end{lemme}

\begin{Proof}
  Soit $n\geqslant 1$, on a 
  $$K=\bigcup_{x\in K}B(x,\nicefrac{1}{n})$$
  par compacité de $K$, on peut numéroter ces boules donc 
  $$K=\bigcup_{j=1}^{N_n}B(x_j^n\nicefrac{1}{n})$$
  et l'ensemble 
  $$\mathcal D:=\left\{x_j^n,\ 1\leqslant j\leqslant N_n,\ n\in\N\right\}$$
  est dénombrable, montrons qu'il est dense,

  soit $x\in K$, alors 
  $$\forall n\in\N,\ x\in\bigcup_{i=1}^{N_n}B(x_i^n, \nicefrac{1}{n})$$
  autrement dit, 
  $$\forall n\in\N,\ \exists i_n\in\llbracket 1,N_n\rrbracket\ |\ x\in B(x_{i_n}^n,\nicefrac{1}{n})$$
  puisque le rayon de la boule tend vers $0$ quand $n\to+\infty$, on peut écrire 
  $$x_{i_n}^n\tend x$$
  ce qui montre que $\mathcal D$ est bien dense dans $K$, et $K$ est séparable.
\end{Proof}


\begin{lemme}
  Soit $(K,d)$ compact,\\

  alors $C(K,\R)$ est séparable.
\end{lemme}

\begin{Proof}
  $K$ est compact, donc d'après le lemme précédent, il existe $\ens{a_n}_{n\geqslant 0}$ une suite dense dans $K$. Pour $n\geqslant 1$, on pose 
  $$f_n:t\longmapsto d(a_n,t),$$
  les applications $f_n$ sont continues et on pose $f_0:=1$.

  On considère l'ensemble $\A$ définit comme suit :
  $$\A:=\mathrm{Vect}\ens{\prod_{i\in I}f_i,\ I\text{ finie}}$$
  c'est une sous-algèbre unitaire et on va montrer que $\A$ sépare les points.

  Soient $(t_1,t_2)\in K$ distincts, alors $\varepsilon:=d(t_1,t_2)>0$. Par densité des $\{a_n\}$ dans $K$, il existe $n\in\N$ tel que 
  $$d(t_1,a_n)<\nicefrac{\varepsilon}{2}$$
  donc 
  $$f_n(t_1)=d(a_n,t_1)<\nicefrac{\varepsilon}{2}$$
  et de plus,
  \begin{align*}
    f_n(t_2)&=d(t_2,a_n)\\
    &\geqslant d(t_2,t_1)-d(t_1,a_n)\\
    &>\varepsilon-\nicefrac{\varepsilon}{2}\\
    f_n(t_2)&>\nicefrac{\varepsilon}{2}
  \end{align*}
  donc $\A$ sépare les points, ce qui nous permet d'appliquer le théorème de Stone-Weierstra\ss, donc $\A$ est dense dans $C(K,\R)$, donc $C(K,\R)$ est séparable.
\end{Proof}



\NouvelleSection{Théorème d'Ascoli}


Le fil directeur de cette partie est la question suivante : peut-on caractériser les partie compacte de $C(K)$ ?


\begin{Def}
  Soit $\mathcal F\subset C(K)$,

  $\mathcal F$ est dite \emph{équicontinue en }$x$ si, et seulement si
  $$\forall \varepsilon>0, \exists\delta>0,\ \forall f\in\mathcal F, \forall y\in B(x,\delta)\cap K,\ |f(x)-f(y)|<\varepsilon.$$
  $\mathcal F$ est dite \emph{équicontinue} si, et seulement si $\mathcal F$ est équicontinue en tout point de $K$.
\end{Def}

$\triangleright$\emph{ex : }pour $\ell>0$, l'ensemble des fonctions $\ell$-lipschitzienne forme une famille équicontinue.


\begin{thC}{Ascoli, v1}
  Soient $(K,d)$ un compact et $\mathcal F=(f_n)$ une famille dans $C(K)$, si,

  $(\bullet)\ \forall x\in K,\ \big(f_n(x)\big)_{n\geqslant 0}$ est une suite bornée de $K$,\\
  $(\bullet\bullet)$ $\mathcal F$ est équicontinue,\\

  alors il existe une sous-suite des $f_n$ qui converge uniformément dans $C(K).$
\end{thC}


\begin{Proof}
  Puisque $K$ est un compact, il est séparable et il existe une famille $\D=\{a_n\}_{n\geqslant 0}$ dense dans $K$. 
  
  Par hypothèse, la suite $\left(f_n(a_0)\right)$ est bornée, donc on peut en extraire une sous-suite convergente, donc il existe $\varphi_0:\N\longrightarrow\N$ strictement croissante telle que $\left(f_{\varphi_0(n)}(a_0)\right)_n$ converge.

  De même, la suite $\left(f_{\varphi_0(n)}(a_1)\right)_n$ est bornée donc admet une extractrice $\varphi_1$ telle que $\left(f_{\varphi_0\circ\varphi_1(n)}(a_1)\right)_n$ converge.

  Par récurrence, on va construire les itérations succssives de $\varphi_n$ comme suit, autrement dit 
  $$\forall k\in\N,\ \left(f_{\varphi_0\circ\cdots\circ\varphi_k(n)}(a_k)\right)_n\text{ converge}$$
  Soit $\varphi:\N\longrightarrow\N$ telle que 
  $$\forall n\in\N,\ \varphi(n)=\varphi_0\circ\cdots\circ\varphi_n(n).$$
  Vérifions, par récurrence, que $\varphi$ est strictement croissante. 
    
  $\star$ pour $n=1$, on a (par croissante stricte de $\varphi_1$ et $\varphi_0)\ \varphi(1)=\varphi_1(\varphi_0(0))>\varphi(0).$

  $\star$ Soit $n\in\N$,
  $$\varphi(n+1)=\varphi_{n+1}(\varphi_0\circ\cdots\circ\varphi_n(n+1))$$
  et puisque $\varphi_0\circ\cdots\circ\varphi_n$ est strictement croissante, $\varphi_0\circ\cdots\circ\varphi_n(n+1)>\varphi_0\circ\cdots\circ\varphi_n(n)$ et en composant par $\varphi_{n+1}$, on trouve bien
  $$\varphi(n+1)>\varphi(n)$$
  ce qui conclut la récurrence.

  Donc $\left(f_{\varphi(n)}\right)$ est une sous-suite de $f_n$ telle que 
  $$\forall k\in\N,\ \left(f_{\varphi(n)}(a_k)\right)\text{ converge}$$
  puisque 
  $$\forall n\geqslant k,\ \left(f_{\varphi(n)}(a_k)\right)=\left(f_{\varphi_{k+1}\circ\cdots\circ\varphi_n(\varphi_1\circ\cdots\circ\varphi_k(n))}(a_k)\right)$$
  où la suite $\varphi_{k+1}\circ\cdots\circ\varphi_n$ est strictement croissante.

  Soit $\varepsilon>0$, par équicontinuité des $f_n$, pour tout $z\in K$, il existe $\delta_z>0$ tel que 
  $$\forall n\in \N,\ \forall u\in B(z,\delta_z),\ \left|f_{\varphi(n)}(u)-f_{\varphi(n)}(z)\right|\leqslant \varepsilon$$
  par inégalité triangulaire,
  $$\forall (u,v)\in B(z,\delta_z), \left|f_{\varphi(n)}(u)-f_{\varphi(n)}(v)\right|\leqslant 2\varepsilon\qquad (i).$$
  On peut écrire $K$ comme 
  $$K=\bigcup_{z\in K}B(z,\delta_z)$$
  par compacité de $K$, on a 
  $$K=\bigcup_{j=1}^NB(z_j,\delta{z_j}).$$
  Par densité de $\D$ dans $K$, pour tout $j\in\llbracket 1,n\rrbracket$, il existe un $a_{n_j}\in\D$ tel que 
  $$a_n\in B(z_j,\delta_{z_j})$$
  où $a_n$ est tel que $\left(f_{\varphi(n)}(a_{n_j})\right)_{n\geqslant 0}$ converge donc est de Cauchy, donc $\exists N_j$ tel que 
  $$\forall p,q\geqslant N_j,\ \left|f_{\varphi(q)}(a_{n_j})-f_{\varphi(p)}(a_{n_j})\right|\leqslant\varepsilon\qquad (ii).$$
  Soit $x\in K$, $\exists z_j\in K$ tel que $x\in B(z_j,\delta_{z_j})$ et $\exists a_{n_j}\in\D$ tel que $a_{n_j}\in B(z_j,\delta_{z_j})$, donc les hypothèses de l'inégalité $(i)$ sont valides et
  $$\forall n\geqslant 1, \left|f_{\varphi(n)}(x)-f_{\varphi(n)}(a_{n_j})\right|\leqslant 2\varepsilon$$
  pour $p,q\geqslant\mathrm{max}\{N_j,\ j\in\llbracket 1,N\rrbracket\}$
  \begin{align*}
    \left|f_{\varphi(q)}(x)-f_{\varphi(p)}(x)\right|&\leqslant \underbrace{\left|f_{\varphi(q)}(x)-f_{\varphi(q)}(a_{n_j})\right|}_{\leqslant 2\varepsilon\text{ d'après }(i)}+\underbrace{\left|f_{\varphi(q)}(a_{n_j})-f_{\varphi(p)}(a_{n_j})\right|}_{\leqslant \varepsilon\text{ d'après }(ii)}+\underbrace{\left|f_{\varphi(q)}(a_{n_j})-f_{\varphi(p)}(x)\right|}_{\leqslant 2\varepsilon\text{ d'après }(i)}\\
    &\leqslant 2\varepsilon+\varepsilon+2\varepsilon\\
    \left|f_{\varphi(q)}(x)-f_{\varphi(p)}(x)\right|&\leqslant 5\varepsilon
  \end{align*}
  par passage au $\mathrm{\sup}$, $||f_{\varphi(q)}-f_{\varphi(p)}||_\infty\leqslant 5\varepsilon$, donc la suite $f_{\varphi(n)}$ est de Cauchy à valeurs dans le complet $C(K)$, donc est convergente.
\end{Proof}

\begin{Def}
  Une partie $A\subset X$ de l'espace topologique $X$ est \emph{relativement compacte} si, et seulement si elle est inclue dans une partie compacte de $X$.\\

  Si $X$ est séparé, alors $A$ est relativement compacte si, et seulement si $\overline A$ est compacte.
\end{Def}


\begin{thC}{Ascoli, v2}
  Soit $(K,d)$ compact,\\

  $(\mathit 1)\quad$ les partie compactes de $C(K)$ sont exactement les parties fermées, bornées et équicontinues de $C(K)$,

  $(\mathit 2)\quad$ les parties relativement compactes de $C(K)$ sont exactement les parties bornées et équicontinues.
\end{thC}


\begin{ProofC}{Démonstration du $(\mathit 1)$}
  $\star$ Soit $\mathcal F\subset C(K)$ compacte fermée et bornée, montrons que cette famille est équicontinue. 
  
  Soit $\varepsilon>0$,
  \begin{align*}
    K&=\bigcup_{f\in\mathcal F}B(f,\varepsilon)\\
    &=\bigcup_{i=1}^NB(f_n,\varepsilon)
  \end{align*}
  Pour $i\in\llbracket 1,N\rrbracket$, pour $z\in K$, par continuité de $f_i$, $\exists \delta_{z,i}>0$ tel que 
  $$\forall u \in B(z,\delta_{z,i}),\ \left|f_i(u)-f_i(z)\right|<\varepsilon\qquad (i)$$
  On pose 
  $$\delta_z:=\mathrm{min}\{\delta_{z,i},\ i\in\llbracket 1,N\rrbracket\}>0.$$
  Soit $f\in\mathcal F$, donc 
  $$\exists i\in \llbracket 1,N\rrbracket\text{ t.q. }||f-f_i||_{\infty}<\varepsilon\qquad (ii)$$
  soit $i\in \llbracket 1,N\rrbracket$, $u\in B(z,\delta_z)\subset B(z,\delta_{z,i})$, on a 
  \begin{align*}
    |f(u)-f(z)|&\underbrace{\leqslant|f(u)-f_i(u)|}_{\leqslant\varepsilon\text{ d'après }(i)}+\underbrace{|f_i(u)-f_i(z)|}_{\leqslant\varepsilon\text{ d'après }(2)}+\underbrace{|f_i(z)-f(z)|}_{\leqslant\varepsilon\text{ d'après }(i)}\\
    |f(u)-f(z)|&\leqslant 3\varepsilon
  \end{align*}
  donc $\mathcal  F$ est équicontinue.\\

  $\star$ Réciproquement, si on prend $F\subset C(K)$ fermée bornée équicontinue, montrons que $\mathcal F$ est compacte.
  
  Soit $(f_n)\in \mathcal F$, c'est une famille bornée et équicontinue donc d'après le théorème d'Ascoli, elle admet une sous-suite convergente, qui appartient à $\mathcal F$ car c'est un fermé, donc $F$ vérifie la propriété de Weierstra\ss, c'est un compact.
\end{ProofC}


\begin{ProofC}{Démonstration du $(\mathit 2)$}
  Se déduit de la proposition précédente en considérant la fermeture de $\mathcal F$.
\end{ProofC}



\begin{thC}{Ascoli, v3}
  Soit $(E,d)$ séparable, $(F,\delta)$ métrique et $f_n\in C(E,F)$, si

  $(\bullet)\ \forall x\in E,\ \ens{f_n(x)}_{n\geqslant 0}$ est relativement compacte dans $F$,\\
  $(\bullet\bullet)\ \left(f_n\right)$ est une famille équicontinue,\\

  alors $f_n$ admet une sous-suite qui converge uniformément sur tout compact de $E$ vers une application continue.
\end{thC}

\newpage
Le théorème suivant découle du théorème d'Ascoli, et est quant à lui censé avoir plein d'applications :

\begin{thC}{Ascoli-Arzèla-Peano}
  soit $t_0\in\R,\ x_0\in\R^n, (a,r)>0$, on pose 
  $$K:=[t_0-a,t_0+a]\times \overline{B(x_0,r)}$$
  (qui est compact car fermé borné de $R^{n+1}$)\\
  on prend $f:K\longrightarrow \R^n$ continue, on pose
  $$M:=\mathrm{sup}_{(x,t)\in K}||f(x,t)||\text{ et }c:=\mathrm{min}(a,\nicefrac{a}{r}).$$
  Enfin on considère le problème de Cauchy suivant 
  $$\left\{\begin{array}{rcl}
    x'(t)&=&f(t,x(t))\\
    x(t_0)&=&x_0
  \end{array}\right.$$
  alors le problème admet une solution $x:[t_0-a,t_0+a]\longrightarrow \overline{B(x_0,r)}$
\end{thC}


\begin{Proof}
  Le principe va être de discrétiser $[t_0,t_0+c]$ puis de construire selon la méthode d'Euler une suite de fonction qui vérifie cette discrétisation.

  Puis on appliquera le théorème d'Ascoli pour montrer que cette suite de fonction CVU vers une fonction, qui sera solution du problème.\\

  $\star$ On considère une subdivision de $[t_0,t_0+c]$, de pas $h:=\nicefrac{c}{n+1}$, donc $\forall i\in\llbracket 0,n\rrbracket,\ t_i=t_0+i\frac{c}{n+1}$. On a alors (méthode d'Euler)
  $$x(t_{i+1})=x(t_i)+\int_{t_i}^{t_{i+1}}f(t,x(t))\mathrm dt.$$
  On construit par récurrence les points $x_i$ tels que 
  $$x_{i+1}=x_i+\frac{c}{n+1}f(t,x_i)$$
  On montre alors 
  $$||x_i-x_0||\leqslant i\frac{cM}{n+1}.$$
  Pour $t\in[t_i,t_{i+1}]$, on pose 
  $$X_n(t):=a_it+b_i$$
  où les $a_i$ et $b_i$ sont tels que 
  $$\left\{\begin{array}{ccl}
    X_n(t_i)&=&x_i\\
    X_n(t_{i+1})&=&x_{i+1}
  \end{array}\right.$$
  Ainsi on a 
  $$\left\{\begin{array}{ccl}
    \displaystyle a_i&=&\frac{x_{i+1}-x_i}{t_{i+1}-t_i}\\
    b_i&=&x_i-a_i t_i
  \end{array}\right.$$
  Chaque $x_n$ est continue sur son $[t_0,t_0+c]$, et de pour tout $t\in [t_i,t_{i+1}]$, $x_n$ y est dérivable et 
  $$X'_n(t)=a_i=\frac{\nicefrac{c}{n+1}f(t_i,x_i)}{\nicefrac{c}{n+1}}=f(t_i,x_i)$$
  donc 
  $$\mathrm{sup}_{t\in]t_i,t_{i+1}[}||X'_n(t)||\leqslant M$$
  donc (inégalité des accroissements finis) les $(x_n)$ forment une famille $M$-lipschitzienne donc équicontinue.

  On a aussi
  \begin{align*}
    \forall n\in\N,\ t\in [t_0,t_0+c],\ ||X_n(t)-x_0||&=||X_n(t)-X_n(t_0)||\\
    &\leqslant M|t-t_0|\\
    &\leqslant \\
    ||X_n(t)-x_0||&\leqslant r\\
  \end{align*}
  Donc $X_n(t)\in\overline{B(x_0,r)}$ qui est un fermé bornée de $\R^n$, donc compact. Ainsi, $\forall t\in [t_0,t_0+c]$, $\left(X_n(t)\right)_{n\geqslant 0}$ est relativement compact dans $\R^n$.

  D'après le théorème d'Ascoli (quitte à considérer une sous-suite, ce qu'on ne fait pas pour alléger les notations), $(X_n)$ CVU vers $X:[t_0,t_0+c]\longrightarrow\overline{B(x_0,r)}$, où $X$ est continue.

  Soit $t\in[t_i,t_{i+1}]$,
  \begin{align*}
    ||X_n(t)-f(t,X_n(t))||&=||f(ti_,x_i)-f(t,X_n(t))||\\
    &\leqslant \omega \frac{(M+1)c}{n+1}
  \end{align*}
  où 
  $$\omega :=\mathrm{sup}\left\{||(t,x)-t(u,y),\ |t-u|<\delta, ||x-y||<\delta\right\}<\infty$$
  avec $\delta>0$
  Donc 
  $$X_n(t)-x_0-\int_{t_0}^tf(u,X_n(u))\mathrm du\leqslant \omega_fc\frac{M+1}{n+1}\tend 0$$
  donc 
  $$X(t)=x_0+\int_{t_0}^tf(u,X_n(u)\mathrm du$$
  autrement dit, $X$ est solution.
\end{Proof}





\setcounter{prop}{0}


\NouvellePart{Théorèmes fondamentaux de l'analyse fonctionnelle}


\PremiereNouvelleSection{Théorème de Baire}

On rappelle le théorème suivant, déjà vu en topologie de L3 : 

\begin{thC}{des fermés emboités :}
  Soit $(E,d)$ un espace métrique complet, $(F_n)$ une suite de fermés non-vide de $E$, si
  $$\forall n\in\N,\ F_{n+1}\subset F_n$$
  et
  $$\mathrm{diam}\ F_n=\mathrm{sup}\left\{d(a,b),\ (a,b)\in\ F_n\right\}\tend 0,$$\\

  alors $\bigcap_{n\geqslant 0}F_n$ est un singleton (donc est non vide).
\end{thC}

\begin{Proof}
  Pour $n\in\N$, soit $x(n)\in F_n$, pour $p<q$, on a $x_q\in F_q\subset F_p$ donc 
  $$d(x_p,x_q)\leqslant \mathrm{diam}\ F_q\leqslant \mathrm{diam}\ F_p\longrightarrow 0$$
  et les $x_n$ sont une suite de Cauchy dans $E$ qui est complet, donc $x_n\tend x\in E$.

  Pour tout $m\geqslant n$, $x_n\in F_m\subset F_n$ donc ($F_n$ étant fermé) $x\in F_n$ donc 
  $$x\in \bigcap_{n\geqslant 0}F_n.$$
  De plus, pour $(a,b)\in \bigcap F_n$, 
  $$d(a,b)\leqslant \mathrm{diam}\ F_n\tend 0$$
  donc $\bigcap_{n\geqslant 0} F_n$ est un singleton, ce qui conclut.
\end{Proof}


\begin{thC}{de Baire, v1}
  Soit $(E,d)$ métrique complet, soit $(O_n)$ une suite d'ouverts tous denses dans $E$,\\

  alors 
  $\bigcap_{n\geqslant 0}O_n$ est dense dans $E$.
\end{thC}


\Dnote{
  Pour $O_n$ une suite d'ouverts, il est possible que $\bigcap_{n\geqslant 0}O_n$ ne soit pas ouvert puisque, par (contre) exemple
  $$\ens{1}=\bigcap_{n\geqslant 1}\big]1-\nicefrac{1}{n}, 1+\nicefrac{1}{n}\big[$$
  (par contre une réunion dénombrable d'ouverts est aussi un ouvert
}

\newpage
En considérant les fermés $F_n:=E\backslash O_n$, on obtient la formulation suivante du théorème de Baire :


\begin{thC}{de Baire, v2}
  Soit $(E,d)$ métrique complet, soit $(F_n)$ une suite de fermés d'intérieurs vides dans $E$,\\

  alors 
  $\bigcup_{n\geqslant 0}F_n$ est d'intérieur vide.
\end{thC}


\Dnote{
  De même, pour $F_n$ un suite de fermé, il est possible que $\bigcup_{n\geqslant 0}F_n$ ne soit pas fermés, puisque, par (contre) exemple
  $$]0,1[=\bigcup_{n\geqslant 1}[\nicefrac{1}{n},1-\nicefrac{1}{n}]$$
 (par contre une intersection dénombrable de fermés est aussi fermée)
}


\begin{Proof}
  Soit $O_n$ une suite d'ouverts, tous denses, soit $\Omega$ un ouvert de $E$, il s'agit de montrer que 
  $$\Omega\cap\bigcap_{n\geqslant 0}O_n\neq \emptyset.$$
  $O_0$ est un ouvert dense dans $E$, donc 
  $$\Omega\cap O_0$$
  est un ouvert non-vide et l'on peut prendre $x_0\in E,\ r_0\in ]0,1[$ tels que
  $$\overline{B(x_0,r_0)}\subset \Omega\cap O_0.$$
  $O_1$ est un ouvert dense dans $E$ donc 
  $$O_1\cap B(x_0,r_0)$$
  est un ouvert non-vide et l'on peut prendre $x_1\in E,\ r_1\in]0,\nicefrac12[$ tels que
  $$\overline{B(x_1,r_1)}\subset O_1\cap B(x_0,r_0)\subset \Omega\cap _0\cap O_1.$$
  On va construire par récurrence les $B_n=B(x_n,r_n)$ tels que
  $$\overline{B_{n+1}}\subset B_n\text{, }r_n\leqslant \nicefrac{1}{2^n}\text{ et }\overline{B_n}\subset \Omega\cap O_n.$$
  $\left(\overline{B_n}\right)$ est une suite décroissante de fermés dont le diamètre tend vers 0, donc le théorème des fermés emboîtés s'applique et $\bigcap_{n\geqslant 0}\overline{B_n}\neq\emptyset$ et puisque $\overline{B_n}\subset \Omega\cap O_n$, alors
  $$\Omega\cap\bigcap_{n\geqslant 0}O_n\neq\emptyset,$$
  autrement dit, $\bigcap_{n\geqslant 0}O_n$ est dense dans $E$.
\end{Proof}


On passe maintenant à la v2, \emph{i.e.} la formulation avec des fermés. Il va suffire pour ça de "passer" des fermés à des ouverts, puis d'appliquer la première version du théorème.\\


\begin{Proof}
  Soit $F_n$ une suite de fermés d'intérieurs vides, on pose 
  $$O_n:=E\backslash F_n$$
  qui est une suite d'ouverts et 
  $$\overline{O_n}=\overline{E\backslash F_n}=E\backslash \mathrm{int}\ {F_n}=E$$
  on peut donc appliquer la première version du théorème de Baire et $\bigcap_{n\geqslant 0}O_n$ est dense dans $E$ donc
  $$\mathrm{int}\ \bigcup_{n\geqslant 0}F_n=\mathrm{int}\ E\backslash \bigcap_{n\geqslant 0}O_n=E\backslash \overline{\bigcap_{n\geqslant 0}O_n}=\emptyset.$$
\end{Proof}



\begin{RQ}
  si l'on écrit $\mathbf Q$ (qui est dénombrable) sous la forme 
  $$\mathbf Q=\left(r_n\right)_{n\in\mathbf N}$$
  et que l'on choisit comme collection d'ouverts les 
  $$O_n:=\R\backslash \{r_n\}$$
  qui sont tous des ouverts denses dans $\R$, on a alors
  \begin{align*}
    \bigcap_{n\geqslant 0}O_n&=\bigcap_{n\geqslant 0}R\backslash \{r_n\}\\
    &=\R\backslash \bigcup_{n\geqslant 0}\{r_n\}\\
    \bigcap_{n\geqslant 0}O_n&=\R\backslash \mathbf Q
  \end{align*}
  qui n'est pas un ouvert.
\end{RQ}


\begin{RQ} 
  le théorème n'est pas valide pour une réunion non-dénombrable, si l'on pose par exemple
  $$O_x:=\R\backslash \{x\}$$
  alors 
  $$\bigcap_{x\in_R}O_x=\emptyset$$
  qui n'est pas dense dans $\R$.
\end{RQ}

\begin{RQ}
  petit rappel de topologie, pour $E$ un espace topologique et $X\subset E$, on a 
  $$\mathrm{int}\ E\backslash X = E\backslash \overline{X}.$$
\end{RQ}


\begin{Proof}
  $\star$ Soit $x\in\mathrm{int}\ E\backslash X$, donc il existe un voisinage $V$ de $x$ tel que 
  $$x\in V\subset E\backslash X$$
  donc $V\cap X=\emptyset$, autrement dit $x\notin \overline X$, donc $x\in E\backslash \overline{X}$.\\

  $\star$ Réciproquement, soit $x\in E\backslash \overline{X}$, puisque $\overline X$ est un fermé, $E\backslash\overline X$ est un ouvert donc il existe un voisinage $U$ de $x$ tel que 
  $$x\in V\subset E\backslash\overline X\subset E\backslash X$$
  donc $x$ est intérieur à $X\subset E\backslash X$, ce qui conclut.
\end{Proof}


\begin{corollaire}
  Soit $(E,d)$ métrique complet et $\left(F_n\right)$ une suite de fermées tels que 
  $$\bigcup_{n\geqslant 0}F_n=E,$$\\

  alors $\Omega:=\bigcup_{n\geqslant 0}\mathrm{int}\ F_n$ est un ouvert dense dans $E$, donc à fortiori il existe au moins un $\mathrm{int}\ {F_{n_0}}\neq\emptyset$.
\end{corollaire}

\begin{Proof}
  On a l'équivalence suivante 
  $$\Omega\text{ dense }\Leftrightarrow E\backslash \Omega\text{ d'intérieur vide}$$
  on remarque que 
  $$E\backslash\Omega=\bigcup_{k\geqslant 0}F_n\backslash\bigcup_{n\geqslant 0}\mathrm{int}\ F_n\subset \bigcup_{k\geqslant 0}F_k\backslash\mathrm{int}\ F_k.$$
  Et $\partial F_k=F_k\backslash\mathrm{int}\ F_k$ est un fermé d'intérieur vide, donc le théorème de Baire s'applique et $\bigcup_{k\geqslant 0}\partial F_k$ est d'intérieur vide, donc $E\backslash\Omega$ est aussi d'intérieur vide, donc d'après l'équivalence du début, $\Omega$ est bien dense dans $E$.
\end{Proof}

$\triangleright$\emph{ ex :} soient $(E,d)$ complet, $f_n:E\longrightarrow \R$ une suite de fonctions continues et $f$ la limite simple des $f_n$, alors 
$$\mathrm{cont}\ f:=\left\{x\in E\ |\ f\text{ est continue en }x\right\}$$
est dense dans $E$.


\begin{Proof}
  Soit $n,k\in \N\times\N^*$, on définit
  \begin{align*}
    F_n^k:&=\left\{x\in E\text{ t.q. }\forall p,\geqslant n,\ |f_p(x)-f_q(x)<\nicefrac{1}{k}\right\}\\
    &=\bigcap_{p,q\geqslant n}\left(f_p-f_q\right)^{-1}\left(\left[-\nicefrac{1}{k},\nicefrac{1}{k}\right]\right)
  \end{align*}
  Chacun des $F_n^k$ est fermé en tant que pré-image d'un fermé par une application continue.
  
  De plus, pour $k\in\N^*$ et $x\in E$, $f_n(x)$ converge donc est de Cauchy donc appartient à au moins un $F_n^k$, ainsi $E=\bigcup_{n\geqslant 0}F_n^k$.

  D'après le corollaire du théorème de Baire, $O_k:=\bigcup_{n\geqslant 0}\mathrm{int}\ F_n^k$ est un ouvert dense dans $E$ donc, en y ré-applicant le théorème de Baire, 
  $$\Omega:=\bigcap_{k\geqslant 0}\bigcup_{n\geqslant 0}\mathrm{int}\ F_n^k$$
  est un dense dans $E$, montrons que $\Omega\subset\mathrm{cont}\ f$.

  Soit $x_0\in\Omega$ et $\varepsilon>0$. Soit $k\geqslant 1$ tel que $\nicefrac{1}{k}\leqslant \varepsilon$.

  $x_0\in \Omega$ donc en particulier $x_0\in O_k=\bigcup_{n\geqslant 0}\inté F_n^k$, donc il existe $n_0$ tel que $x_0\in\inté F_{n_0}^k$ et 
  $$\exists r>0\tq B(x_0,r)\subset\inté F_n^k\subset F_n^k$$
  donc
  $$\forall x\in B(x_0,r),\ \forall p\geqslant n_0,\ |f_p(x)-f_{n_0}(x)|\leqslant \nicefrac{1}{k}\geqslant \varepsilon.$$
  Puisque $p\geqslant n_0$, on peut le faire tendre vers $+\infty$ et 
  $$\forall x\in B(x_0,r),\ |f(x)-f_{n_0}(x)|\leqslant \varepsilon.$$
  Par inégalité triangulaire, pour tout $x\in B(x_0,r)$, on a 
  \begin{align*}
    |f(x)-f(x_0)|&=|\big(f(x)-f_{n_0}(x)\big)+\big(f_{n_0}(x_0)-f(x_0)\big)+\big(f_{n_0}(x)-f_{n_0}(x_0)\big)|\\ 
    &\leqslant 2\varepsilon + |f_{n_0}(x)-f_{n_0}(x_0)|
  \end{align*}
  $f_{n_0}$ est continue en $x_0$ donc il existe $\tilde{r}$ tel que 
  $$\forall x\in B(x_0,\tilde{r}),\ |f_{n_0}(x)-f_{n_0}(x_0)|\leqslant \varepsilon$$
  donc, en posant $R:=\mathrm{min}(r, \tilde r)$, on a 
  $$\forall x\in N(x_0,R),\ |f(x)-f(x_0)|\leqslant 3\varepsilon$$
  donc $f$ est continue en $x_0$, ainsi $\Omega$ est bien dense dans $E$, donc $\mathrm{cont}\ f\supset\Omega$ est dense dans $E$
\end{Proof}


\begin{RQ}
  si $f:\R\longrightarrow \R$ est dérivable, alors $f'$ est continue sur un ensemble dense, puisque la suite de fonction 
  $$f_n(x)=\frac{1}{n}\Big(f\big(x-\nicefrac{1}{n}\big)-f(x)\Big)$$
  est une suite de fonctions continues qui CV vers $f'$, ainsi l'exemple s'applique.
\end{RQ}



\NouvelleSection{Quelques rappels sur les applications linéaires continues}


$X$ et $Y$ sont deux espaces vectoriels normés, on note 
$$L(X,Y):=\ens{T:X\longrightarrow Y\text{ linéaire}}$$

$T\in L(X,Y)$ est continue si, et seulement si $\exists c>0$ tel que
$$\forall x\in X,\ ||Tx||_Y\leqslant c||x||_X.$$

On note 
$$\mathcal L(X,Y):=\left\{T:X\longrightarrow Y\text{ linéaire continue}\right\}$$
Pour $T\in\mathcal L(X,Y)$, on note
$$||T||_{\mathrm{op}}:=\mathrm{inf}\left\{c>0\ |\ \forall x\in X,\ ||Tx||\leqslant c||x||\right\}$$
on montre que cette définition est équivalente à
$$\begin{array}{rcl}
  ||T||_{\mathrm{op}}&=&\mathrm{sup}\left\{||Tx||,\ x\in X\ |\ ||x||\leqslant 1\right\}\\
  &=&\mathrm{sup}\left\{||Tx||,\ x\in X\ |\ ||x||= 1\right\}\\
  ||T||_{\mathrm{op}}&=&\mathrm{sup}\left\{\dstyle\frac{||Tx||}{||x||},\ x\in X\backslash\{0\}\right\}
\end{array}$$
on montre aussi que cette norme est \emph{sous-multiplicative}, \emph{i.e.}
$$\forall T\in\mathcal L(Y,Z),\ S\in\mathcal L(X,Y),\ ||TS||_{\mathrm{op}}\leqslant ||T||_{\mathrm{op}} ||S||_{\mathrm{op}}$$
on généralise pour les itérations successives de T,
$$\forall n\in\N,\ ||T^n||_{\mathrm{op}}=||\underbrace{T\circ T\cdots\circ T}_{n\text{ fois}}||_{\mathrm{op}}\leqslant ||T||_{\mathrm{op}}^n.$$

\begin{prop}
  Soient $(X,Y)$ deux espaces vectoriels normés, où $Y$ est un Banach,\\

  alors $\mathcal L(X,Y)$ est un Banach (\emph{i.e.} un e.v.n. complet).
\end{prop}

\begin{Proof}
  Soit $(T_n)$ une suite de Cauchy dans $\mathcal L(X,Y)$, et soit $\varepsilon>0$,
  $$\exists n_0\in\N\text{ t.q. }\forall p,q\geqslant n_0,\ ||T_p-T_p||_{\mathrm{op}}\leqslant \varepsilon$$
  soient $x\in X,\ p,q\geqslant n_0$
  $$||T_px-T_qx||=||\left(T_p-T_q\right) x||\leqslant \nop{T_p-T_q}\ ||x||\leqslant \varepsilon ||x||\qquad (i)$$
  donc $(T_n(x))$ est une suite de Cauchy à valeurs dans $Y$, donc elle est convergente et on peut poser
  $$T(x):=\underset{n\to +\infty}{\mathrm{lim}}T_n(x).$$
  Montrons que $T$ est bien linéaire, soient $(u,v)\in X,\ \lambda\in\K$,
  \begin{flalign*}
    T(\lambda u+v)&=\underset{n\to +\infty}{\mathrm{lim}}T_n(\lambda u+v)\\
    &=\underset{n\to +\infty}{\mathrm{lim}}\lambda T_n(u)+\underset{n\to +\infty}{\mathrm{lim}}T_n(v)\\
    &=\lambda\underset{n\to +\infty}{\mathrm{lim}} T_n(u)+\underset{n\to +\infty}{\mathrm{lim}}T_n(v)\\
    T(\lambda u+v)&=\lambda T(u)+T(v)
  \end{flalign*}
  donc $T$ est bien linéaire et pour montrer la continuité de $T$, on fait tendre $p\to +\infty$ dans $(i)$, on obtient alors
  $$\nop{T-T_q}\ ||x||\leqslant \varepsilon||x||$$
  donc $T-T_p$ est continue, et puisque $T_q$ est continue, donc $T$ l'est et $T\in\mathcal L(X,Y)$.
\end{Proof}


\begin{Def}
  Soient $X,Y$ deux espaces vectoriels, un \emph{opérateur compact} est une application continue (un \emph{opérateur} continu donc) $T:X\longrightarrow Y$ tel que pour tout $A\in X$ borné, $T(A)$ est une partie relativement compacte de $Y$.

  Si la topologie sur $X$ est la topologie métrique habituelle, alors $T$ est compact si, et seulement si $T(B(0,1))$ est une partie relativement compacte.\\
\end{Def}


$\triangleright$\emph{ ex : }soient $-\infty<a<b< \infty$ et $K:[a,b]\times[a,b]\longrightarrow \mathbf C$ continue. Pour $f\in C([a,b],\mathbf C)$ et $x\in[a,b]$, on pose 
$$T_K(f)(x):=\int_a^bK(x,y)f(y)\mathrm dy.$$
$\star$ Vérifions que $T_K:C([a,b],\mathbf C)\longrightarrow C([a,b],\mathbf C)$

$(x,y)\longmapsto K(x,y)f(y)$ est continue et on intègre des fonctions continues sur un segment, donc $T_K(f)$ est bien continue. Par linéarité de l'intégrale, $T_K$ est aussi linéaire.

Soit $x\in [a,b]$,
$$|T_K(f)(x)|\leqslant ||K||_{\infty}(b-a)^2||f||_{\infty}=c||f||_{\infty}$$
avec $||f||_{\infty}<\infty$ car $f$ est continue sur un segment, donc $T_K$ est bien une application linéaire continue.

$\star$ Montrons que $T_K$ est compact, \emph{i.e.} montrons que 
$$T_K(\overline{B(0,1)})$$
est relativement compact dans $C([a,b],\C)$.

On a
$$T_K(\overline{B(0,1)})=\left\{T_K(f),\ f\in C([a,b],\C)\text{ t.q.}\ ||f||_{\infty}\leqslant 1\right\}.$$
$\star\star$ Montrons que la famille 
$$\left\{T_K(f),\ f\in\overline{B(0,1)}\right\}$$
est équicontinue, soit $\varepsilon>0$ et $x\in[a,b]$, alors pour tout $z\in[a,b]$
$$T_K(f)(x)-T_K(f)(z)=\int_a^b\left[K(x,y)-K(z,y)\right]f(y)\mathrm dy\qquad (i)$$
$K$ est continue sur le compact $[a,b]\times[a,b]$ donc $K$ est équicontinue et il existe $\delta>0$ tel que
$$\forall (x,z)\in[a,b]\times[a,b],\ |x-z|\leqslant\delta\Longrightarrow \forall y\in [a,b],\ |K(x,y)-K(z,y)|\leqslant \varepsilon$$
soit $y\in B(x,\delta)$, d'après $(i)$ on a 
\begin{flalign*}
  |T_K(f)(x)-T_K(f)(z)|&\leqslant \varepsilon \int_a^b|f(y)\mathrm dy\\
  &\leqslant \varepsilon ||f||_{\infty}(b-a)\\
  |T_K(f)(x)-T_K(f)(z)|&\leqslant \varepsilon (b-a)
\end{flalign*}
on va donc changer de $\delta$, on va prendre le $\tilde\delta$ défini à partir de $\nicefrac{\varepsilon}{b-a}$ et on a maintenant
$$\forall |x-a|\leqslant \tilde\delta,\ |T_K(f)(x)-T_K(f)(z)|\leqslant \varepsilon$$
donc la famille est bien équicontinue et on peut appliquer le théorème d'Ascoli, et 
$$T_K(\overline{B(0,1)})$$
est relativement compacte.

\begin{prop}
  Soient $X$ un espace vectoriel normé, $Y$ un espace de Banach et $E\subset X$ un sous-espace dense, soit $T\in \mathcal L(E,Y)$,\\  

  alors il existe un unique $\tilde T\in \mathcal L(X,Y)$ tel que 
  $$\forall x\in E,\ \tilde Tx=Tx$$
  et plus, ces applications sont de normes égales, \emph{i.e.}
  $$||\tilde T||_{\mathrm{op},\ \mathcal L(X,Y)}=||T||_{\mathrm{op},\ \mathcal L(E,Y)}.$$
\end{prop}


\begin{Proof}
  La preuve est en trois partie; la construction de $\hat T$, s'assurer que cette application est continue puis s'assurer qu'elle est unique.\\

  $\star$ Pour la construction de $\hat T$, on considère $x\in X\backslash E$. Par densité il existe $e:\N\longrightarrow E$ telle que 
  $$e_n\tend e$$
  et pour $x\in E$, on considère la suite constante, donc pour tout $x\in X$, il existe une suite $e_n$ à valeur dans $E$ convergeant vers $x$.

  Montrons que $Te_n$ converge dans $Y$, on a 
  \begin{align*}
    ||Te_n-Te_m||&=||T(e_n-e_m)||\\
    ||Te_n-Te_m||&\leqslant \nop{T}\ ||e_n-e_m||
  \end{align*}
  la suite $(e_n)$ étant convergente, elle est de Cauchy donc $(Te_n)$ est aussi de Cauchy, et $Y$ étant un espace de Banach, cette suite est convergente et on peut poser, 
  $$\hat Tx:=\underset{n\to+\infty}{\mathrm{lim}}Te_n.$$
  Vérifions que $\hat T$ est bien définie, $\emph{i.e.}$ que la définition n'est pas fonction du choix de la suite convergeant vers $x$. Soient $x\in X$ et $e_n,e_n'$ deux suites convergeants vers $x$
  $$||Te_n-Te_n'||\leqslant||T||_{\mathrm{op}}\ ||e_n-e_n'||\tend0$$
  donc 
  $$Te_n-Te_n'\tend 0$$
  et $\hat Tx$ est correctement définie.\\

  $\star$ Pour la linéarité, soient $(x,y)\in X$, $\lambda\in\K$. On considère $y_n\longrightarrow y$ et $x_n\longrightarrow x$, alors
  \begin{align*}
    \hat T(\lambda x+y)&=\underset{n\to+\infty}{\mathrm{lim}}T(\lambda x_n+y_n)\\
    &=\lambda\underset{n\to+\infty}{\mathrm{lim}}Tx_n+\underset{n\to+\infty}{\mathrm{lim}}Ty_n\\
    \hat T(\lambda x+y)&=\lambda\hat Tx+\hat Ty
  \end{align*}
  et $\hat T$ est bien une application linéaire.\\

  $\star$ Pour la continuité de $\hat T$, soient $x\in X$ et $x_n\longrightarrow x$, on sait, puisque $T$ est continue, que 
  $$\forall n\in\N,\ ||Te_n||_{\mathrm{op}}\leqslant ||T||\ ||e_n||$$
  par passage à la limite simple
  $$||\hat Tx||\leqslant ||T||_{\mathrm{op}}\ ||x||$$
  donc $\hat T$ est continue, et on a même l'inégalité des normes $||\hat T||\leqslant ||T||$.\\

  $\star$ Quant aux normes, on a $\hat T\big|_E=T$ donc pour $e\in E$,
  $$||Te||=||\hat Te||\leqslant ||\hat T||\ ||e||$$
  donc $||T||\leqslant||\hat T||$ ainsi on a l'égalité des normes recherchée.\\

  $\star$ Pour l'unicité, on suppose qu'il existe $U$ distinct de $\hat T$ qui convient, alors $\forall x\in X$, il existe $x_n\longrightarrow x$ une suite d'élements  de $E$, alors
  \begin{align*}
    U(x)&=\underset{n\to+\infty}{\mathrm{lim}} Ue_n\\
    &=\underset{n\to+\infty}{\mathrm{lim}}Te_n\\
    &=\underset{n\to+\infty}{\mathrm{lim}}\hat Te_n\\
    Ux&=\hat Tx
  \end{align*}
  ainsi $U=\hat T$, donc $\hat T$ est effectivement unique.
\end{Proof}


\begin{thC}{Banach-Steinhaus}
  Soient $X$ un espace de Banach, $Y$ un espace vectoriel normé et $\left\{T_i,\ i\in I\right\}$ une suite quelconque d'applications linéaires continues $X\longrightarrow Y$,\\
  
  alors exactement l'une des assertions suivantes est vraie\\
  $(i)$ $\underset{i\in I}{\mathrm{sup}}||T_i||_{\mathrm{op}}<\infty$\\[1em]
  $(ii) $ $\Big\{x\in X\text{ t.q. } \underset{i\in I}{\mathrm{sup}}||T_ix||=+\infty\Big\}$ est dense dans $X$.
\end{thC}

\begin{Proof}
  On suppose que l'assertion $(ii)$ est fausse, donc que 
  $$\left\{x\in X\text{ t.q. } \mathrm{sup}_{i\in I}||T_ix||=+\infty\right\}$$ 
  n'est pas dense dans $X$, autrement dit son complémentaire, noté $F$ est d'intérieur non vide.

  Soit $n\geqslant 0$, on pose 
  \begin{align*}
    F_n:&=\left\{x\in X\text{ t.q. }\underset{i\in I}{\mathrm{sup}}||T_ix||\leqslant n\right\}\\
    F_n&=\bigcap_{i\in I}T_i^{-1}\left(\overline{B(0,1)}\right)
  \end{align*}
  Les $F_n$ sont tous fermés car ils sont des intersections de pré-images de fermés par l'application continue $T_i$, donc le théorème de Baire s'applique et 
  $$\exists n_0\geqslant 0, x_0\in X, r>0\text{ t.q. }\overline{B(x_0,r)}\subset F_{n_0}$$
  pour $x\in X$ tel que $||x||\leqslant r$, on écrit
  $$x=\frac12\left((x+x_0)+(x-x_0)\right)$$
  donc, puisque $T_i$ est linéaire
  $$T_ix=\frac12T(x_0+x)-\frac12T(x_0-x)$$
  et puisque $x_0+r$ et $x_0-x$ sont tous deux dans $\overline{B(x_0,r)}$, 
  $$\left\{\begin{array}{l}
    T_i(x_0+x)\in F_{n_0}\\
    T_i(x_0-x)\in F_{n_0}
  \end{array}\right.$$
  d'où la majoration suivante 
  $$||T_ix||\leqslant n_0.$$
  Dans le cas général, pour $u\in X$, on pose $x:=r\frac{u}{||u||}$ (on suppose $u\neq 0$), alors $||x||\leqslant r$ et on est dans le cadre du cas précédent, 
  \begin{align*}
    ||T_iu||&=\norm{T_i\left(\frac{||u||}{r}x\right)}\\
    &=\frac{||u||}{r}\norm{T_i\left(x\right)}\\
    ||T_iu||&\leqslant \frac{||u||}{r} n_0
  \end{align*}
  en faisant varier $||u||$ sur le disque unité, on a (passage au sup)
  $$||T_i||_{\mathrm{op}}\leqslant \frac{n_0}{r}$$
  ainsi 
  $$\mathrm{sup}_{i\in I}||T_i||_{\mathrm{op}}<\infty$$
  et l'on est bien dans le cas $\mathit 1$.
\end{Proof}


\begin{corollaire}
  Soient $X$ un espace de Banach, $Y$ un espace vectoriel normé et $\left\{T_i,\ i\in I\right\}$ une suite quelconque d'applications linéaires continues$X\longrightarrow Y$, si 
  $$\forall x\in X,\ \mathrm{sup}_{i\in I}||T_ix||<\infty,$$

  alors $\mathrm{sup}_{i\in I}||T_i||<\infty.$
\end{corollaire}


\begin{Proof}
  C'est une application immédiate du théorème de Banach-Steinhaus, on est pas dans le cas $(ii)$ donc on est nécessairement dans le cas $(i)$.
\end{Proof}

$\triangleright$\emph{ ex : }application au séries de Fourier, pour $f:[-\pi,\pi]\longrightarrow\mathbf C$ intégrable, on définit les coefficients de Fourier comme suit :
$$\forall k\in \N,\ \hat f(k):=\frac{1}{2\pi}\int_{-\pi}^{\pi}f(t)\ex{-ikt}\mathrm dt$$
et la somme partielle de Fourier comme 
$$S_n(f)(t):=\sum_{k=-n}^n\hat f(k)\ex{ikt}$$

On montre que 
$$\left\{f\in C([-\pi,\pi])\tq \underset{n\geqslant 0}{\mathrm{sup}}|S_nf(0)|=+\infty\right\}$$
est dense dans $C([-\pi,\pi])$, muni de la norme $||\cdot||_{\infty}$.

\begin{Proof}
  Soit, pour $n\in \N$, 
  $$\ell_n: 
  \begin{array}{|rcl}
    \mathcal C[-\pi,\pi]&\longrightarrow&\R\\
    f&\longmapsto&S_nf(0)\\
  \end{array}$$
  $\ell_n$ est une application linéaire.\\

  $\star$ Montrons que $\ell_n$ est continue, soit $f\in\mathcal C[-\pi,\pi]$, on calcule 
  \begin{align*}
    \ell_n(f)&=\sum_{k=-n}^n\frac{1}{2_i}\int_{-\pi}^{\pi}f(t)\ex{-ikt}\dt\\
    \ell_n(f)&=\frac{1}{2\pi}\int_{-\pi}^{\pi}f(t)\sum_{k=-n}^n\ex{-ikt}\dt
  \end{align*}
  On définit 
  $$D_n(t):=\sum_{k=-n}^n\ex{-ikt}$$
  le noyau de Dirichlet, que l'on peut calculer
  $$D_n(t)=\left\{\begin{array}{rcr}
    &\dstyle\frac{\sin(\nicefrac{2n+1}{2}t)}{\sin(\nicefrac{t}{2})}&\text{ si }t\neq 0\\[1em]
    & 2n+1&\text{ si }t=0  
  \end{array}\right.$$
  Avec ces notations, on a 
  \begin{align*}
    |\ell_n(f)|&=\frac{1}{2\pi}\left|\int_{-\pi}^{\pi}f(t)D_n(t)\dt\right|\\
    &\leqslant ||f||_{\infty}\frac{1}{2\pi}\left|\int_{-\pi}^{\pi}D_n(t)\dt\right|\\
    |\ell_n(f)|&\leqslant ||f||_{\infty}\frac{1}{2\pi}\int_{-\pi}^{\pi}\left|D_n(t)\right|\dt\qquad (i)
  \end{align*}
  donc $\ell_n$ est continue.\\

  $\star$ Montrons que $\underset{n\geqslant 0}{\mathrm{sup}}\ ||\ell_n||=+\infty$

  On montre d'abord que $||\ell_n||=\dstyle\frac{1}{2\pi}\int_{-\pi}^{\pi}|D_n|\dt$. 
  
  $\star\star$ L'inégalité $(i)$ nous donne la moitié de l'égalité
  $$||\ell_n||\leqslant\frac{1}{2\pi}\int_{-\pi}^{\pi}\left|D_n(t)\right|\dt.$$
  $\star\star$ On va chercher à montrer l'autre sens de cette inégalité, soit $\varepsilon>0$ et 
  $$f_\varepsilon:t\longmapsto \frac{D_n(t)}{|D_n(t)|+\varepsilon}$$
  $\sin$ étant $2\pi$-périodique, $f_\varepsilon$ l'est aussi donc $f_\varepsilon\in\mathcal C[-\pi,\pi]$ et 
  $$\ell_n(f_\varepsilon)=\frac{1}{2\pi}\int_{-\pi}^{\pi}\frac{D_n^2(t)}{|D_n(t)|+\varepsilon}\dt$$
  De plus, 
  $$\underset{\varepsilon\to 0^+}{\mathrm{lim}}\frac{D_n^2(t)}{|D_n(t)|+\varepsilon}=|D_n(t)|$$
  et on peut dominer l'intégrande par $|D_n|$, qui est dans $\mathcal L_1$, le théorème de converge dominée s'applique et
  $$\underset{\varepsilon\to 0^+}{\mathrm{lim}}\frac{1}{2\pi}\int_{-\pi}^{\pi}\frac{D_n^2(t)}{|D_n(t)|+\varepsilon}\dt=\frac{1}{2\pi}\int_{-\pi}^{\pi}|D_n(t)|\dt.$$
  De plus, $\forall \varepsilon>0,\ ||f_\varepsilon||_\infty\leqslant 1$ donc $||f_\varepsilon||_1\leqslant 1$ et $||\ell_n||\geqslant |\ell_n(f_\varepsilon)|$, donc par passage à la limite, 
  $$||\ell_n||\geqslant \frac{1}{2\pi}\int_{-\pi}^{\pi}|D_n(t)|\dt.$$
  Ainsi on a les deux majorations / minorations, donc 
  $$||\ell_n||= \frac{1}{2\pi}\int_{-\pi}^{\pi}|D_n(t)|\dt.$$
  On peut maintenant passer au calcul de $||\ell_n||$;
  \begin{align*}
    ||\ell_n||&=\frac{1}{\pi}\int_{-\pi}^{\pi}|D_n(t)|\dt\\
    &=\frac{1}{\pi}\int_{-\pi}^{\pi}\left|\frac{\sin\left(t\nicefrac{(2n+1)}{2}\right)}{\sin\left(\nicefrac{t}{2}\right)}\right|\dt\\
    &\geqslant\frac{1}{\pi}\int_{-\pi}^{\pi}\frac{\left|\sin\left(t\nicefrac{(2n+1)}{2}\right)\right|}{|t|}\dt\\
    &\geqslant\frac{2}{\pi}\int_0^{\pi}\frac{\left|\sin\left(t\nicefrac{(2n+1)}{2}\right)\right|}{|t|}\dt\\
    &\geqslant \frac{2}{\pi}\int_0^{\nicefrac{2n+1}{2}}\frac{|\sin u|}{\nicefrac{2u}{2n+1}}\frac{2\mathrm du}{2n+1}\\
    ||\ell_n||&\geqslant \frac{2}{\pi}\int_0^{\nicefrac{2n+1}{2}}\frac{|\sin u|}{u}\mathrm du\\
  \end{align*}
  et puisque 
  $$\frac{2}{\pi}\int_0^{\nicefrac{2n+1}{2}}\frac{|\sin u|}{u}\mathrm du\longrightarrow\dstyle\int_{0}^{\infty}\frac{|\sin u|}{u}\mathrm du = \infty\text{ quand }n\to\infty,$$
  $||\ell_n||\longrightarrow\infty$, donc $\underset{n\geqslant 0}{\mathrm{sup}}\ ||\ell_n||=+\infty$. Ainsi, d'après le théorème de Banach-Steinhaus,
  $$\ens{f\in C\Par{[-\pi,\pi]}\ :\ \underset{n\geqslant 0}{\mathrm{sup}}\norm{\ell_n(f)}}$$
  est dense dans $C\Par{[-\pi,\pi]}$ et puisque $\ell_n(f)=S_n(f)(0)$, on obtient le résultat annoncé.
\end{Proof}


\begin{thC}{de majoration automatique}
  Soient $(X,Y)$ des espaces de Banach et $T\in\mathcal L(X,Y)$, si $T$ est surjective,\\

  alors il existe $c>0$ tel que $\forall y\in Y,\ \exists x\in X$ vérifiant
  $$y=Tx\text{ et }||x||\leqslant c||y||=c||Tx||$$
\end{thC}


\begin{Proof}
  $\star$ On commence par montrer qu'il existe $M>0$ tel que $B(0,1)\subset \overline{T\left[B(0,M)\right]}$. On a l'écriture suivante pour $X$,
  $$X=\bigcup_{n\geqslant 1}B(0,)$$
  Par surjectivité de $T$, on a aussi
  $$Y=T(X)=\bigcup_{n\geqslant 1}T\left[B(0,n)\right]$$
  et on a aussi, en posant $F_n:=\overline{T\left[B(0,n)\right]}$
  $$Y=\bigcup_{n\geqslant 1}F_n.$$
  Les $F_n$ sont tous fermés et $Y$ est un espace de Banach, on peut donc y appliquer le théorème de Baire, donc il existe $n_0$ tel que $F_{n_0}$ est d'intérieur non vide, donc
  $$\exists y_0\in Y,\ \exists r>0\tq B(y_0,r)\subset F_{n_0}\subset \overline{T\left[B(0,n)\right]}.$$

  Montrons que $B(y_0,r)\subset F_{n_0}$, soit $y\in B(0,r)$ alors 
  $$y_0-y,y_0=Y\in B(y_0,r)$$
  donc, pour $(u_n,v_n)\in B(y_0,n_0)$ tellle que 
  $$T(u_n)\longrightarrow y_0+y\text{ et }T(v_n)\longrightarrow y_0-y,$$
  $$T\left(\frac12\left(u_n+v_n\right)\right)\longrightarrow y$$
  et $\forall n\in\N,\ \frac12\left(u_n+v_n\right)\in B(0,n_0).$

  Puisque $F_{n_0}$ est un fermé, $y\in F_{n_0}$ et $B(0,r)\subset\overline{T\left[B(0,n_0)\right]}$, autrement dit (linéarité ? oui), en posant $M:=\nicefrac{n_0}{r}$, on obtient 
  $$B(0,1)\subset \overline{T\left[B(0,M)\right]}.$$
  $\hfill\square$\\

  $\star$ On va maintenant montrer qu'il existe $c_1>0$ tel que $B(0,1)\subset\left[B(0,c_1)\right].$ Soit $z_0\in B(0,1)\subset \overline{T\left[B(0,M)\right]}$ alrs il existe $x_0\in B(0,M)$ tel que
  $$||z_0-Tx_0||<\frac12$$
  on pose $z_1:=z_0-Tx_0$, alors $z_1\in B(0,\nicefrac12)\subset \subset \overline{T\left[B(0,\nicefrac{M}{2})\right]}$ donc il existe $x_1\in B(0,\nicefrac{M}{2})$ tel que 
  $$||z_1-Tx_1||<\frac14$$
  on pose (encore) $z_2:=z_1-Tx_1$ et on montre (par récurrence) qu'il existe $x_n,z_n$ telles que 
  $$\forall n\in \N,\ z_n\in B\left(0,\frac{1}{2^n}\right),\ x_n\in B\left(0,\frac{M}{2^n}\right)\tq z_{n+1}=z_n-Tx_n$$
  Ainsi 
  $$\sum_{n=0}^\infty ||x_n||<\sum_{n=0}^\infty\frac{M}{2^n}=2M$$
  $X$ étant un espace de Banach, la convergence absolue implique la convergence et l'on peut définir 
  $$x:=\sum_{n=0}^\infty x_n$$
  et $x\in B(0,2M)$.

  On peut calculer $Tx$,
  $$Tx=\sum_{n=0}^\infty Tx_n=\sum_{n=0}^\infty z_n-z_{n+1}$$
  et puisque $z_n\longrightarrow 0$, on a alors $Tx=z_0$ et $c_1:=2M$ convient.
  $\hfill\square$\\

  $\star$ Pour conclure, soit $y\in Y\backslash \{0\}$ et $z:=\frac{y}{2||y||}$, alors $z\in B(0,1)$ donc il existe $x_1\in X$ tel que $||x_1||\leqslant c_1$ et $z=Tx_1$, donc
  $$y=T(2||y||x_1)=Tx$$
  (en posant $x:=2||y||x_1$) et l'on a 
  $$||x||=2||y||\ ||x_1||<2c_1||y||.$$
\end{Proof}


\begin{thC}{d'isomorphisme de Banach}
  Soient $X,Y$ est espaces de Banach et $T\in \mathcal L(X,Y)$, si $T$ est bijective,\\

  alors $T^{-1}\in \mathcal L(Y,X)$.
\end{thC}

\begin{Proof}
  D'après le théorème de majoration automatique, il existe $c>0$ tel que
  $$\forall y\in Y,\ \exists x\in X\tq y=Tx\text{ et }||x||\leqslant c||y||$$
  puisque $T$ est bijective, le $x$ est unique, donc on peut ré-écrire 
  $$\forall x\in X,\ ||x||\leqslant c||y||$$
  et puisque $x=T^{-1}y$, on a 
  $$\forall y\in Y,\ ||T^{-1}y||\leqslant c||x||$$
  autrement dit, $T^{-1}$ est continue. 
\end{Proof}


\begin{thC}{de l'application ouverte}
  Soient $X,Y$ des espaces de Banach et $T\in\mathcal L(X,Y)$, si $T$ est surjective,\\

  alors $T$ est ouverte, \emph{i.e.} $\forall O$ ouvert, $T(O)$ est un ouvert.
\end{thC}


\begin{Proof}
  On applique le théorème de majoration automatique, donc il existe $c>0$ tel que 
  $$B(0,1)\subset T\left(B(0,c)\right),$$
  ce qui conclut.
\end{Proof}


\begin{prop}
  Soient $X$ un espace vectoriel normé et $||\cdot||_1$, $||\cdot||_2$ deux normes sur $X$. On suppose que $X$ est complet pour les deux normes et qu'il existe $c>0$ tel que pour tout $x\in X$, $||x||_2\leqslant c||x||_1$,\\ 

  alors $||\cdot||_1$ et $||\cdot||_2$ sont équivalentes.
\end{prop}


\begin{Proof}
  Soit $\mathrm{\Id}:(X,||\cdot||_1)\longrightarrow(X,||\cdot||_2)$, cette application est bijective et linéaire. L'hypothèse $||\cdot||_2\leqslant c||\cdot||_1$ nous indique que $\mathrm{\Id}$ est continue, donc le théorème d'isomorphisme de Banach s'applique et $\mathrm{\Id}^{-1}:(X,||\cdot||_2)\longrightarrow(X,||\cdot||_1)$ est continue, donc 
  $$\exists c_2>0\ |\ \forall x\in X,\ |x||_1\leqslant c||x||_2$$
  et les normes $||\cdot||_1$ et $||\cdot||_2$ sont effectivement équivalentes.
\end{Proof}

\NouvelleSection{Théorème du graphe fermé}


\begin{thC}{du graphe fermé}
  Soient $X,Y$ deux Banach, $T\in L(X,Y)$, on suppose que pour $x_n\in X$ tel que $x_n\tend x\in X$, et $Tx_n\tend y\in Y$, alors $Tx=y$,\\

  alors $T$ est continue
\end{thC}

\begin{Proof}
  On muni l'espace $X\times Y$ de la norme
  $$||(x,y)||_{(X\times Y)}:=\mathrm{max}(||x||,||y||)$$
  et on a alors les inégalités
  $$\forall (x,y)\in X\times Y,\ ||(x,y)||_{(X\times Y)}\geqslant ||x||\text{ et }||(x,y)||_{(X\times Y)}\geqslant ||y||.$$
  $\star$ Montrons que $(X\times Y, ||\cdot||_{(X\times Y)})$ est un espace de Banach, soit $Z_n=(x_n,y_n)$ une suite de Cauchy à valeurs dans $X\times Y$ est $\varepsilon>0$, alors 
  $$\exists N\geqslant 0\tq\forall (p,q)\geqslant N,\ ||Z_p-Z_q||_{(X\times Y)}\leqslant \varepsilon$$
  donc
  $$\forall (p,q)\geqslant N,\ \mathrm{max}(||x_p-x_p||,||y_p-y_p||)\leqslant \varepsilon$$
  donc
  $$\left\{\begin{array}{c}
    ||x_p-x_q||\leqslant \varepsilon\text{ donc }(x_n)\text{ est de Cauchy, donc }\exists x\tq x_n\longrightarrow x\\
    ||y_p-y_q||\leqslant \varepsilon\text{ donc }(y_n)\text{ est de Cauchy, donc }\exists y\tq y_n\longrightarrow y
  \end{array}\right.$$
  et on vérifie alors que $Z_n\longrightarrow (x,y)$.$\hfill\square$\\

  $\star$ Soit 
  $$G_x(T):=\left\{(x,Tx),\ x\in X\right\}\subset X\times Y$$
  le graphe de $T$, montrons qu'il est fermé, soit $(x_n,Tx_n)$ une suite dans $G_x(T)$ telle que 
  $$(x_n,Tx_n)\tend (x,y)\text{ dans }X\times Y$$
  donc 
  $$\left\{\begin{array}{c}
    x_n\tend x\\
    Tx_n\tend y
  \end{array}\right.$$
  donc par hypothèse, $Tx=y$ et $(x,y)\in G_x(T)$ ce qui montre que le graphe est bien fermé.\\

  $\star$ On note $\pi_X:G_x(T)\longrightarrow X$ et $\pi_Y:G_x(T)\longrightarrow Y$ les projecteurs, on a 
  $$\begin{array}{rcccr}
    ||\pi_x(x,Tx)||&=&||x||&\leqslant&||(x,Tx)||_{(X\times Y)}\\
    ||\pi_x(x,Tx)||&=&||Tx||&\leqslant&||(x,Tx)||_{(X\times Y)}\\
  \end{array}$$
  donc ces deux applications sont (linéaires) continues et de surcroît $\pi_x$ est bijective, donc on peut calculer
  \begin{flalign*}
    \forall x\in X,\ \pi_y\circ\pi_x^{-1}(x)&=\pi_y\left(\pi_x^{-1}(x)\right)\\
    &=\pi_y\left(x,Tx\right)\\
    \pi_y\circ\pi_x^{-1}(x)&=Tx
  \end{flalign*}
  ainsi $T=\pi_y\circ\pi_x^{-1}$ est continue
\end{Proof}


$\triangleright$\emph{ex :} application en analyse complexe, soit 
$$H^2:=\left\{f\in\mathrm{Hol}\ D(0,1),\ f(z)=\sum_{n\in\mathbf Z}a_n z^n \tq \sum_{n\in\mathbf Z}|a_n|^2=||f||_2^2<\infty\right\}$$
l'espace de Hardy, qui est un espace de Banach car l'application
$$T:\begin{array}{|ccc}
  \ell^2(\N)&\longrightarrow&H^2\\
   a_n&\longmapsto&f(x)=\dstyle\sum_{n\in\mathbf Z}a_n z^n
\end{array}$$
est un isomorphisme isométrique, donc la complétude de $ \ell^2(\N)$ implique la complétude de $H^2$.\\

On pose, pour $|\lambda|<1$, $E_\lambda:f\in H^2\longmapsto f(\lambda)$ et l'on a (inégalité de Cauchy-Schwarz),
$$E_{\lambda}(f)|=\left|\sum_{n\in\mathbf Z}a_n\lambda^n\right|\leqslant \sqrt{\sum_{n\in\mathbf Z}|a_n|^2}\sqrt{\sum_{n\in\mathbf Z}|\lambda|^{2n}}\longrightarrow 0\text{ quand }||f||_2\to 0$$
donc $E_\lambda$ est (linéaire) continue en 0, $E_\lambda$ est continue, ainsi pour $f_n\to f$ dans $H^2$, on a 
$$\forall \lambda\in D(0,1),\ f_n(\lambda)\longrightarrow f(\lambda).$$
Soit $X\subset H^2$ un espace de Banach de fonctions holomorphes sur $D(0,1)$ telle que 
$$f_n\longrightarrow f\text{ dans }X\Rightarrow\forall \lambda\in D(0,1),\ f_n(\lambda)\longrightarrow f(\lambda)$$
alors $i:X\hookrightarrow H^2$ (c'est l'\emph{injection canonique}) est continue (au sens de la topologie induite) :

\begin{Proof}
  $i$ est linéaire, $H^2$ et $X$ sont complets et soit $f_n$ convergeante dans $X$, donc il existe $f\in X$ telle que $f_n\longrightarrow f$ dans $X$. $f_n$ est une suite de fonction convergeante, donc il existe aussi $g\in H^2$ telle que $f_n\longrightarrow g$ dans $H^2$.

  Par hypothèse, on a 
  $$\left\{\begin{array}{c}
    f_n(\lambda)\tend f(\lambda)\\
    f_n(\lambda)\tend g(\lambda)
  \end{array}\right.$$
  par unicité de la limite, $f(\lambda)=g(\lambda)$, d'où $i(f)=g$.

  On peut donc appliquer le théorème du graphe fermé et $i$ est continue.
\end{Proof}


Pourquoi ce résultat est non-trivial ? question pour demain.



\NouvelleSection{Théorème de Hahn-Banach}

Soit $E$ un espace vectoriel normé, on note $E^*$ ou $E'$ l'ensemble des formes linéaires continues
$$E^*:=\left\{\varphi:E\longrightarrow \R\text{ linéaire continue }\right\},$$ 
muni de la norme d'opérateur 
$$||\varphi||=\underset{x\in E\backslash\{0\}}{\mathrm{sup}}\frac{|\varphi(x)|}{||x||}.$$
On montre que c'est un espace de Banach, \emph{i.e.} un espace vectoriel normé complet.

$\triangleright$\emph{ex : }si $\dim E<\infty$, en considérant $(e_1,\cdots, e_n)$ une base de $E$, alors 
$$\forall \varphi\in E^*,\ \varphi(x)=\sum_{i=1}^n\varphi(e_i)x_i$$
où $x=(x_1,\cdots,x_n)$.

$\triangleright$\emph{ex : }pour les espaces $\ell^p(\N)$ avec $p<\infty$, on note $q$ le conjugué de $p$, on se donne $a=\left(a_n\right)\in\ell^q(\N)$, on pose 
$$\varphi_a:\begin{array}{|ccc}
  \ell^p(\N)&\longrightarrow&\C\\
  u&\longmapsto&\dstyle\sum_{n\in\N}u_na_n 
\end{array}$$
$\varphi_a$ est bien définie (inégalité de Hölder) et est continue, avec 
$$||\varphi_a||\leqslant ||a||_q.$$

On rappelle la propriété suivante : 


\begin{prop}
  Soit $\varphi:E\longrightarrow \C$ linéaire,\\

  alors $\varphi\in E^*\Leftrightarrow\ker\ \varphi$ est fermé dans $E$
\end{prop}


\begin{Proof}
  $\star$ si $\varphi$ est continue, alors 
  $$\ker\ \varphi=\varphi^{-1}(\{0\})$$
  est un fermé en tant que pré-image du fermé $\{0\}$ par l'application continue $\varphi$.\\

  $\star$ Réciproquement, on suppose que $\ker\ \varphi$ est fermé et on procède par l'absurde, c'est à dire qu'on suppose que $\varphi$ n'est pas continue, donc que $\mathrm{sup}_{||x||=1}|\varphi(x)|=+\infty$. Ainsi il existe une  suite $x_n\in E$ telle que 
  $$\forall n\in \N,\ ||x_n||=1\text{ et }|\varphi(x_n)|\tend +\infty.$$
  En particulier, il existe $N\geqslant 0$ tel que $x_n\neq 0$, pour $n\geqslant N$.

  On considère la suite $y_n$ telle que 
  $$\forall n\geqslant N,\ y_n:=x_N-\frac{\varphi(x_N)}{\varphi(x_n)}x_n,$$
  par le calcul, on vérifie que $\forall n\geqslant N,\ y_n\in\ker\ \varphi$.

  De plus, on a 
  $$||y_n-x_N||=\left|\frac{\varphi(x_N)}{\varphi(x_n)}\right|\ ||x_n||=\left|\frac{\varphi(x_N)}{\varphi(x_n)}\right|\tend 0.$$
  Donc $x_N$ est la limite d'une suite d'éléments de $\ker\ \varphi$ qui est un fermé, donc $x_N\in\ker\ \varphi$ ce qui est absurde, donc $\varphi$ est nécessairement continue.
\end{Proof}


Soit $E$ un $\mathbf K$-espace vectoriel normé et $F\subset E$ un sous-espace vectoriel, on prend $\varphi:F\longrightarrow \mathbf K$ linéaire continue, on va se demander si il existe un prolongement de $\varphi$ sur $E$ de même norme que $\varphi$. Plus formellement, on cherche $f:E\longrightarrow \mathbf K$ telle que 
$$f\big|_F=\varphi\text{ et }||f|_E=||\varphi||_F.$$


\begin{Def}
  Soit $E$ un $\R$-espace vectoriel, une application $p:E\longrightarrow\R$ est une \emph{fonctionnelle sous-linéaire} si, et seulement si\\
  $(\bullet)\ \forall (x,y)\in E,\ p(x+y)\leqslant p(x)+p(y)$\\
  $(\bullet\bullet)\ \forall x\in E,\ \forall \lambda\in \R_+,\ p(\lambda x)=\lambda p(x).$
\end{Def}


$\triangleright$\emph{ex :} toute forme linéaire est une fonctionnelle sous-linéaire.\\

$\triangleright$\emph{ex bis:} toute norme est une une fonctionnelle sous-linéaire.\\[2em]


\NouvelleSubsection{Parenthèse ensembliste}

On fait d'abord une parenthèse en théorie des ensembles, on se donne $(\mathcal E,\preccurlyeq)$ un \emph{ensemble ordonné}, \emph{i.e.} muni d'une relation d'ordre.

$A\subset \mathcal P(\mathcal E)$ est \emph{totalement ordonné} si, et seulement si 
$$\forall (x,y)\in A, x\preccurlyeq y\text{ ou }y\preccurlyeq x.$$

On dit que $(\mathcal E,\preccurlyeq)$ est \emph{inductif} si toute partie de $\mathcal E$ totalement ordonnée admet un majorant.

On dit que $z\in \mathcal E$ est un \emph{élément maximal} si, et seulement si 
$$\forall x\in \mathcal E, z\preccurlyeq x\Rightarrow z=x.$$


$\triangleright$\emph{ex :} soit $\O$ un ensemble de cardinal supérieur à 2, pour $A,B\in\mathcal P(\O)$, on pose
$$A\preccurlyeq B\Leftrightarrow A\subset B,$$
alors $(\mathcal P(\O),\preccurlyeq)$ est un ensemble ordonné non totalement ordonné donc $\O$ est un élément maximal.\\[1em]

On a le résultat / axiome suivant (en fait, c'est équivalent à l'axiome du choix);


\begin{thC}{lemme de Zorn}
  Tout ensemble non vide, ordonné et inductif possède un élément maximal.
\end{thC}


\begin{thC}{de Hahn-Banach}
  Soit $E$ un $\R$-espace vectoriel, $p:E\longrightarrow \R$ une fonctionnelle sous-linéaire, $V$ un sous-espace vectoriel de $E$, soit $\varphi:V\longrightarrow\R$ une forme linéaire telle que $\forall x\in V,\ \varphi(x)\leqslant p(x).$\\

  Alors il existe $\tilde\varphi:E\longrightarrow\R$ prolongeant $\varphi$ et majorée par $p$.
\end{thC}


Avant de prouver ce résultat, on pose la définition suivante : $h:G\longrightarrow\R$, où $G\subset E$ est un sous-espace vectoriel, est un \emph{prolongement admissible} si, et seulement si\\
$(\bullet)\ F\subset G$\\
$(\bullet\bullet)\ h\big|_F=\varphi$\\
$(\bullet\bullet\bullet)\ \forall x\in G,\ h(x)\leqslant p(x).$\\


On aura besoin du lemme suivant : 


\begin{lemme}
  Soit $h:G\longrightarrow\R$ un prolongement admissible de $\varphi$ et $x_0\in E\backslash G$, alors\\

  il existe $\tilde h:G\oplus x_0\R\longrightarrow\R$ qui est un prolongement admissible de $\varphi$.
\end{lemme}


\begin{Proof}
  on pose $G_1:= G\oplus x_0\R\subset E$ et 
  $$\tilde h:\begin{array}{|ccc}
    G_1&\longrightarrow&\R\\
    x+\lambda x_0&\longmapsto&\varphi(x)+\lambda\alpha
  \end{array}$$
  où $\alpha$ est un paramètre à choisir de façon à avoir $\tilde h\big|_F=\varphi$ et $\tilde h(x+\lambda x_0)=\varphi(x)+\lambda\alpha\leqslant p(x+\lambda _0)$. 


  Puisque $\tilde h$ est linéaire, il nous faut \og juste\fg\ 
  $$\forall \lambda>0,\ \forall x\in \tilde G,\ \begin{cases}
    \varphi(x)+\alpha\lambda\leqslant p(x+\lambda x_0)\\
    \varphi(x)-\alpha\lambda\leqslant p(x-\lambda x_0)
  \end{cases}$$
  c'est équivalent à 
  $$\forall \lambda>0,\ \forall x\in\tilde G,\ \varphi\left(\frac{x}{\lambda}\right)-p\left(\frac{x}{\lambda}-x_0\right)\leqslant \alpha\leqslant p\left(\frac{x}{\lambda}+x_0\right)-\varphi\left(\frac{x}{\lambda}\right).$$
  Ainsi pour qu'$\alpha$ existe, il suffit que 
  $$\mathrm{sup}\left\{\varphi\left(v\right)-p\left(v-x_0\right), v\in G\right\}\leqslant \mathrm{inf}\left\{p\left(u-x_0\right)-\varphi\left(u\right), u\in G\right\}.$$
  Cette condition est vérifiée puisque 
  $$\forall (u,v)\in G,\ \varphi(u)+\varphi(v)=\varphi(u+v)\leqslant p(u+v)\leqslant p(u-x_0)+p(v+x_0)$$
  autrement dit 
  $$\varphi(u)-p(u-x_0)\leqslant p(v-x_0)-\varphi(v).$$
  Donc $\alpha$ existe bien, et puisque $F\subset G\subset E$, $\tilde h$ est bien un prolongement admissible.
\end{Proof}


\begin{ProofC}{Démonstration du théorème}
  On note $X$ l'ensemble des couples
  $$(G,h)$$
  où $F\subset G\subset E$ et où $h:G\longrightarrow \R$ est un prolongement admissible.

  On peut munir $X$ de la relation d'ordre (partielle) suivante 
  $$(G_1,h_1)\preccurlyeq(G_2,h_2)\Leftrightarrow\begin{cases}
    G_1\subset G_2\\
    \forall x\in G_1,\ h_1(x)=h_2(x)
  \end{cases}$$

  $\star$ Montrons que $(X,\preccurlyeq)$ est inductif, soit $(G_i,h_i)_{i\in I}$ une famille totalement ordonné d'éléments de $X$, on va montrer que $(G_i,h_i)$ possède un majorant.

  Soit 
  $$G:=\bigcup_{i\in I}G_i$$
  et $h$ vérifiant 
  $$\forall x\in G,\ \forall i\in I\tq x\in G_i,\ h(x):=h_i(x).$$
  On a bien défini $h$ car si \og conflit\fg\ de définition, pour $x\in G_i\cap G_j$, on a (car la famille est totalement ordonnée) (sans perte de généralité)
  $$(G_i,h_i)\preccurlyeq(G_j,h_j)$$
  donc $h_i=h_j\big|_{G_i}$ et $h_i(x)=h_j(x).$

  $\star\star$ $G$ est un espace vectoriel : soient $(x,y)\in G,\ \lambda\in \R$, il existe un couple $(i,j)$ tel que 
  $$x\in G_i,\ y\in G_j$$
  puisque la famille est totalement ordonnée, on a (sans perte de généralité) $G_i\subset G_j$ donc $x\in G_j$ et (car $G_j$ est un espace vectoriel) $x+\lambda y\in G_j\subset G.$

  $\star\star$ $h$ est admissible : soit $x\in G$ donc $x\in G_i$ et 
  $$|h(x)|=|h_i(x)|\leqslant p(x)$$
  donc $h$ est admissible et $(G,h)\in X$.

  On vérifie alors que $(G,h)$ est un majorant de $(G_i,h_i)_{i\in I}$, donc $(X,\preccurlyeq)$ est inductif.\\

  On applique le lemme de Zorn, donc il existe $(\tilde G,\tilde\varphi)$ maximal.\\

  $\dagger$ Si $\tilde G=E$, alors $\tilde\varphi$ convient.\\

  $\dagger$ Si $\tilde G\neq E$, alors il existe $x_0\in E\backslash g$ et en considérant le prolongement admissible du lemme, il existe $(G_1, \tilde\varphi_0)$ tel que 
  $$(\tilde G,\tilde\varphi)\preccurlyeq(G_1, \tilde\varphi_0)$$
  c'est absurde, donc $G=E$ et le théorème est prouvé.
\end{ProofC}

\begin{thC}{Hahn-Banach, cas réel}
  Soit $E$ un $\R$-espace vectoriel et $F\subset E$ un sous-espace vectoriel, soit $\varphi\in F^*$,\\

  alors il existe $\tilde\varphi\in E^*$ telle que 
  $$\tilde\varphi\big|_F=\varphi\text{ et }||\varphi||_{F^*}=||\tilde\varphi||_{E^*}$$
\end{thC}


\begin{Proof}
  On pose 
  $$\forall x\in E,\ p(x):=||\varphi||\ ||x||$$
  d'après le théorème de Hahn-Banach, il existe $\tilde\varphi:E\longrightarrow\R$ prolongeant $\varphi$ et telle que 
  $$\forall x\in E,\ \tilde\varphi(x)\leqslant ||\varphi||\ ||x||.$$
  Par linéarité, on a aussi 
  $$\forall x\in E,\ |\tilde\varphi(x)|\leqslant ||\varphi||\ ||x||$$
  donc $\tilde\varphi$ est continue et telle que $||\tilde\varphi||\leqslant ||\varphi||$
  Et comme $\tilde\varphi$ est un prolongement de $\varphi$, on a aussi $||\tilde\varphi||\geqslant ||\varphi||$ donc 
  $$\tilde\varphi\big|_F =\varphi\text{ et }||\tilde\varphi||= ||\varphi||.$$
\end{Proof}

On peut étendre ce théorème dans le cas complexe, 


\begin{thC}{Hahn-Banach, cas complexe}
  Soit $E$ un $\C$-espace vectoriel et $F\subset E$ un sous-espace vectoriel, soit $\varphi\in F^*$,\\

  alors il existe $\tilde\varphi\in E^*$ telle que 
  $$\tilde\varphi\big|_F=\varphi\text{ et }||\varphi||_{F^*}=||\tilde\varphi||_{E^*}$$
\end{thC}
  

\begin{Proof}
  $\varphi_\R:=\Re\ \varphi$ est une forme linéaire réelle continue, donc on peut appliquer le théorème de Hahn-Banach réel. Il existe donc $\tilde\varphi_\R\in E^*$ telle que 
  $$\tilde\varphi_\R\big|_F=\tilde\varphi_\R\text{ et }||\tilde\varphi_\R||=||\varphi_\R||$$
  En posant 
  $$\tilde\varphi:x\in E\longmapsto \tilde\varphi_\R(x)-i\tilde\varphi_\R(ix)$$
  on vérifie que $\tilde\varphi:E\longrightarrow\C$ est linéaire continue et 
  $$\tilde\varphi\big|_F=\varphi\text{ et }||\varphi||=||\tilde\varphi||$$
\end{Proof}


\begin{corollaire}
  Soient $E$ un $\K$-espace vectoriel et $x\in E$,\\

  alors\\
  $(\mathit 1)\ \exists x^*\in E^*\tq ||x^*||=1$ et $x^*(x)=||x||$\\[1em]
  $(\mathit 2)\ ||x||=\mathrm{sup}\left\{|x^*(x)|,\ x^*\in E^*\tq ||x^*||=1\right\}$.
\end{corollaire}


\begin{ProofC}{Démonstration du (1)}
  Soit $x\in  E$, on considère 
  $$\varphi:\begin{array}{|rcl}
    \K x&\longrightarrow&\K\\
    \lambda x&\longmapsto&\lambda||x||
  \end{array}$$
  $\varphi$ est une application linéaire isométrique (donc de norme 1) donc $\varphi$ est continue, donc d'après le théorème de Hahn-Banach, il existe $x^*\in E^*\tq ||x^*||=||\varphi||=1\text{ et }\varphi=x^*\big|_{\K x}$.

  En particulier, $x^*(x)=\varphi(x)=||x||$, ce qui conclut.
\end{ProofC}


\begin{ProofC}{Démonstration du (2)}
  Soit $x^*\in E^*$ telle que $||x^*||=1$, alors
  $$|x^*(x)|\leqslant ||x^*||\ ||x||=||x||$$
  donc 
  $$\mathrm{sup}\left\{|x^*(x)|,\ x^*\in E^*\tq ||x^*||=1\right\}\leqslant ||x||$$
  et le point $(1)$ du corollaire nous indique que $||x||$ est atteinte, d'où l'égalité annoncée.
\end{ProofC}


\begin{corollaire}
  Soit $E$ un $\K$-espace vectoriel,\\

  alors $E^*$ sépare les points, \emph{i.e.}\\
  $\forall (a,b)\in E\tq a\neq b,\ \exists f\in E^*\tq f(a)\neq f(b).$
\end{corollaire}


\begin{Proof}
  Soit $(a,b)\in E$ distincts, on pose $x:=a-b$, d'après le corolaire précédent, il existe $x^*\in E^*$ tel que $x^*(x)=||x||$, donc 
  $$x^*(a)-x^*(b)=||x||\neq0.$$
\end{Proof}


\begin{corollaire}
  Soit $E$ un $\K$-espace vectoriel, $F\subset E$ un sous-espace vectoriel fermé et $a\in E\backslash F$,\\

  alors il existe $x^*\in E^*$ tel que 
  $$x^*(a)=1\text{ et }x^*\big|_F=0.$$
\end{corollaire}


\begin{Proof}
  Soit 
  $$\varphi_a:\begin{array}{|ccc}
    F\oplus \K a&\longrightarrow&\K\\
    x+\lambda a&\longmapsto&\lambda
  \end{array}$$
  $\varphi_a$ est linéaire et de noyau $\ker\ \varphi_a=F$ fermé (par hypothèse) donc $\varphi_a$ est continue.

  D'après le théorème de Hahn-Banach, il existe $x*\in E^*$ telle que 
  $$||x^*||=||\varphi_a||\text{ et }x^*\big|_{F\oplus \K a}=\varphi_a$$
  on vérifie alors que ce $x^*$ convient, \emph{i.e.} 
  $$x^*(a)=\varphi_a(a)=1\text{ et }x^*\big|_F=0.$$
\end{Proof}

\begin{Def}
  Soit $E$ un $\K$-espace vectoriel, $F\subset E$ un sous-espace vectoriel, on définit \emph{l'orthogonal de F} par 
  $$F^\bot:=\left\{x^*\in E^*\tq x^*\big|_F=0\right\}.$$
\end{Def}


\begin{corollaire}
  Soit $E$ un $\K$-espace vectoriel, $F\subset E$ un sous-espace vectoriel fermé,\\

  alors les proposition suivantes sont équivalentes,\\
  $(i)\ \overline{F}=E$\\
  $(ii)\ F^\bot=\{0\}$.
\end{corollaire}


\begin{Proof}
  $\star$ Si $\overline F=E$, soit $x^*\in F^\bot$ donc $x^*\big|_F=0$, montrons que $x^*=0$. 
  
  Soit $x\in E$, par densité de $F$ dans $E$, il existe $x_n\in F$ convergeant vers $x$. 
  
  Par continuité, on a 
  $$x^*(x)=\underset{n\to\infty}{\mathrm{lim}}x^*(x_n)=\underset{n\to\infty}{\mathrm{lim}} 0=0,$$
  et $F^\bot=\{0\}$.\\

  $\star$ Réciproquement, si $F^\bot=\{0\}$ on va supposer que $\overline F\neq E$, donc il existe $x\in E\backslash\overline{F}$.
  
  Puisque $\overline{F}$ est un fermé, on applique le corollaire précédent et il existe $x^*\in E^*$ tel que 
  $$x^*(x)=1\text{ et }x^*\big|_{\overline{F}}=0,$$
  puisque $F\subset\overline{F}$, on a aussi $x^*\big|_{{F}}=0$ donc $x^*\in F^\bot$, ce qui est absurde car $x^*$ est non nul (en $x$), ce qui contredit l'hypothèse sur $F^\bot$
\end{Proof}


$\triangleright$\emph{ex :} soit $a\in\C$ avec $|a|>1$, on considère
$$f_a:\begin{array}{|rcl}
  [-1,1]&\longrightarrow&\C\\
  t&\longmapsto&\dstyle\frac{1}{a-t}
\end{array}$$
et on prend $(a_n)\in \C$ tel que $\forall n\in\N,\ |a_n|>1$ et $|a_n|\tend +\infty$,\\

alors $\mathrm{Vect}\left(f_{a_n},\ n\in\N\right)$ est dense dans $\left(C[-1,1],||\cdot||_\infty\right)$.


\begin{Proof}
  Soit $\varphi\in C([-1,1])$ (éventuellement nulle) telle que 
  $$\forall n\geqslant 0,\ \varphi(f_{a_n})=0.$$
  Montrons que $\varphi$ est nécessairement nulle, soit $t\in[-1,1]$, on a 
  \begin{align*}
    f_a(t)&=\frac{1}{a\left(1-\nicefrac{t}{a}\right)}\\
    &=\frac{1}{a}\sum_{k=0}^\infty\frac{t^k}{a^k}\\
    f_a(t)&=\sum_{k=0}^\infty\frac{t^k}{a^{k+1}}
  \end{align*}
  Puisque 
  $$\mathrm{sup}\left\{\frac{t^k}{a^{k+1}},\ t\in[-1,1]\right\}=\frac{1}{a^{k+1}}$$
  et que $\sum_{k\geqslant 0}\frac{1}{a^{k+1}}<\infty$,
  $$\sum_{k=0}^\infty\frac{t^k}{a^{k+1}}\text{ CVN sur }[-1,1].$$
  Soit $e_n:t\in[-1,1]\longmapsto t^n$, alors 
  $$f_a=\sum_{k=0}^\infty\frac{e_n}{a^{k+1}}$$
  donc
  $$\varphi(f_a)=\sum_{k=0}^\infty\varphi(e_k)\frac{1}{a^{k+1}}.$$
  On pose 
  $$L(z):=\sum_{k=0}^\infty\frac{t^k}z^k$$
  et on a l'inégalité suivante 
  $$\forall k\in\N,\ \varphi(e_k)\leqslant ||\varphi||\ ||e_k||_\infty\leqslant ||\varphi||$$
  donc pour $|z|<1$, $L(z)$ converge et le rayon de convergeance de cette série est plus grand que $1$.

  $f$ est holomorphe, $f$ est nulle en $\frac{1}{a_n}$, pour $n\in\N$ et $\frac{1}{a_n}\longrightarrow 0$, donc en vertu du principe des zéros isolés, $f=0$.

  En particulier, pour tout $k\in \N,\ \varphi(e_k)=0$, donc par linéarité, 
  $$\forall P\in \C[X],\ \varphi(P)=0$$
  or $C[X]$ est dense dans $C([-1,1])$ donc $\varphi$ est nulle.

  Or $\varphi\in \mathrm{Vect}\left(f_{a_n},\ n\in\N\right)^\bot$, donc $\varphi\in \mathrm{Vect}\left(f_{a_n},\ n\in\N\right)^\bot=\{0\}$ donc d'après le lemme précédent, $\mathrm{Vect}\left(f_{a_n},\ n\in\N\right)$ est effectivement dense dans $C([-1,1])$.
\end{Proof}


\setcounter{prop}{0}


\NouvellePart{Espaces de Hilbert}


\PremiereNouvelleSection{Rappels sur les espaces préhilbertiens}


\begin{Def}
  Soit $E$ un $\K$-espace vectoriel et $\langle \cdot,\cdot\rangle:E\times E\longrightarrow E$ une application,\\

  $(E,\langle \cdot,\cdot\rangle)$ est un \emph{espace préhilbertien} si, et seulement si \\
  $\forall (x,y,z)\in E,\ \forall \lambda\in\K$,\\
  $(\bullet)\quad \langle x,y\rangle =\overline{\langle y,x\rangle}$\\
  $(\bullet\bullet)\quad \langle x+y,z\rangle =\langle y,z\rangle+\langle y,z\rangle$\\
  $(\bullet\bullet\bullet)\quad \langle \lambda x,y\rangle =\lambda\langle x,y\rangle$\\
  $(\bullet\bullet\bullet\bullet)\quad \langle x,x\rangle \geqslant 0$\\
  $(\bullet\bullet\bullet\bullet\bullet)\quad \langle x,x\rangle = 0\Leftrightarrow x=0.$\\
\end{Def}


\begin{RQ}
  Si $\langle \cdot,\cdot\rangle$ vérifie uniquement les deux premiers points, c'est une \emph{forme hermitienne}, si il vérifie les quatres premiers points c'est une \emph{forme hermitienne positive} et si il vérifies tous ces points, on peut dire que c'est un \emph{produit scalaire} ou une \emph{forme hermitienne définie positive}, 
\end{RQ}


\begin{RQ}
  Pour $\ps{\cdot,\cdot}$ une forme hermitienne, on a aussi 
  $$\forall (x,y)\in H,\forall \lambda\in \K,\ \ps{x,\lambda y}=\overline{\lambda}\ps{x,y}.$$
  (c'est trivial, mais il faut garder ça en tête)
\end{RQ}


\begin{propC}{Cauchy-Schwarz}
  Soit $(E,\langle \cdot,\cdot\rangle)$ un espaces préhilbertien,\\

  alors 
  $$\forall (x,y)\in E,\ |\langle x,y\rangle|\leqslant ||x||\cdot ||y||=\sqrt{\langle x,x\rangle}\sqrt{\langle y,y\rangle},$$
  une autre version est 
  $$\forall (x,y)\in E,\ \langle x,y\rangle^2\leqslant \ps{x,x}\cdot\ps{y,y}.$$
  De plus, en posant 
  $$||\cdot||:x\longmapsto \sqrt{\langle x,x\rangle}$$
  on obtient un espace vectoriel normé $(E,||\cdot||)$.
\end{propC}


$\triangleright$\emph{ex :} soit $A=\left(a_{i,j}\right)\in M_n(\C)$ une matrice hermitienne, alors 
$$f_A:\begin{array}{ccl}
  \C^n\times\C^n&\longrightarrow&\C\\
  (u,v)&\longmapsto&\langle Au,v\rangle
\end{array}$$
est une forme hermitienne et 
$$f_a\text{ positive }\Leftrightarrow\mathrm{Sp}\ A\subset\R_+,$$
$$f_a\text{ définie positive }\Leftrightarrow\mathrm{Sp}\ A\subset\R_+^*,$$


$\triangleright$\emph{ex :} si 
$$\ell^2(\N):=\left\{u:\N\longrightarrow\C\tq \sum_{n\geqslant 0}|u_n|^2<\infty\right\}$$
alors 
$$\langle u,v\rangle:=\sum_{n\geqslant 0}u_n\overline{v_n}$$
est un produit scalaire, dit \emph{canonique} sur $\ell^2(\N).$


\begin{propC}{identité du parallélogramme}
  Pour $H$ un espace préhilbertien et $(x,y)\in H$, on a 
  $$||x+y||^2+||x-y||^2=2\left(||x||^2+||y||^2\right).$$
\end{propC}


\begin{Proof}
  Soient $(x,y)\in H$,
  \begin{align*}
    ||x+y||^2&=\langle x+y,x+y\rangle\\
    &=||x||^2+||y||^2+\langle x,y\rangle+\langle y,x\rangle\\
    &=||x||^2+||y||^2+\langle x,y\rangle+\overline{\langle y,x\rangle}\\
    ||x+y||^2&=||x||^2+||y||^2+\Re \langle x,y\rangle\qquad (i)\\
  \end{align*}
  et la même manière, on a 
  $$||x-y||^2=||x||^2+||y||^2-\Re \langle x,y\rangle\qquad (ii)$$
  on obtient donc, en sommant les points $(i)$ et $(ii)$
  $$||x+y||^2+||x-y||^2=2\left(||x||^2+||y||^2\right).$$
\end{Proof}


\begin{propC}{identité de polarisation}
  Pour $H$ un espace préhilbertien et $(x,y)\in H$, on a 
  $$\text{ si }\K=\R,\ \langle x,y\rangle=\frac14\left(||x+y||^2-||x-y||^2\right),$$
  $$\text{ si }\K=\C,\ \langle x,y\rangle=\frac14\left(\sum_{k=0}^3i^k||x+i^ky||\right).$$
\end{propC}


\begin{Proof}
  Par le calcul.
\end{Proof}


\begin{prop}
  Soient $H$ un espace préhilbertien, $(x_n)\in H$ telle que $x_n\longrightarrow x$ et $y_n\in H$ telle que $y_n\longrightarrow y$,\\

  alors
  $$\langle x_n,y_n\rangle\tend \langle x,y\rangle$$
\end{prop}


\begin{Proof}
  Soit $n\in\N$, on calcule :
  \begin{align*}
    \left|\langle x_n,y_n\rangle-\ps{x,y}\right|&=\abs{\ps{x_n-x,y_n}+\ps{x,y_n-y}}\\
    &\leqslant \abs{\ps{x_n-x,y_n}}+\abs{\ps{x,y_n-y}}\\
    \left|\langle x_n,y_n\rangle-\ps{x,y}\right|&\leqslant \norm{x_n-x}\ \norm{y_n}+\norm{x}\ \norm{y_n-y}
  \end{align*}
  et $\norm{x_n-x}\longrightarrow 0$, $\norm{y_n-y}\longrightarrow 0$ donc 
  $$\left|\langle x_n,y_n\rangle-\ps{x,y}\right|\tend 0.$$
\end{Proof}


\begin{Def}
  Soit $(H,\ps{\cdot,\cdot})$ un préhilbertien,\\
  $\bullet\ (x,y)\in H$ sont dits \emph{orthogonaux} \ssi 
  $$\ps{x,y}=0.$$
  $\bullet$ pour $A\subset H$, l'\emph{orthogonal de }$A$, noté $A^\bot$, est défini tel que 
  $$A^\bot:=\ens{x\in H\tq \forall y\in A,\ \ps{x,y}=0}.$$
  $\bullet$ $A\subset H$ et $B\subset H$ sont \emph{orthogonaux}, noté $A\perp B$, \ssi
  $$\forall x\in A,\ \forall b\in B,\ \ps{a,b}=0.$$
\end{Def}


\begin{propC}{caractérisation de l'orthogonal}
  Soit $H$ préhilbertien, pour $(x,y)\in H$,\\

  alors 
  $$x\perp y\Leftrightarrow \norm{x+y}^2=||x||^2+||y||^2.$$
\end{propC}


\begin{Proof}
  Par le calcul.
\end{Proof}


on a aussi les propriétés générales suivantes, que l'on ne démontrera pas (par flemme + déjà vu en prépa)


\begin{propC}{propriétés générales}
  Soit $H$ préhilbertien et $(A,B,C)\subset H$,\\

  $(\mathit 1)\ A\subset B\Rightarrow A^\bot\supset B^\bot$\\
  $(\mathit 2)\ \overline{A}=H\Rightarrow A^\bot=\{0\}$\\
  $(\mathit 3)$ est un sous-espace vectoriel fermé de $H$, même si $A$ n'est pas un sous-espace vectoriel.
\end{propC}



\NouvelleSection{Système orthonormé}


Dans ce chapitre, $(H,\ps{\cdot,\cdot})$ est un espace préhilbertien et $(x_i)_{i\in\Z}$ est un ensemble de vecteurs de $H$.



\begin{Def}{\color{white} blabla}\\
  $\bullet$ Le système $(x_i)$ est un \emph{système orthogonal} \ssi 
  $$\forall (i,j)\in\Z,\ \ps{x_i,x_j}=0,$$
  $\bullet$ le système $(x_i)$ est un \emph{système orthonormé} \ssi il est orthogonal et 
  $$\forall i\in\Z,\ ||x_i||=1.$$
\end{Def}


$\triangleright$\emph{ex : }sur $\ell^1(\N)$, les 
$$e_i:=(\underbrace{0,\cdots,0}_{i-1},1,0,\cdots)$$
forment un système orthonormé.


$\triangleright$\emph{ex bis: }on pose 
$$L^2([-\pi,\pi]):=\ens{f:[-\pi,\pi]\longrightarrow\C\text{ mesurable }\tq\dstyle\int_{-\pi}^\pi |f(t)|^2\dt}$$
muni du produit scalaire 
$$\ps{f,g}:=\frac{1}{\pi}\int_{-\pi}^\pi f(t)\overline{g(t)}\dt$$
alors les 
$$e_n:t\longmapsto \ex{int}$$
forme un système orthonormé.



\begin{prop}
  Soit $(H,\ps{\cdot,\cdot})$ préhilbertien,\\
  $(\mathit 1)$ pour $(x_1,\cdots,x_n)$ un système orthogonal, on a 
  $$\sum_{i=1}^n||x_i||^2=\norm{\sum_{i=1}^n x_i}^2$$
  $(\mathit 2)$ pour $(e_i)_{i\geqslant 0}$ un système orthonormé, on a 
  $$\forall x\in H,\ \sum_{n=0}^\infty\abs{\ps{e_n,x}}\leqslant ||x||^2.$$
\end{prop}


\begin{ProofC}{Démonstration du $(\mathit 1)$}
  Par récurrence sur $n$, à l'aide de l'égalité de Pythagore.
\end{ProofC}


\begin{ProofC}{Démonstration du $(\mathit 2)$}
  Soit $N\in\N$, on pose 
  $$u:=\sum_{n=0}^N\ps{x,e_n}e_n$$
  et $v:=x-u.$ On calcule
  \begin{align*}
    \ps{u,v}&=\ps{\sum_{n=0}^N\ps{e_n,x}e_n,x-\sum_{n=0}^N\ps{x,e_n}e_n}\\
    &=\ps{\sum_{n=0}^N\ps{e_n,x}e_n,x}-\ps{\sum_{n=0}^N\ps{e_n,x}e_n,\sum_{n=0}^N\ps{x,e_n}e_n}\\
    &=\sum_{n=0}^N\ps{x,e_n}\ps{e_n,x}-\norm{\sum_{n=0}^N\ps{x,e_n}e_n}^2\\
    \ps{u,v}&=\sum_{n=0}^N\abs{\ps{x,e_n}}^2-\norm{\sum_{n=0}^N\ps{x,e_n}e_n}^2
  \end{align*}
  puisque la famille $(e_n)$ est orthonormée, l'égalité de Pythagore s'applique et 
  $$\ps{u,v}=\sum_{n=0}^N\abs{\ps{x,e_n}}^2-\sum_{n=0}^N\abs{\ps{x,e_n}}^2=0.$$
  Par ailleurs, on a aussi
  \begin{align*}
    ||x^2||=||u+v||^2&=||u||^2+||v||^2+2\Re\ps{u,v}\\
    &=||u||^2+||v||\\
    ||x||^2&\geqslant ||u||^2
  \end{align*}
  ainsi 
  $$||x||^2\geqslant \sum_{n=0}^N\abs{\ps{x,e_n}}^2\tend \sum_{n=0}^\infty\abs{\ps{e_n,x}}$$
\end{ProofC}


\NouvelleSection{Espace de Hilbert}


\begin{Def}
  Un \emph{espace de Hilbert} est un espace préhilertien complet.
\end{Def}


\begin{RQ}
  En dimension finie, tout espace préhilbertien est un espace de Hilbert
\end{RQ}


$\triangleright$\emph{ex : }$\ell^2(\N)$ muni du produit scalaire habituel est un espace de Hilbert.\\


$\triangleright$\emph{ex : }pour $(\O,\A,\mu)$ un espace mesuré, $L^2(\O,\A,\mu)$ est un espace de Hilbert.\\


$\triangleright$\emph{contre-ex : }$C([0,1])$ muni du produit scalaire 
$$\ps{f,g}:=\int_0^1f(t)\overline{g(t)}\dt$$
n'est pas un espace de Hilbert.


\begin{thC}{de projection sur un convexe fermé}
  Soit $H$ de Hilbert et $K$ (non vide) un convexe fermé de $H$,\\

  alors pour tout $x\in H$, il existe un unique $u\in K$ tel que 
  $$||x-u||=d(x,K)=\mathrm{inf}\ens{||x-z||,\ z\in K}$$
  l'élement en question est le \emph{projeté de }$x$\emph{ sur }$K$, on le note $P_K(x).$
\end{thC}


une formulation équivalente de ce théorème est 


\begin{thC}{formulation variationnelle}
  Soit $H$ de Hilbert et $K$ (non vide) un convexe fermé de $H$,\\

  les assertions suivantes sont équivalentes\\
  $(\mathit 1)\ u=P_K(x)$\\
  $(\mathit 2)\ u\in K\text{ et }\forall v\in K,\ \Re\ps{x-u,v-u}\leqslant 0$
\end{thC}


\begin{ProofC}{Démonstration du théorème de projection}
  Soit $H$ un espace de Hilbert, $K\subset H$ un convexe fermé et non vide, soit $x\in H$,\\

  $\star$ Existence : soit $d:=\mathrm{dist}(x,K)$, par caractérisation de l'inférieur d'une partie, il existe $u_n\in K$ telle que 
  $$||u_n-x||\tend d.$$
  Montrons que $(u_n)$ est de Cauchy, soit $a_n:=x-u_n$, on calcule
  $$||a_n+a_m||^2+||a_n-a_m||^2=2\left(||a_n||^2+||a_m||^2\right)$$
  d'après l'égalité du parallélogramme, donc 
  $$||2x-(u_n+u_m)||^2+||u_n-u_m||^2=2\left(||x-u_n||^2+||x-u_m||^2\right)$$
  et 
  $$||u_n-u_m||^2=2\left(||x-u_n||^2+||x-u_m||^2\right)-4\norm{x-\frac{u_n+u_m}{2}}^2.\qquad (i)$$
  $K$ étant convexe, $\displaystyle\frac{u_n+u_m}{2}\in K$ donc 
  $$||x-\frac{u_n+u_m}{2}||^2\geqslant d^2$$
  et l'égalité $(i)$ devient 
  $$||u_n-u_m||^2\leqslant2\left(||x-u_n||^2+||x-u_m||^2\right)-4d^2$$
  puisque $||x-u_n||^2\longrightarrow d^2$, on a 
  $$||u_n-u_m||\underset{n,m\to\infty}{\longrightarrow}0.$$
  $(u_n)$ est donc de Cauchy, donc elle converge vers $u\in E$ car $E$ est complet, et puisque $K$ est fermé, $u_n\longrightarrow u\in K$. De plus, 
  $$d=\underset{n\to\infty}{\mathrm{lim}}||x-u_n||=||x-u||$$
  et la distance est donc atteinte.\\

  $\star$ Unicité : on suppose qu'il existe aussi $u\in K$ telle que $d=||x-u||$, alors d'après l'égalité du parallélogramme, on a 
  $$||(x-u)+(x-v)||^2+||(x-u)-(x-v)||^2=2\left(||x-u||^2+||x-v||^2\right)$$
  donc 
  $$||2x-(u+v)||^2+||v-u||^2=4d^2.$$
  Puisque $K$ est convexe, on a la majoration suivante ;
  $$||u-v||^2\leqslant 4d^2-\underbrace{4\norm{x-\frac{u+v}{2}}^2}_{\leqslant 4d^2}\leqslant 0$$
  donc $u=v$ et le projeté est effectivement unique.
\end{ProofC}


\begin{ProofC}{Démonstration de la formulation variationnelle}
  $\star$ Si $u=P_K(x)$, alors pour $t\in [0,1],\ v\in K$, on pose 
  $$w(t):=(1-t)u+tv$$
  par convexité de $K$, $w\in K$ et on a 
  \begin{align*}
    ||x-u||^2&\leqslant ||x-w||^2=||x-(1-t)u-tw||^2\\
    &\leqslant ||(x-u)-t(v-u)||^2\\
    ||x-u||^2&\leqslant ||x-u||^2+t^2||u-v||-2\Re\ps{x-u,v-u}
  \end{align*}
  donc 
  $$2\Re \ps{x-u,v-u}\geqslant t||u-v||$$
  en faisant tendre $t$ vers $0^+$, on obtient bien 
  $$2\Re \ps{x-u,v-u}\geqslant 0.$$

  $\star$ Réciproquement, si $u\in K$ et $(\mathit 2)$ est vérifiée, alors 
  \begin{align*}
    ||x-v||^2&=||(x-u)+(u-v)||^2\\
    &=||x-u||^2+||u-v||^2+2\Re\ps{x-u,u-v}\\
    ||x-v||^2&=||x-u||^2+||u-v||^2-2\Re\ps{x-u,v-u}\\
  \end{align*}
  donc, puisque $2\Re\ps{x-u,v-u}\leqslant 0$, 
  $$||x-v||\geqslant ||x-u||$$
  et 
  $$||x-u||\leqslant\mathrm{inf}\ens{||x-v||, v\in K}=P_K(x).$$
\end{ProofC}


$\triangleright$\emph{ex : }Soit $K=\overline{B(0,1)}$, montrons que 
$$\forall ||x||\geqslant 1,\ P_K(x)=\frac{x}{||x||}.$$
Soit $||x||\geqslant 1$, on a utiliser la caractérisation variationnelle :\\

$\star$ il est clair que $\displaystyle\frac{x}{||x||}\in K$\\

$\star$ soit $v\in K$, on a 
\begin{align*}
  \ps{x-\dstyle\frac{x}{||x||},v-\dstyle\frac{x}{||x||}}&=\ps{x,v}-||x||-\frac{1}{||x||}\ps{x,v}+1\\
  &=1-||x||+\ps{x,v}\left(1-\frac{1}{||x||}\right)\\
  &=1-||x||+\ps{\dstyle\frac{x}{||x||},v}\left(||x||-1\right)\\
  \ps{x-\dstyle\frac{x}{||x||},v-\dstyle\frac{x}{||x||}}&=\left(1-||x||\right)\left(1-\ps{\dstyle\frac{x}{||x||},v}\right)
\end{align*}
on passe à la partie réelle :
$$\Re\ps{x-\dstyle\frac{x}{||x||},v-\dstyle\frac{x}{||x||}}=\underbrace{\left(1-||x||\right)}_{\leqslant 0}\Bigg(1-\underbrace{\Re\ps{\dstyle\frac{x}{||x||},v}}_{\leqslant 1}\Bigg)\leqslant 0$$
donc pour $||x||\geqslant 1$, on a bien $P_K(x)=\dstyle\frac{x}{||x||}$ et pour $x\in K,\ P_K(x)=x.$
\begin{prop}
  Soit $H$ de Hilbert, $K$ un convexe non-vide et fermé de $E$,\\

  alors 
  $$\forall (x,y)\in H,\ \norm{P_K(x)-P_K(y)}\leqslant\norm{x-y}$$
  donc $P_K$ est une application continue.
\end{prop}


\begin{Proof}
  Soient $(x_1,x_2)\in H$, on note $u_1:=P_K(x_1)$ et $u_2:=P_K(x_2)$, donc par caractérisation variationnelle
  $$\forall v\in K,\ \begin{cases}
    \Re\ps{x_1-u_1,v-u_1}\leqslant 0\\
    \Re\ps{x_2-u_2,v-u_2}\leqslant 0\\
  \end{cases}$$
  on évalue ce système en $v=u_2$ dans la première équation et $v=u_1$ dans la deuxième équation,
  $$\begin{cases}
    \Re\ps{x_1-u_1,u_2-u_1}\leqslant 0\\
    \Re\ps{x_2-u_2,u_1-u_2}\leqslant 0\\
  \end{cases}$$
  par antisymétrie sur la deuxième ligne (deux fois), on a
  $$\begin{cases}
    \Re\ps{x_1-u_1,u_2-u_1}\leqslant 0\\
    \Re\ps{u_2-x_2,u_2-u_1}\leqslant 0\\
  \end{cases}$$
  en sommant les deux lignes on trouve 
  $$\Re\ps{(x_1-u_1)+(u_2-x_2),u_2-u_1}\leqslant 0$$
  autrement dit
  $$\Re\ps{(x_1-x_2)+(u_2-u_1),u_2-u_1}\leqslant 0.$$
  et 
  $$\Re\ps{x_1-x_2,u_2-u_1}+||u_2-u_1||^2\leqslant 0.$$
  On peut ensuite majorer comme suit
  \begin{align*}
    ||u_1-u_2||^2&=-\Re\ps{x_1-x_2,u_2-u_1}\\
    &\leqslant \abs{\ps{x_1-x_2,u_2-u_1}}\\
    ||u_1-u_2||^2&\leqslant ||x_1-x_2||\ ||u_1-u_2||
  \end{align*}
  puisque $u_1\neq u_2$, on peut simplifier l'inégalité et 
  $$\norm{P_K(x)-P_K(y)}=\norm{u_1-u_2}\leqslant\norm{x-y}$$
\end{Proof}



\begin{propC}{projection sur un s.e.v. fermé}
  Soit $H$ de Hilbert et $M\subset H$ un sous-espace vectoriel fermé de $E$, pour $x\in H$ et $u\in H$,\\

  alors les assertions suivantes sont équivalentes :\\
  $(i)\ u=P_M(x)$\\
  $(ii)\ u\in M\text{ et }\forall v\in M,\ \ps{x-u,v}=0.$
\end{propC}


\begin{RQ}
  $M$ étant un sous-espace vectoriel, $M$ est convexe donc la définition de $P_M$ ne pose aucun problème.
\end{RQ}

\begin{Proof}
  $\star$ Si $u=P_M(x)$, alors d'après la caractérisation variationnelle, on a $u\in M$ et 
  $$\forall m\in M,\ \Re\ps{x-u,m-u}\leqslant 0.\qquad (i)$$
  Puisque $M$ est un espace vectoriel, $m-u\in M$ et on peut reformuler $(i)$ en 
  $$\forall v\in M,\ \Re\ps{x-u,v}\leqslant 0.$$
  Par linéarité du produit scalaire (car $v\in M\Rightarrow -v\in M$), on a aussi 
  $$\Re\ps{x-u,-v}\leqslant 0\Leftrightarrow-\Re\ps{x-u,v}\leqslant 0\Leftrightarrow \Re\ps{x-u,v}=0.$$
  ainsi on a 
  $$\forall v\in M,\ \Re\ps{x-u,v}=0.$$

  $\dagger$ Si $H$ est un $\R$-espace vectoriel, c'est fini.

  $\dagger$ Si $H$ est un $\C$-espace vectoriel, alors en considérant $v\to iv$, on a 
  $$\Re\ps{x-u,iv}=0\Leftrightarrow -\Im\ps{x-u,v}=0$$
  donc $\ps{x-u,v}=0$ et c'est fini.\\

  $\star$ Réciproquement, si $(ii)$ est vérifiée, puisque $v-u\in M$, on a 
  $$\forall v\in M,\ \ps{x-u,v-u}=0$$ 
  donc à fortiori
  $$\forall v\in M,\ \Re\ps{x-u,v-u}\leqslant0$$
  puisque $u\in M$, la caractérisation variationnelle est vérifié et $u=P_M(x).$
\end{Proof}

\begin{corollaire}
  Soit $H$ de Hilbert et $M\subset H$ un sous-espace vectoriel fermé de $E$, pour $x\in H$ et $u\in H$,\\
  
  $(\mathit 1)\ P_M$ est linéaire continue et de norme $1$,\\
  $(\mathit 2)\ P_M^2=P_M$,\\
  $(\mathit 3) \ \forall (x,y)\in H, \ps{P_M(x),P_M(y)}=\ps{x,P_M(y)}=\ps{P_M(x),y}.$
\end{corollaire}


\begin{ProofC}{Démonstration du $(\mathit 1)$}
  Soient $(x,y)\in H$, $\lambda\in\K$, on a 
  $$\forall v\in M,\ \ps{x-P_M(x),v}=0\text{ et }\ps{\lambda y-\lambda P_M(y),v}=0$$
  par linéarité par rapport à la première variable, on a 
  $$\forall v\in M,\ \ps{(x+\lambda y)-(P_M(x)+\lambda P_M(y)),v}=0$$
  donc 
  $$P_M(x+\lambda y)=P_M(x)+\lambda P_M(y)$$
  et $P_M$ est linéaire.
  $P_M$ est $1$-lipschitzienne et, pour $x\neq 0$ et $x\in M$, $P_M(x)=x$ donc $||P_M||=1.$
\end{ProofC}


\begin{ProofC}{Démonstration du $(\mathit 2)$}
    Clair
\end{ProofC}


\begin{ProofC}{Démonstration du $(\mathit 3)$}
  Puisque $P_M(y)\in M$, on a 
  $$\ps{x-P_M(x),P_M(y)}=0$$
  donc (linéarité par rapport à la première variable)
  $$\ps{x,P_M(y)}=\ps{P_M(x),P_M(y)}.$$
  En échangeant $x$ et $y$, on trouve
  $$\ps{y,P_M(x)}=\ps{P_M(y),P_M(x)}$$
  on conjugue cette égalité
  $$\overline{\ps{y,P_M(x)}}=\overline{\ps{P_M(y),P_M(x)}}$$
  et on trouve bien 
  $$\ps{P_M(x),y}=\ps{P_M(x),P_M(y)}$$
  d'où l'égalité recherchée 
  $$\ps{P_M(x),P_M(y)}=\ps{x,P_M(y)}=\ps{P_M(x),y}.$$
\end{ProofC}


\begin{corollaire}
  Soit $H$ de Hilbert et $M\subset H$ un sous-espace vectoriel fermé de $E$, pour $x\in H$ et $u\in H$,\\
  
  $(\mathit 1)\ H=M\oplus M^\bot$\\
  $(\mathit 2)\ \forall x\in H,\ x=P_M(x)+P_{M^\bot}(x)$\\
\end{corollaire}


\begin{RQ}
  On a l'expression de $P_{M^\bot}$ en fonction de $P_M$ suivante :
  $$\forall x\in H,\ P_{M^\bot}(x)=x-P_M(x).$$
\end{RQ}

\begin{RQ}
  La décomposition de $x$ est orthogonale, c'est à dire que l'on a 
  $$||x||^2=||P_M(x)||^2+||P_{M^\bot}(x)||^2=\norm{P_M(x)}^2+\norm{x-P_M(x)}^2.$$
\end{RQ}


\begin{ProofC}{Démonstration du $(\mathit 1)$}
  Pour $x\in M$, on a 
  $$\forall v\in M, \ps{x-P_M(x),v}=0$$
  donc $x-P_M(x)\in M^\bot$ et en écrivant 
  $$x=\underbrace{x-P_M(x)}_{\in M^\bot}+\underbrace{P_M(x)}_{\in M}$$
  on obtient que $H=M+M^\bot.$ Cette somme est directe car pour $x\in M\cap M^\bot$ on a 
  $$\begin{array}{c}
    x\in M^\bot\text{ donc }\forall v\in M,\ \ps{x,v}=0\\
    x\in M\text{ donc }\ps{x,x}=0\Leftrightarrow x=0.
  \end{array}$$
  d'où $H=M\oplus M^\bot$.
\end{ProofC}


\begin{ProofC}{Démonstration du $(\mathit 2)$}
  Montrons que $P_{M^\bot}(x)=x-P_M(x)$, on calcule
  $$\forall v\in M^\bot,\ \ps{x-(x-P_M(x)),v}=\ps{P_M(x),v}=0,$$
  car $P_M(x)\in M$, d'où $P_{M^\bot}(x)=x-P_M(x)$.
\end{ProofC}



\begin{prop}
  Soit $H$ un espace préhilbertien et $F$ un sous-espace vectoriel fermé de $H$, alors\\
  
  $$F=F^{\bot\bot}$$
\end{prop}


\begin{Proof}
  $\star$ Soit $x\in F$, alors pour $w\in F^\bot$, $\ps{x,w}=0$ donc $x\in F^{\bot\bot}$ et $F\subset F^{\bot\bot}$.\\

  $\star$ Réciproquement, soit $x\in F^{\bot\bot}$, alors on va écrire $x$ sous la forme 
  $$x=x_F+x_{F^\bot}$$
  et on a 
  \begin{align*}
    ||x_{F^\bot}||^2&=\ps{x_{F^\bot},x_{F^\bot}}\\
    &=\ps{x-x_F,x_{F^\bot}}\\
    ||x_{F^\bot}||^2&=0
  \end{align*}
  donc $x_{F^\bot}=0$ et $x\in F$.
\end{Proof}


\begin{prop}
  Soit $F$ un sous-espace vectoriel de $H$, \\

  alors\\
  $(\mathit 1)\ F^{\bot\bot}=\overline{F}$\\
  $(\mathit 2)\ F$ est dense dans $H$ \ssi $F^\bot=\{0\}.$
\end{prop}


\begin{ProofC}{Démonstration du $(\mathit 1)$}
  $\star$ D'une part, on a $F\subset \overline{F}$, donc $\overline{F}^\bot\subset F^\bot$ et, en composant à nouveau par $^\bot$, $F^{\bot\bot}\subset \overline{F}^{\bot\bot}$. $\overline{F}$ étant fermé, on peut appliquer la propriété précédente et $\overline{F}^{\bot\bot}=\overline{F}$ d'où $F^{\bot\bot}\subset\overline{F}$.\\

  $\star$ Réciproquement, $F\subset F^{\bot\bot}$ et puisque $F^{\bot\bot}$ est un fermé, on peut considérer la fermeture de $F$ et $\overline{F}\subset F^{\bot\bot}$. On a donc l'inclusion réciproque, et $\overline{F}= F^{\bot\bot}$.
\end{ProofC}



\begin{ProofC}{Démonstration du $(\mathit 2)$}
  $F$ est dense \ssi $\overline{F}=H$ donc $F$ est dense \ssi $\overline{F}^\bot=H^\bot=\{0\}.$
\end{ProofC}


\NouvelleSection{Base de Hilbert}



\begin{Def}
  Soit $(e_n)_{n\geqslant 0}\in H$, on dit que $(e_n)$ est \emph{complète} \ssi,\\
  $$\overline{\mathrm{Vect}\left\{e_n,\ n\geqslant 0\right\}}=H,$$
  où $\mathrm{Vect}\left\{e_n,\ n\geqslant 0\right\}$ désigne l'ensemble des combinaisons linéaires finies des vecteurs de $(e_n)$.
\end{Def}



\begin{RQ}
  Si $H$ admet une suite complète, alors $H$ est nécessairement séparable
\end{RQ}


\begin{Proof}
  Pourquoi ? à investiguer.
\end{Proof}


\begin{prop}
  Soit $H$ un espace de Hilbert séparable et $(e_n)\in H$,\\

  alors \lasse\\
  $(\mathit 1)\ (e_n)$ est complète\\
  $(\mathit 2)\forall x\in H\tq \big[\forall n\in\N,\ \ps{x,e_n}=0\big],\ x=0.$
\end{prop}


\begin{Proof}
  $\star$ Si $(e_n)$ est complète, soit $x\in H$ tel que $\forall n\in \N,\ \ps{x,e_n}=0$, par linéarité par rapport à la seconde variable du produit scalaire, on a 
  $$x\perp\mathrm{Vect}\left\{e_n,\ n\geqslant 0\right\}$$
  puisque $w\longmapsto\ps{x,w}$ est continue, on a 
  $$x\perp\overline{\mathrm{Vect}\left\{e_n,\ n\geqslant 0\right\}}=H$$
  donc $x\in H^\bot=\{0\}$ donc $x=0$.\\

  $\star$ Réciproquement, si la famille vérifie $(ii)$, alors 
  $$\mathrm{Vect}\left\{e_n,\ n\geqslant 0\right\}^\bot=\{0\}$$
  et d'après la dernière propriété de la sous-partie précédente, $\mathrm{Vect}\left\{e_n,\ n\geqslant 0\right\}$ est dense dans $H$, donc $(e_n)$ est complète.
\end{Proof}


\begin{Def}
  Soit $H$ un espace de Hilbert séparable et $(e_n)_{n\geqslant 0}\in H$ une suite, $(e_n)$ est une \emph{base hilbertienne} \ssi\\
  $\Bullet\ (e_n)$ est orthonormale\\
  $\BBullet\ (e_n)$ est une suite complète.
\end{Def}


$\triangleright$\emph{ex : }si $H=\ell^2(\N)$ et $e_i=(\underbrace{0,\cdots,0}_{i-1},1,0,\cdots)$, alors $(e_i)_{i\geqslant 0}$ est une famille orthonormale et, pour $u\in\ell^2(\N)$, on a 
$$\forall i\in \N,\ \ps{u,e_i}=u_i$$
donc si $u$ vérifie 
$$\forall i\in \N,\ \ps{u,e_i}=0$$
alors $u=0$ donc $(e_i)$ est une base hilbertienne.


\begin{Th}
  Soit $H$ de Hilbert séparable, $(e_i)$ une suite orthonormale,\\

  alors \lasse\\
  $(i)\ (e_i)$ est une base hilbertienne\\
  $(ii)\ \forall x\in H,$ 
  $$x=\sum_{n=0}^\infty\ps{x,e_n}e_n$$
  (et cette série est convergente)\\
  $(iii)\ \forall (x,y)\in H,$
  $$\ps{x,y}=\sum_{n=0}^\infty\ps{x,e_n}\overline{\ps{y,e_n}}$$
  $(iv)\ \forall x\in H,$ 
  $$||x||^2=\sum_{n=0}^\infty\abs{\ps{x,e_n}}^2$$
\end{Th}


\begin{RQ}
  L'égalité $(iv)$ est appellée \emph{égalité de Parseval}.
\end{RQ}


\begin{Proof}
  $\star\ (i)\Rightarrow(ii)$ Soit $S_N:=\sum_{n=0}^N\ps{x,e_n}e_n$, montrons que la suite des sommes partielles est de Cauchy, soient $N\geqslant M$,
  $$||S_N-S_M||^2=\norm{\sum_{n=M+1}^N\ps{x,e_n}e_n}^2=\sum_{n=M+1}^N|\ps{x,e_n}\underset{M,N\to\infty}{\longrightarrow}0$$
  car la série est convergente, donc $S_N$ est de Cauchy dans un espace complet, ainsi $S_N$ converge vers $\dstyle S=\displaystyle\sum_{n=0}^\infty\ps{x,e_n}e_n$.

  Soit $j\geqslant 0$, on calcule
  \begin{align*}
    \ps{x-S,e_j}&=\ps{x-\dstyle\sum_{n=0}^\infty\ps{x,e_n}e_n,e_j}\\
    &=\ps{x,e_j}-\sum_{n=0}^\infty\ps{x,e_n}\ps{e_n,e_j}\\
    &=\ps{e,e_j}-\ps{x,e_j}\\
    \ps{x-S,e_j}&=0
  \end{align*}
  donc $x-S\perp e_j,\ \forall j\in\N$ donc par complétude de la suite $(e_n)$, $x-S=0$ et 
  $$x=\displaystyle\sum_{n=0}^\infty\ps{x,e_n}e_n.$$

  $\star\ (ii)\Rightarrow (iii)$ Soient $(x,y)\in H$, d'après le point $(ii)$, on a alors 
  $$x=\displaystyle\sum_{i=0}^\infty\ps{x,e_i}e_i\text{ et }y=\displaystyle\sum_{n=0}^\infty\ps{y,e_n}e_n$$
  ainsi on calcule
  \begin{align*}
    \ps{x,y}&=\ps{\displaystyle\sum_{i=0}^\infty\ps{x,e_i}e_i,\displaystyle\sum_{j=0}^\infty\ps{y,e_j}e_j}\\
    &=\dstyle\sum_{i=0}^\infty\sum_{j=0}^\infty\ps{x,e_i}\overline{\ps{y,e_j}}\ps{e_i,e_j}\\
    \ps{x,y}&=\dstyle\sum_{n=0}^\infty\ps{x,e_n}\overline{\ps{y,e_n}}.
  \end{align*}

  $\star\ (iii)\Rightarrow(iv)$ On applique $(iii)$ avec $y=x$.\\

  $\star\ (iv)\Rightarrow(i)$ Soit $x\in H$ tel que $\forall n\in\N,\ \ps{x,e_n}=0$, alors 
  $$||x||^2=\sum_{n=0}^\infty\abs{\ps{x,e_n}}^2=0$$
  donc $x=0$ et la suite $(e_n)$ est complète.
\end{Proof}


\begin{Th}
  Soit $H$ de Hilbert séparable,\\
  
  alors $H$ admet une base de Hibert.
\end{Th}


\begin{Proof}
  $\dagger$ Si $\dim H<\infty$, c'est immédiat.\\

  $\dagger$ On suppose donc $\dim H=\infty$. $H$ est séparable donc il existe une suite $\ens{g_n}_{n\geqslant 0}$ dense dans $H$, on va supposer que les $(_n)$ sont tous non-nuls. 

  $\star$ On va construire une sous-suite $(g_{\varphi(n)})_{n\geqslant 0}$ telle que\\
  $\Bullet\ \forall n\geqslant 0,\ \ens{g_{\varphi(k)},\ 0\leqslant k\leqslant n}$ est une famille libre\\
  $\BBullet\ \forall n\geqslant 0,$
  $$\mathrm{Vect}\ens{g_j,\ 0\leqslant j\leqslant \varphi(n)}=\mathrm{Vect}\ens{g_{\varphi(j)},\ 0\leqslant j\leqslant n}.$$

  $\star\star$ Pour $n=0$, on pose $\varphi(0)=0$ et c'est bon.

  $\star\star$ On suppose que $\varphi(0),\cdots,\varphi(n)$ sont construits;

  SI 
  $$\forall k\geqslant \varphi(n),\ g_k\in \mathrm{Vect}\ens{g_j,\ 0\leqslant j\leqslant \varphi(n)}$$ 
  alors
  $$\mathrm{Vect}\ens{g_k,\ k\geqslant 0}=\mathrm{Vect}\ens{g_{\varphi(j)},\ 0\leqslant j\leqslant \varphi(n)}=\mathrm{Vect}\ens{g_j,\ 0\leqslant j\leqslant n}=:F$$
  $F$ est un sous-espace vectoriel de dimension $n+1$, donc $F$ est fermé et $H=F$, c'est absurde. Ainsi il existe $k_0>\varphi(n)$ tel que 
  $$g_{k_0}\notin \mathrm{Vect}\ens{g_j,\ 0\leqslant j\leqslant \varphi(n)}.$$
  On va donc pouvoir poser 
  $$\varphi(n+1):=\mathrm{min}\Big\{k\tq\varphi(n)<k\leqslant k_0\text{ et }g_{k}\notin \mathrm{Vect}\ens{g_j,\ 0\leqslant j\leqslant \varphi(n)}\Big\}$$
  donc
  $$g_{\varphi(n+1)}\notin \mathrm{Vect}\ens{g_j\ 0\leqslant j\leqslant \varphi(n)}=\mathrm{Vect}\ens{g_{\varphi(k)},\ 0\leqslant k\leqslant n}$$
  donc la famille 
  $$\ens{g_{\varphi(k)},\ 0\leqslant k\leqslant n+1}$$
  est libre.\\

  De plus, 
  $$\mathrm{Vect}\ens{g_j,\ 0\leqslant j\leqslant \varphi(n+1)}=\mathrm{Vect}\ens{g_{\varphi(j)},\ 0\leqslant j\leqslant n+1}.$$
  L'inclusion réciproque est immédiate, et pour l'inclusion, si $\varphi(n)\leqslant k\leqslant \varphi(n+1)$, alors 
  $$g_k\in\mathrm{Vect}\ens{g_j,\ 0\leqslant j\leqslant \varphi(n)}$$
  par définition même de $\varphi(n+1)$, d'où l'égalité des ensembles.\\

  $\star$ On applique le procédé de Gramm-Schmidt à la famille $(g_{\varphi(n)})$ donc il existe une famille $(e_n)$ orthonormale telle que 
  $$\forall n\in \N,\ \mathrm{Vect}\ens{g_{\varphi(k)},\ 0\leqslant k\leqslant n}= \mathrm{Vect}\ens{e_k,\ 0\leqslant k\leqslant n}$$

  $\star$ Montrons que la famille est complète, soit $x\in H$ et $\varepsilon>0$, il existe $N\in\N\tq \forall n\geqslant N,\ \norm{g_N-x}<\varepsilon.$ Soit $n$ tel que $\varphi(n)>N$, alors 
  $$g_N\in \mathrm{Vect}\ens{g_j,\ 0\leqslant j\leqslant \varphi(n)}=\mathrm{Vect}\ens{g_{\varphi(k)},\ 0\leqslant k\leqslant n}=\mathrm{Vect}\ens{e_k, 0\leqslant k\leqslant n}$$
  donc $g_N=\dstyle\sum_{k=0}^na_ke_k$ et 
  $$\norm{x-\sum_{k=0}^na_ke_k}<\varepsilon,$$
  et $\mathrm{Vect}\ens{e_k,e\ 0\leqslant k\leqslant n}$ est dense dans $H$.
\end{Proof}


\begin{Th}
  Soit $H$ un espace de Hibert séparable,\\

  alors $H$ est isométriquement isomorphe à $\ell^2(\N)$.
\end{Th}


\begin{Proof}
  Soit $(e_n)$ une base hilbertienne de $H$, on pose
  $$\phi:\begin{array}{|ccl}
    H&\longrightarrow&\ell^2(\N)\\
    x&\longmapsto&\phi(x):=\left(\ps{x,e_n}\right)_{n\geqslant 0}
  \end{array}$$
  d'après l'inégalité de Bessel, $\phi$ est bien définie (et linéaire).
  D'après l'égalité de Parseval, pour $x\in H$, on a
  $$||x||_{H}^2=\sum_{n=0}^\infty\abs{\ps{x,e_n}}^2=||\phi(x)||_{\ell^2(\N)}^2$$
  donc $\phi$ est continue et isométrique.

  $\phi$ est injective, montrons qu'elle est aussi surjective : soit $i\in \N$, 
  $$\phi(e_j)=(\underbrace{0,\cdots,0}_{i-1},1,0,\cdots)=:\varepsilon_i$$
  donc 
  $$\mathrm{Im}\ \phi\subset\overline{\mathrm{Im}\ \phi}\subset\overline{\mathrm{Vect}\ens{\varepsilon_i,\ i\in\N}}=\ell^2(\N)$$
  et $\phi$ est bon bien bijective, ce qui conclut.
\end{Proof}{\color{white}test}\\[2em]


\NouvelleSubsection{Application aux séries de Fourier}

On rappelle que, par définition, 
$$\mathcal L^2[-\pi,\pi]:=\ens{f:[-\pi,\pi]\longrightarrow\C\text{ mesurable telle que }||f||_2<\infty}$$
où la norme 2 $||\cdot||_2$ est 
$$||f||_2:=\left(\frac{1}{2\pi}\int_{-\pi}^{\pi}|f(t)|^2\dt\right)^{\nicefrac12}.$$
Cette norme est en fait une \emph{semi-norme}, donc pour faire de $\mathcal L^2[-\pi,\pi]$ un espace métrique, on considère la relation d'équivalence suivante 
$$f\sim g\Leftrightarrow\ f=g\ \lambda\text{-pp}$$
et on va donc se placer dans 
$$L^2[-\pi,\pi]:=\mathcal L^2[-\pi,\pi]/\sim.$$


$L^2[-\pi,\pi]$ est un espace de Hilbert, on note, pour $n\in\Z$, $e_n:t\longmapsto\ex{itn}$, on a alors le théorème suivant :


\begin{Th}
  $\ens{e_n,\ n\in\Z}$ est une base hilbertienne de $L^2[-\pi,\pi]$
\end{Th}

Avant de prouver ce résultat, on va (temporairement) admettre le lemme suivant : 


\begin{lemme}
  On pose 
  $$C_c]-\pi,\pi[:=\ens{f:]-\pi,\pi[\longrightarrow\C\text{ continue à support compact}}$$

  alors cet ensemble est dense dans $L^2[-\pi,\pi]$
\end{lemme}



\begin{Proof}
  $\star$ La suite est orthogonale : par le calcul.\\

  $\star$ La suite est complète : soit $f\in L^2[-\pi,\pi]$ et $\varepsilon>0$. Par densité de $C_c]-\pi,\pi[$, il existe $g\in C_c]-\pi,\pi[$ telle que 
  $$||f-g||_2<\Nfrac{2}$$
  en particulier, puisque $g$ est continue, on a 
  $$g(-\pi)=0=g(\pi)$$
  donc $g$ est prolongeable en une fonction $2\pi$-périodique et continue. On peut donc appliquer le théorème de Féjer, donc il existe 
  $$p\in\mathrm{Vect}\ens{e_n,\ n\in\Z}\text{ telle que }||g-p||_2<\Nfrac{2}$$
  et par inégalité triangulaire, on trouve 
  $$||f-p||_2<\varepsilon,$$
  donc la famille $\ens{e_n,\ n\in\Z}$ est bien une base hilbertienne.
\end{Proof}


On va donc retrouver tous les résultats de la sous-partie précédente, à savoir 


\begin{Th}
  On peut décomposer toute fonction $f\in L^2[-\pi,\pi]$ en sa série de Fourier, plus précisément,\\

  $(\mathit 1)$ 
  $$f=\sum_{n=-\infty}^{\infty}\hat f(n)e_n$$
  et la série converge, où pour tout $n\in\Z$, on a posé 
  $$\hat f(n):=\ps{f,e_n}=\frac{1}{2\pi}\int_{-\pi}^{\pi}f(t)\ex{-int}\dt$$
  $(\mathit 2)$  
  $$||f||_2^2=\sum_{n=-\infty}^{\infty}|\hat f(n)|^2$$
  $(\mathit 2)$ pour $g\in L^2[-\pi,\pi]$,
  $$\ps{f,g}=\sum_{n=-\infty}^{\infty}\hat f(n)\overline{\hat g(n)}$$
\end{Th}



\Dnote{Tous ces résultats sont valables pour la norme 2, pour la norme infinie il n'y a pas nécessairement de convergence. Une conséquence du théorème de Banach-Steinhaus est que 
$$\big\{f\in C([-\pi,\pi])\tq\underset{n\geqslant 0}{\mathrm{sup}}|S_nf(0)|=+\infty\big\}$$
est dense dans $\left(C([-\pi,\pi]),||\cdot||_{\infty}\right)$.}


\NouvelleSection{Dualité et théorème de Riesz}


Soit $H$ un espace de Hilbert et $x_0\in H$, on pose 
$$\varphi_{x_0}:\begin{array}{|ccl}
  H&\longrightarrow&\C\\
  x&\longmapsto&\ps{x,x_0}
\end{array}$$
c'est une forme linéaire et, pour $x\in H$, on a 
$$\abs{\varphi_{x_0}(x)}=\abs{\ps{x,x_0}}\leqslant ||x||\ ||x_0||$$
donc $\varphi_{x_0}$ est continue avec $||\varphi_{x_0}||_{\mathrm{op}}\leqslant ||x_0||.$

On évalue en $x_0/||x_0||$ (si $x_0=0,$ $\varphi_{x_0}$ est l'application nulle) et
$$\varphi_{x_0}\left(\frac{x_0}{||x_0||}\right)=\frac{1}{||x_0||}\ps{x_0,x_0}=||x_0||$$
d'où $\nop{\varphi_{x_0}}=||x_0||.$


\begin{thC}{Riesz}
  Soit $\Phi\in H^*$,\\

   alors il existe un unique $x_0\in H$ tel que $\Phi=\varphi_{x_0}.$
\end{thC}


\begin{RQ}
  De plus, $x_0$ vérifie $||x_0||=\nop{\Phi}.$
\end{RQ}


\begin{Proof}
  $\star$ Existence :

  $\dagger$ Si $\Phi=0$, alors $x_0=0$ convient.

  $\dagger$ On suppose que $\Phi\neq0$, donc $\ker\ \Phi$ est un sous-espace strict de $H$. Il est de plus fermé car $\Phi$ est continue, il existe donc $g\in \left(\ker\Phi\right)^\bot.$
  Pour $x\in H$, on a 
  $$x-\frac{\Phi(x)}{\Phi(g)}g\in\ker\Phi$$
  donc (puisque $g\in \left(\ker\Phi\right)^\bot$)
  $$\forall x\in H,\ \ps{x-\frac{\Phi(x)}{\Phi(g)}g,g}=0$$
  autrement dit, 
  $$\forall x\in H,\ \Phi(x)=\ps{x,\Phi(g)g}$$
  on obtient le résultat en posant $x_0=\Phi(g)g$.\\

  $\star$ Unicité :
  Si $x_0$ et $x_1$ conviennent, 
  $$\forall x\in H,\ \Phi(x)=\ps{x,x_0}=\ps{x,x_1}$$
  donc 
  $$\forall x\in H,\ \ps{x,x_0-x_1}=0$$
  $$\norm{x_0-x_1}^2=0$$
  et $x_0=x_1$ d'où l'unicité de $x_0.$
\end{Proof}

\newpage

\NouvelleSubsection{Adjoint d'un opérateur continu}


\begin{corollaire}
  Soient $H_1,H_2$ des espaces de Hilbert munis des produits scalaire $\ps{\cdot,\cdot}_{H_1}$ et $\ps{\cdot,\cdot}_{H_2}$, soit $T\in\mathcal L(H_1,H_2)$,\\

  alors il existe un unique $T^*\in \mathcal L(H_2,H_1)$ tel que 
  $$\forall (h_1,h_2)\in H_1\times H_2,\ \ps{Th_1,h_2}_{H_1}=\ps{h_1,T^*h_2}_{H_2}.$$
  $T^*$ est appellé \emph{adjoint de }$T$.
\end{corollaire}


\begin{RQ}
  Si $H_1=H_2$ et $\dim\ H_1<\infty$, alors pour $(e_1,\cdots,e_n)$ un base de $H_1$, on a 
  $$\ps{Te_i,e_j}=\ps{e_i,T^*e_j}=\overline{\ps{T^*e_j,e_i}}$$
  donc la matrice de $T^*$ est la transposée conjuguée de la matrice de $T$. 
\end{RQ}


\begin{Proof}
  Soit $h_2\in H_2$, l'application
  $$\Phi:\begin{array}{|ccc}
    H_1&\longrightarrow&\K\\
    h_1&\longmapsto&\ps{Th_1,h_2}
  \end{array}$$
  est linéaire et continue, donc d'après le théorème de Riesz, il existe un unique $h(h_2)\in H_2$ tel que 
  $$\forall h_1\in H_1,\ \ps{Th_1,h_2}_{H_2}=\ps{h_1,h(h_2)}_{H_1}$$
  en notant $T^*=h$, on vérifie que $T^*$ est linéaire. Quant à la continuité, on a 
  $$\norm{T^*h_2}=\norm{h(h_2)}=\nop{\Phi}=\underset{||h_1||=1}{\mathrm{sup}}\abs{\ps{Th_1,h_2}}\leqslant ||h_2||\cdot\underset{||h_1||=1}{\mathrm{sup}}\norm {Th_1}\leqslant \nop{T}||h_2||,$$
  donc $T^*$ est continue et on a même la majoration de $\norm{T^*}$ suivante : $\norm{T^*}\leqslant \norm{T}$.
\end{Proof}


\begin{propC}{prop générales}
  Soient $T,S\in\mathcal L(H_1,H_2)$ et $\lambda\in\K$,\\
 
  alors\\
  $(\mathit 1)\ \left(\lambda T+ S\right)^*=\overline{\lambda}T^*+S^*$\\
  $(\mathit 2)\ \left(T^*\right)^*=T$\\
  $(\mathit 3)\ \norm{T^*}_{\mathrm{op}}=\norm{T}_{\mathrm{op}}$\\
  $(\mathit 4)\  \norm{TT^*}_{\mathrm{op}}=\norm{T}_{\mathrm{op}}^2$\\
  $(\mathit 5)$ si $TS$ a un sens, $\left(TS\right)^*=S^*T^*$
\end{propC}


\begin{ProofC}{Démonstration du $(\mathit 3)$}
  On a déjà $\norm{T^*}\leqslant \norm{T}$ et de plus, 
  $$\norm{T^*}\leqslant \norm{T}=\norm{T^{**}}\leqslant \norm{T^*}$$ 
  d'où l'égalité recherchée.  
\end{ProofC}


$\triangleright$\emph{ex :} on considère l'opérateur \emph{shif}
$$S:\begin{array}{|ccc}
  \ell^2(\N)&\longrightarrow&\ell^2(\N)\\
  (a_0,\cdots a_n,\cdots)&\longmapsto&(0,a_0,\cdots a_n,\cdots)
\end{array}$$
$S$ est un opérateur linéaire (unitaire) donc $S\in\mathcal L(\ell^2(\N))$, on va calculer son ajdoint. Soient $(a,b)\in\ell^2(\N)$, alors 
$$\ps{Sa,b}=\ps{a,S^*b}\Leftrightarrow\sum_{n=1}^\infty a_{n-1}\overline{b_n}=\sum_{n=0}^\infty a_{n}\overline{S^*b_n}$$
En particulier, en prenant $a=e_0=(1,0,\cdots)$, on trouve que $\overline{(S^*b)_0}=\overline{b_1}$, en prenant $a=e_1$, on trouve que $\overline{(S^*b)_1}=\overline{b_2}$, d'où (par récurrence);
$$(S^*b)=(b_1,\cdots,b_n,\cdots).$$\\



{\bf{\emph{Spectre d'un opérateur}}}\\[2em]



Soit $H$ un espace de Hilbert complexe et $T\in\mathcal L(H)$, on définit le \emph{spectre de }$T$ par 
$$\sigma(T):=\ens{\lambda\in \C\tq T-\lambda \Id\text{ n'est pas inversible}}$$
on définit aussi le \emph{spectre ponctuel de }$T$ par 
$$\sigma_p(T):=\ens{\lambda\in \C\tq \ker\ \left(T-\lambda \\Id\right)\neq \ens{0}}.$$



\begin{RQ}
  Si $\dim H<\infty$, alors $T-\lambda \Id$ non-inversible $\Leftrightarrow \ker \left(T-\lambda \Id\right)\neq \ens{0}$ donc $\sigma(T)=\sigma_p(T).$\\
\end{RQ}


\begin{RQ}
  Si $\lambda$ est tel que $\ker\left(T-\lambda \Id\right)\neq \ens{0}$, alors $T-\lambda \Id$ n'est pas inversible, donc il y a l'inclusion $\sigma_p(T)\subset \sigma(T).$\\
\end{RQ}


\begin{prop}
  Soit $T\in\mathcal L(H)$\\

  alors $\sigma(T)$ est un compact contenu dans $\overline{D(0,\nop{T})}.$
\end{prop}


On aura besoin des deux lemmes suivants : 


\begin{lemme}
  Pour $U\in\mathcal L(H)$ tel que $\nop{U}<1$,\\

  alors $\Id-U$ est inversible et 
  $$\Par{\Id-U}^{-1}=\sum_{n=0}^{\infty}U^n.$$
\end{lemme}


\begin{Proof}
  Tout d'abord, puisque $\norm{U}<1$, on a 
  $$\norm{U^n}\leqslant \norm{U}^n\tend 0$$
  donc $U^n\longrightarrow_{n\to\infty}0$.

  On calcule $(\Id-U)\sum_{n=0}^{\infty}U^n$ et $\sum_{n=0}^{\infty}U^n(\Id-U)$ :
  \begin{align*}
    (\Id-U)\sum_{n=0}^{\infty}U^n&=\Lim (\Id-U)\sum_{k=0}^{n}U^k\\
    &=\Lim \sum_{k=0}^{n}U^k-U^{k+1}\\
    &=\Lim \Id-U^{n+1}\\
    (\Id-U)\sum_{n=0}^{\infty}U^n&=\Id
  \end{align*}
  Donc $\sum_{n=0}^{\infty}U^n$ est un inverse à droite de $\Id-U$, montrons que c'est aussi un inverse à gauche :
  \begin{align*}
    \sum_{n=0}^{\infty}U^n(\Id-U)&=\Lim \sum_{k=0}^{n}U^k(\Id-U)\\
    &=\Lim \sum_{k=0}^{n}U^k-U^{k+1}\\
    &=\Lim \Id-U^{n+1}\\
    \sum_{n=0}^{\infty}U^n(\Id-U)&=\Id
  \end{align*}
\end{Proof}


\begin{lemme}
  L'ensemble 
  $$\mathrm{Inv}\mathcal L(H):=\ens{U\in\mathcal L(H)\tq U\text{ est inversible}}$$
  est un ouvert de $\mathcal L(H).$
\end{lemme}


\begin{Proof}
  Soit $T_0\in\mathrm{Inv}\mathcal L(H)$, on cherche $\delta>0$ tel que 
  $$\norm{T-T_0}\leqslant \delta\Rightarrow T\in\mathrm{Inv}\mathcal L(H)$$
  Soit 
  $$T=T_0-(T_0-T)=T_0\Par{\Id-T_0^{-1}\Par{T_0-T}}$$
  donc 
  si l'on prend $\delta$ tel que 
  $$\delta\norm{T_0^{-1}}<1$$
  alors $$\norm{T-T_0}\leqslant \delta\Rightarrow\norm{T_0^{-1}\Par{T_0-T}}<1$$ 
  donc
  $$T=\Id-T_0^{-1}\Par{T_0-T}\in\mathrm{Inv}\mathcal L(H)$$ 
  et $\mathrm{Inv}\mathcal L(H)$ est bien un ouvert de $\mathcal L(H)$.
\end{Proof} 


\begin{ProofC}{La démonstration de la propriété}
  $\star$ On commence par montrer que $\sigma(T)\subset \overline{D(0,\norm{T})}$, soit $\abs{\lambda}>\norm{T}$, montrons que $T-\lambda \Id$ est inversible : on a 
  $$T-\lambda\Id=-\lambda\left(\Id-\frac{T}{\lambda}\right)$$
  et puisque $\dstyle\abs{\frac{\norm T}{\lambda}}<1$, on peut utiliser le développement en série entière de $\dstyle x\longmapsto\Par{1-x}^{-1}$
  donc 
  $$\Par{T-\lambda\Id}=-\lambda\sum_{n=0}^\infty\Par{\frac{T}{\lambda}}^k.$$
  donc $\abs{\lambda}>\norm{T}$ implique que $T-\lambda \Id$ est inversible donc que $T-\lambda\Id\notin\sigma(T).$\\

  $\star$ On a aussi 
  $$\C\backslash\sigma(T)=\ens{\lambda\in\C\ :\ T-\lambda\Id\in\mathrm{Inv}\mathcal L(H)}=f^{-1}\ens{\mathrm{Inv}\mathcal L(H)}$$
  où $f$ est 
  $$f:\begin{array}{|ccc}
    \C&\longrightarrow&\mathcal L(H)\\
    \lambda&\longmapsto&T-\lambda\Id
  \end{array}$$
  $f$ étant continue, $\C\backslash\sigma(T)$ est un ouvert donc $\sigma(T)$ est fermé. $\sigma(T)$ est aussi borné (car inclus dans $\overline{D(0,\norm{T})}$) donc c'est un fermé borné de $\C$, autrement dit un compact.
\end{ProofC}


\begin{prop}
  Soit $T\in\mathcal L(H)$,\\

  alors 
  $$\sigma(T^*)=\overline{\sigma(T)}=\ens{\overline{\lambda},\ \lambda\in\sigma(T)}.$$
\end{prop}


\begin{Proof}
  Soit $\mu\in \C$, on a 
  \begin{align*}
    \mu\notin\sigma(T^*)&\Leftrightarrow T^*-\mu\Id\text{ est inversible}\\
    &\Leftrightarrow \Par{T-\overline\mu\Id}^*\text{ est inversible}\\
    &\Leftrightarrow\exists A\in\mathcal L(H)\ : \ A\Par{T-\overline\mu\Id}^*=\Id=\Par{T-\overline\mu\Id}^*A\\
    &\Leftrightarrow\exists A\in\mathcal L(H)\ : \ \Par{T-\overline\mu\Id}A^*=\Id=A^*\Par{T-\overline\mu\Id}\\
    &\Leftarrow T-\overline\mu\Id\text{ est inversible}\\
    \mu\notin\sigma(T^*)&\Leftrightarrow\overline{\mu}\notin\sigma(T)
  \end{align*}
\end{Proof}


$\triangleright$\emph{ex :} retour sur le \emph{shift}, montrons que $\sigma(S)=\overline{D(0,1)}$:


\begin{Proof}
  $\star$ Pour l'inclusion, puisque $\nop{S}=1$, on sait déjà que $\sigma(S)\subset\overline{D(0,1)}$\\

  $\star$ Pour l'inclusion réciproque, soit $\lambda\in\C$ et $æ\in\ell^2(\N)$, alors 
  \begin{align*}
    S^*a=\lambda a&\Leftrightarrow\Par(a_1,a_2\cdots)=\lambda(a_0,a_1,\cdots)\\
    &\Leftrightarrow \forall k\geqslant 0,\ a_{k+1}=\lambda a_k\\
    S^*a=\lambda a&\Leftrightarrow \forall k\geqslant 0,\ a_{k+1}=\lambda^k a_0
  \end{align*}
  et $a\in\ell^2(\N)$ \ssi $\abs{\lambda}<1$ (suite géométrique).
  
  Ainsi, $\sigma(T^*)=D(0,1)$.

  On a donc la chaîne d'inclusion suivante : 
  $$\sigma_p(S^*)=D(0,1)\subset \sigma(S)\subset\overline{D(0,1)}$$
  et puisque $\sigma(S)$ est un fermé, nécessairement $\sigma(S)=\overline{D(0,1)}$
\end{Proof}



\NouvellePart{Espaces $\mathcal L^p$ et $L^p$}


Beaucoups de rappels du cours d'Analyse pour l'ingénieur, \emph{s/o} Augustin Mouze;

\PremiereNouvelleSection{Espaces $\mathcal L^p$, inégalités de Hölder et Minkowski}

\begin{Def}
  Soit $(X,m,\mu)$ un espace mesuré et $1\leqslant p< \infty$, on pose 
$$\mathcal L^p=\mathcal L^p(X,m,\mu)=\ens{f:X\longrightarrow\C\text{ mesurable telle que}\int_X\abs{f}^p\mathrm d\mu<\infty}$$
et l'on pose 
$$\norm{f}_p:=\left(\int_X\abs{f}^p\mathrm d\mu\right)^{\nicefrac{1}{p}}.$$
\end{Def}


\begin{RQ}
  Pour $p=1$, $\mathcal L^1$ est l'espace des fonctions intégrales, on sait que c'est un sous-espace vectoriel et que $\norm{\cdot}_1$ définit une \emph{semi-norme}, \emph{i.e.} une norme à qui il manque la propriété de définition, c'est à dire que 
$$\norm{f}_1=0\Leftrightarrow f=0\ \mu\text{-p.p.}$$
\end{RQ}

\begin{Def}
  Une fonction $f:X\longrightarrow\C$ est dite \emph{essentiellement bornée} si, et seulement si 
$$\exists M\geqslant 0\text{ tel que }\mu\left(\ens{x\in X\ :\ \abs{f(x)}\geqslant M}\right)=0\qquad (i)$$
la borne inférieure des $M$ vérifiant $(i)$ est \emph{la borne supérieure essentielle de }$f$, on la note $\norm{f}_{\infty}$, donc 
$$\norm{f}_\infty=\mathrm{inf}\ens{M\geqslant 0 : \abs{f}\leqslant M\ \mu\text{-p.p.}}.$$
On peut donc définir $\mathcal L^\infty$ comme suit
$$\mathcal L^\infty=\mathcal L^\infty(X,m,\mu)=\ens{f:X\longrightarrow\C\text{ essentiellement bornée}}.$$
\end{Def}


\begin{RQ}
Les élément de $\mathcal L^\infty$ peuvent être assez moches, par exemple 
$$f(x)=\begin{cases}
  1\text{ si }x\in\R\backslash\Q\\
  +\infty\text{ si }x\in\Q
\end{cases}$$
appartient à $\mathcal L^\infty$.
\end{RQ}


\begin{RQ}
  Pour $f\in\mathcal L^\infty$, on a 
  $$\abs{f}\leqslant \norm{f}_\infty\ \mu\text{-p.p.}$$
\end{RQ}


\begin{Proof}
  Pour $n\geqslant 1$, on pose 
  $$N_n:=\ens{x\in X\ :\ \abs{f(x)}\geqslant \norm{f}_\infty+\frac1n}$$
  alors $\mu(N_n)=0$ par hypothèse et 
  $$N:=\bigcup_{n\geqslant 0}N_n=\ens{x\in X\ :\ \abs{f(x)}>\norm{f}_\infty}$$
  est aussi de mesure nulle, puisque c'est la réunion dénombrable d'ensembles de mesures nulles.
\end{Proof}


On voit voir que les $\mathcal L^p$ sont des espaces vectoriels, et qu'il est possible d'en faire des espaces vectoriels normés.

On commence par deux inégalité de convexité, qui seront utiles au cours du chapitre.


\NouvelleSubsection{Deux inégalités de convexité}


\begin{prop}
  Soient $(\alpha,\beta)\in[0,1]$ tels que $\alpha+\beta=1$,\\

  alors 
  $$\forall (u,v)\in\R_+,\ u^\alpha+v^\beta\leqslant \alpha u+\beta v.$$
\end{prop}


\begin{Proof}
  Par concavité de la fonction $\ln$, on a 
  $$\alpha\ln u+\beta\ln v\leqslant\ln\left(\alpha u+\beta v\right),$$
  autrement dit, 
  $$\ln\left(u^\alpha v^\beta\right)\leqslant\ln\left(\alpha u+\beta v\right).$$
  On compose cette inégalité par la fonction $\exp$ qui est strictement croissante et l'on obtient
  $$u^\alpha+v^\beta\leqslant \alpha u+\beta v.$$
\end{Proof}

Avant de passer à l'inégalité de Jensen, on rappelle l'inégalité des pentes : 


\begin{prop}
  Soit $\varphi:I\subset R\longrightarrow\R$ une application convexe, où $I$ est un intervalle ouvert de $\R$,\\

  alors 
  $$\forall x<y<z,\ \frac{\varphi(y)-\varphi(x)}{y-x}\leqslant\frac{\varphi(z)-\varphi(x)}{z-x}\leqslant \frac{\varphi(z)-\varphi(y)}{z-y}$$
\end{prop}


\begin{propC}{inégalité de Jensen}
  Si $\mu$ est une mesure de probabilité, pour $f:X\longrightarrow I\subset \R$ intégrable (où $I$ est un intervalle ouvert), pour $\varphi:I\longrightarrow\R$ convexe,\\

  on a 
  $$\varphi\Par{\int_Xf\mathrm d\mu}\leqslant \int_X\varphi\circ f\mathrm d\mu.$$
\end{propC}


On commence par un lemme,


\begin{lemme}
  Soit $\varphi:I\subset R\longrightarrow\R$ une application convexe, où $I$ est un intervalle ouvert de $\R$ et $x\in I$,\\

  alors $h_x:t\in I\backslash\ens{x}\longmapsto\dstyle\frac{\varphi(x)-\varphi(t)}{x-t}$ est croissante admet des limites en $x^+$ et $x^-$.
\end{lemme}


\begin{Proof}
  On va appliquer l'inégalité des pentes dans les trois configurations possibles,\\
  $\dagger$ si $a<b<x$,
  $$\begin{array}{rcccl}
    \dstyle\frac{\varphi(b)-\varphi(a)}{b-a}&\leqslant&\dstyle\frac{\varphi(x)-\varphi(a)}{x-a}&\leqslant& \dstyle\frac{\varphi(x)-\varphi(b)}{x-b}\\[1.1em]
    &&h_x(a)&\leqslant& h_x(b)
  \end{array}$$
  $\dagger$ si $a<x<b$,
  $$\begin{array}{rcccl}
    \dstyle\frac{\varphi(x)-\varphi(a)}{x-a}&\leqslant&\dstyle\frac{\varphi(b)-\varphi(a)}{b-a}&\leqslant& \dstyle\frac{\varphi(b)-\varphi(x)}{b-x}\\[1.1em]
    h_x(a)&\leqslant& &\leqslant& h_x(b)
  \end{array}$$
  $\dagger$ si $x<a<b$,
  $$\begin{array}{rcccl}
    \dstyle\frac{\varphi(a)-\varphi(x)}{a-x}&\leqslant&\dstyle\frac{\varphi(b)-\varphi(x)}{b-x}&\leqslant&\dstyle\frac{\varphi(a)-\varphi(b)}{a-b}\\[1.1em]
  h_x(a)&\leqslant&h_x(b)&&
  \end{array}$$
  Ainsi on constate que $h_x$ est effectivement croissante.

  Dans la deuxième ligne, on constate que $h_x$ est majorée sur $]-\infty,x[\cap I$, donc puisque $h_x$ est croissante, $\underset{t\to x^-}{\mathrm{lim}}h_x(t)$ existe (on fait tendre $a\to x^-$).

  De même, $h_x$ est minorée sur $I\cap]x,+\infty[$, donc puisque $h_x$ est croissante, $\underset{t\to x^+}{\mathrm{lim}}h_x(t)$ existe (on fait tendre $b\to x^+$).
\end{Proof}


\begin{ProofC}{Démonstration de l'inégalité de Jensen}
  Soit $x\in I$ et $\alpha$ tel que $\underset{t\to x^-}{\mathrm{lim}}h_x(t)\leqslant \alpha\leqslant \underset{t\to x^+}{\mathrm{lim}}h_x(t)$, ainsi 
  $$\forall t\in I,\ \alpha(t-x)\leqslant \varphi(t)-\varphi(x)$$
  (on considère d'abord le cas $t\neq x$, puis on étend en $t=x$), donc 
  $$\forall t\in I,\ \varphi(t)\geqslant \alpha t +\Par{\varphi(x)-\alpha x}=\alpha t +\beta.$$
  On remarque que $\varphi(x)=\alpha x+\beta$, donc en chaque point de $I$, il existe une fonction affine inférieure à $\varphi$ partout et qui coincide en un unique point. On peut donc donc ré-écrire $\varphi$ comme suit
  $$\forall x\in I,\ \varphi(x)=\underset{(\alpha,\beta)\in E}{\mathrm{sup}}\ \Par{\alpha x+\beta}$$
  où $E$ est l'ensemble des fonctions affines majorées par $\varphi$, 
  $$E:=\ens{(\alpha,\beta)\in \R\tq \forall x\in\R,\ \varphi(x)\geqslant \alpha x+\beta}.$$
  On a donc les majorations suivantes :
  \begin{align*}
    \int_X\varphi\circ f\mathrm d\mu&\geqslant \underset{(\alpha,\beta)\in E}{\mathrm{sup}}\int_X \Par{\alpha f+\beta}\mathrm d\mu\\
    &\geqslant \underset{(\alpha,\beta)\in E}{\mathrm{sup}} \Par{\alpha\int_X f\mathrm d\mu+\beta}\\
     \int_X\varphi\circ f\mathrm d\mu&\geqslant \varphi\Par{\int_Xf\mathrm d\mu}.
  \end{align*}
\end{ProofC}


\begin{Def}
  Soit $1\leqslant p\leqslant+\infty$, l'\emph{exposant conjugué de }$p$ est l'unique réel vérifiant
  $$\frac1p+\frac1q=1$$
  où, par convention, si $p=1,\ q:=+\infty$ et réciproquement si $p=+\infty$ alors $q:=1$
\end{Def}


\begin{propC}{inégalité d'Hölder}
  Soient $1< p< \infty$ et $q$ son exposant conjugué, soient $(f,g):X\longrightarrow\C$ mesurable\\

  alors
  $$||fg||_1\leqslant||f||_p||g||_q.$$
\end{propC}
  
\begin{Proof}
  On suppose que $||f||_p\notin \ens{0,\infty}$ et $||g||_p\notin \ens{0,\infty}$ sinon l'inégalité est triviale, d'où $f\in\mathcal L^p$ et $g\in\mathcal L^q$.

  On pose 
  $$F:=\frac{f}{\norm{f}_p}\text{ et }G:=\frac{g}{\norm{g}_q}$$
  on a, d'après le lemme,
  $$\forall x\in X,\ \abs{F(x)G(x)}\leqslant \frac{1}{p}\abs{F(x)}^p+\frac{1}{q}\abs{G(x)}^q$$
  on intègre sur $X$ d'où
  $$\int_X\frac{\abs{fg}}{\norm{f}_p\norm{g}_q}\mathrm d\mu\leqslant \frac{1}{p}\int_X\abs{F(x)}^p\mathrm d\mu+\frac{1}{q}\int_X\abs{G(x)}^q\mathrm d\mu=\frac1p+\frac1q=1$$
  donc
  $$\int_X\abs{fg}\mathrm d\mu\leqslant \norm{f}_p\norm{g}_q.$$
\end{Proof}


\begin{propC}{inégalité d'Hölder, bis}
  Soient $f\in\mathcal L^1$ et $g\in\mathcal L^\infty$\\

  alors
  $$||fg||_1\leqslant||f||_1||g||_\infty.$$
\end{propC}


\begin{Proof}
  On a 
  $$\abs{g(x)}\leqslant \norm{g}_\infty\ \mu\text{-p.p.}$$
  donc 
  $$\abs{f(x)g(x)}\leqslant \abs{f(x)}\norm{g}_\infty\ \mu\text{-p.p.}$$
  et
  $$\norm{fg}_1\leqslant\int_X\abs{f(x)g(x)}\mathrm d\mu(x)\leqslant\norm{f}_1\norm{g}_\infty$$
\end{Proof}
  
  
\begin{propC}{inégalité de Minkowski}
  Soient $1\leqslant p\leqslant+\infty$ et $(f,g)\in\mathcal L^p$, alors\\
  
  $$||f+g||_p\leqslant ||f||_p+||g||_p,$$
  autrement dit, $\norm{\cdot}_p$ vérifie l'inégalité triangulaire.
\end{propC}
  
\begin{Proof}
  Soient $(f,g)\in\mathcal L^p$, on suppose que l'une au moins est non nulle $\mu$-presque partout,\\
  $\dagger\quad$si $p=1$ ou $p=+\infty$, l'inégalité triangulaire suffit.\\
  $\dagger\quad$si $p>1$, on considère $q$ l'exposant conjugué de $p$. On écrit $$|f+g|^p=|f+g||f+g|^{p-1}\leqslant|f||f+g|^{p-1}+|g||f+g|^{p-1}\ (i).$$ On va maintenant appliquer l'inégalité de Hölder, qui nous dit tout d'abord que $|f||f+g|^{p-1}$ et $|g||f+g|^{p-1}$ sont $\mu$-intégrables et :
  $$||f(f+g)^{p-1}||_1=\int_X|f||f+g|^{p-1}\mathrm d\mu \leqslant ||f||_p\bigg(\int_X\Big(|f+g|^{(p-1)}\Big)^q\mathrm d\mu\bigg)^{\nicefrac1q}\ (ii)$$
  $$||g(f+g)^{p-1}||_1=\int_X|g||f+g|^{p-1}\mathrm d\mu \leqslant ||g||_p\bigg(\int_X\Big(|f+g|^{(p-1)}\Big)^q\mathrm d\mu\bigg)^{\nicefrac1q}\ (iii).$$
  En sommant $(ii)$ et $(iii)$ et en comparant avec $(i)$, on obtient :
  $$\int_X|f+g|^p\mathrm d\mu \leqslant \bigg(||f||_p+||g||_p\bigg)\bigg(\int_X|f+g|^{(p-1)q}\mathrm d\mu\bigg)^{\nicefrac1q}.$$
  On peut diviser par $\dstyle\Big(\int_X|f+g|^{(p-1)q}\mathrm d\mu\Big)^{\nicefrac1q}$, puisque $f$ ou $g$ est non nulle $m$-presque partout, on obtient (on rappelle que $(p-1)q=p$) :
  $$\bigg(\int_X|f+g|^p\mathrm d\mu\bigg)^{1-\nicefrac1q} \leqslant \bigg(||f||_p+||g||_p\bigg).$$
  Puisque $\dstyle1-\nicefrac1q=\nicefrac1p$, on obtient finalement :
  $$\bigg(\int_X|f+g|^p\mathrm d\mu\bigg)^{\nicefrac1p} \leqslant \bigg(||f||_p+||g||_p\bigg).$$
  Ce qui se ré-écrit en $||f+g||_p\leqslant ||f||_p+||g||_p$.
\end{Proof}


Ainsi, nos espaces $\mathcal L^p$ (pour $1\leqslant p\leqslant \infty$) sont tous des espaces vectoriels, et $\norm{\cdot}_p$ est une \emph{semi-norme}, on va voir qu'il est possible d'en faire des espaces vectoriels normés.


\NouvelleSection{Théorème de Riesz-Fischer, espaces $L^p$}

Soit $(X,m,\mu)$ un espace mesuré et $1\leqslant p\leqslant \infty$. On sait que, pour $f\in\mathcal L^p$, 
$$\norm{f}_p=0\Leftrightarrow f=0\ \mu\text{-p.p.},$$
on va donc considérer la relation d'équivalence (exercice) suivante
$$f\sim g \Leftrightarrow f=g\ \mu\text{-p.p.}$$
On va définir les espaces $L^p$ comme étant le quotient des $\mathcal L^p$ par cette relation d'équivalence, donc 
$$L^p=L^p(X,m,\mu):=\mathcal L^p/\sim$$
ainsi, en notant $[f]$ la classe d'une fonction de $\mathcal L^p$, on a 
$$L^p:\ens{[f],\ f\in\mathcal L^p}.$$
On va le munir d'une norme, 
$$\norm{[f]}_p:=\norm{f}_p$$
il n'y a pas d'ambiguité dans la définition de $\norm{[f]}_p$ puisque les élements d'une classe d'équivalence sont identiques à un ensemble de mesure nulle près.

Les $L^p$ sont donc des espaces vectoriels normés, puisque 
$$\norm{[f]}_p=0\Leftrightarrow [f]=[0]=0.$$

Dans la suite du document, on fera toujours l'identification entre $f$ et $[f]$, donc pour $(f,g)\in L^p$, $f=$ signifie que $f=g\ \mu$-p.p.


\begin{Th}
  Soit $1\leqslant p< \infty$, soit $(f_n)\in L^p$ tels que $\sum_{n\geqslant 0}\norm{f_n}_p<\infty$,\\

  alors $\sum_{n\geqslant 0}f_n$ converge $\mu$-p.p. et en notant $F=\sum_{n\geqslant 0}f_n$, on a 
  $$\norm{\sum_{n= 0}^Nf_n-F}_p\longrightarrow 0\text{ quand }N\to\infty.$$
\end{Th}


\begin{Proof}
  On va définir
  $$G_n:=\sum_{k=0}^n\abs{f_n},\ G:=\sum_{n\geqslant 0}\abs{f_n}$$
  et
  $$F_n:=\sum_{k=0}^nf_n,\ F:=\sum_{n\geqslant 0}f_n.$$
  D'après l'inégalité de Minkowski, on a 
  $$\forall n\in\N,\ \norm{G_n}_p\leqslant \sum_{k=0}^n\norm{f_n}<\infty.$$
  De plus, $\Par{G_n(x)}_{n\geqslant 0}$ est une suite croissante donc on peut y appliquer le théorème de Beppo-Lévi et 
  \begin{align*}
    \int_X \abs{G(x)}^p\mathrm d\mu&=\int_X\underset{n\to\infty}{\mathrm{lim}}G_n(x)^p\mathrm d\mu\\
    &=\underset{n\to\infty}{\mathrm{lim}}\int_XG_n(x)^p\mathrm d\mu\\
    &=\underset{n\to\infty}{\mathrm{lim}}\norm{G_n}_p^p\\
    &=\underset{n\to\infty}{\mathrm{lim}}\norm{\sum_{i=0}^n f_n}_p^p\\
    \int_X \abs{G(x)}^p\mathrm d\mu&\leqslant\underset{n\to\infty}{\mathrm{lim}}\sum_{i=0}^n\norm{ f_n}_p^p
  \end{align*}
  Puisque la séries de $\sum_{n\geqslant 0}\norm{f_n}_p$ converge, APCR $\norm{f_n}_p<1$ donc APCR $\norm{f_n}_p^p<\norm{f_n}_p$ et par comparaison de séries à termes positifs, la série des $\sum_{n\geqslant 0}\norm{f_n}_p^p$ converge et 
  $$\int_X \abs{G(x)}^p\mathrm d\mu<\infty.$$

  Donc $G^p\in L^1$, ains $G<\infty\ \mu$-p.p. donc il existe $N\subset X$ de mesure nulle tel que 
  $$\forall x\in X\backslash N,\ F(x)=\sum_{n\geqslant 0}f_n(x)<\infty.$$
  Donc, à part sur un ensemble de mesure nulle, $F_n$ converge simplement vers $F$ et on a aussi 
  \begin{align*}
    \forall x\in X\backslash N,\ \abs{F_n(x)-F(x)}^p&=\abs{\sum_{k=n+1}^\infty f_n(x)}^p\\
    &\leqslant \Par{\sum_{k=n+1}^\infty \abs{f_n(x)}}^p\\
    \abs{F_n(x)-F(x)}^p&\leqslant G(x)^p\in L^1
  \end{align*}
  on peut donc appliquer le théorème de convergence dominée et 
  $$\Lim \int_{X\backslash N}\abs{F_n-F }^p\mathrm d\mu=\int_{X\backslash N}\Lim\abs{F_n-F }^p\mathrm d\mu=0$$
  et puisque l'intégrale ne change pas si on rajoute/enlève un ensemble de mesure nulle, 
  $$\Lim \int_X\abs{F_n-F }^p\mathrm d\mu=\Lim\norm{F_n-F}_p^p=0.$$
\end{Proof}


\begin{thC}{de Riesz-Fischer}
  Soit $1\leqslant p< \infty$,\\

  l'espace $L^p$ est un Banach, ou plus précisément, pour $(f_n)\in L^p$ une suite de Cauchy\\
  $(\mathit 1)\ \exists f\in L^p$ telle que 
  $$\norm{f_n-f}_p\tend 0$$
  $(\mathit 2)$ il existe une sous-suite telle que 
  $$f_{n_k}(x)\tend f(x)\ \mu\text{-p.p.}$$
\end{thC}


\begin{Proof}
  Soit $(f_n)\in L^p$ une suite de Cauchy, on construit l'extractrice $n:N\longrightarrow N$ strictement croissante de telle sorte que 
  $$\forall k\geqslant 1,\ \forall n,m\geqslant k,\ \norm{f_{n_n}-f_{n_m}}_p\leqslant\frac{1}{k^2}$$
  Soit 
  $$u_0:=f_{n_1}\text{ et }u_k:=f_{n_{k+1}}-f_{n_k}$$
  ainsi $\forall k\in \N,\ \norm{u_k}_p\leqslant \nicefrac{1}{k^2}$ donc 
  $$\sum_{k\geqslant 0}\norm{u_n}_p<\infty$$
  D'après le théorème précédent, il existe $f\in L^p$ tel que 
  $$\sum_{k=0}^nu_k\tend f\ \mu\text{-p.p.}$$
  et
  $$\norm{\sum_{k=0}^n u_k-f}_p\tend 0$$
  donc, puisque la série des $u_k$ est téléscopique, 
  $$f_{n_n}\tend f\ \mu\text{-p.p.}$$
  et 
  $$\norm{f_{n_n}-f}_p\tend 0.$$
  Pour le point $(\mathit 1)$, on écrit 
  $$\norm{f_n-f}_p\leqslant \norm{f_n-f_{n_k}}_p+\norm{f_{n_k}-f}_p$$
  puisque chacun de termes tend vers 0, on obtient bien 
  $$\norm{f_n-f}_p\tend 0.$$
\end{Proof}


\begin{Th}
  $L^\infty$ est un espace de Banach.
\end{Th}


\begin{Proof}
  Soit $(f_n)\in L^\infty$ une suite de Cauchy, donc 
  $$\forall\varepsilon>0,\ \exists n(\varepsilon)\in\N\ :\ \forall n,m\geqslant n(\varepsilon),\ \norm{f_n-f_m}_\infty\leqslant \varepsilon.$$
  On pose 
  $$N_{n,m}(\varepsilon):=\ens{x\in X\tq \abs{f_n(x)-f_m(x)}>\varepsilon}$$
  tous les $N_{n,m}(\varepsilon)$ sont de mesure nulles par hypothèse, donc 
  $$N:=\bigcup_{j\geqslant 1}\bigcup_{n,m\geqslant n(\nicefrac1j)}N_{n,m}(\nicefrac1j)$$
  $N$ est de mesure nulle en tant que réunion dénombrable d'ensembles de mesures nulles et 
  $$\forall x\in X\backslash N,\ \forall j\in\N,\ \forall n,m\geqslant n(\nicefrac1j),\ \abs{f_n(x)-f_m(x)}\leqslant\frac1j$$
  par passsage au $\mathrm{sup}$, on a 
  $$\forall j\in\N,\ \forall n,m\geqslant n(\nicefrac1j),\ \underset{x\notin N}{\mathrm{sup}}\abs{f_n(x)-f_m(x)}\leqslant\frac1j.$$
  De plus, les $(f_n)$ sont essentiellements bornées, donc chaque $f_n$ est bornée, sauf sur un ensemble $M_n$, qui est de mesure nulle.
  On considère 
  $$\tilde N:=N\cup\Par{\bigcup_{n\geqslant 0}M_n}$$
  qui est aussi de mesure nulle.
  Donc $(f_n)$ est de Cauchy dans l'espace $C_b(X\backslash \tilde N,\norm{\cdot}_\infty)$ où $\norm{\cdot}_\infty$ est la norme $\mathrm{sup}$ habituelle. Cet espace est un Banach (on le montre de la même façon qu'on montre que $(C(K,\K,\norm{\cdot}_\infty)$ est complet) donc $f_n$ converge uniformément sur $X\backslash \tilde N$ vers $f\in L^{\infty}$.

  Sur $\tilde N$, on pose $f=0$ donc $f$ est toujours dans $L^\infty$, donc on a bien 
  $$\norm{f_n-f}_\infty\longrightarrow 0\text{ quand }n\to\infty.$$
\end{Proof}


\NouvelleSection{Théorèmes de densité}


Soit $1\leqslant p< \infty$


\begin{lemme}
  Soit $f:X\longrightarrow[0,+\infty]$ une application mesurable, pour $n\in\N$ et $0\leqslant k\leqslant n2^n-1$, on pose 
  $$E_{n,k}:=\ens{x\in X\tq\frac{k}{2^n}\leqslant f(x)\leqslant \frac{k+1}{2^n}},$$
  $$F_n:=\ens{x\in X\tq f(x)\geqslant n}.$$
  Alors
  $$s_n:=\sum_{k=0}^{n2^n-1}\frac{k}{2^n}\chi_{E_{n,k}}+n\chi_{F_n}$$
  est une suite croissante de fonctions étagées convergeant simplement vers $f$.
\end{lemme}


\begin{Proof}
  Se référer au cours de L3/AIN
\end{Proof}

\begin{corollaire}
  Soit $\mathcal E:=\mathrm{Vect}\ens{\chi_A,\ a\in m}$ l'ensemble des fonctions étagées,\\

  alors $\mathcal E\cap L^p$ est dense dans $L^p$.
\end{corollaire}


\begin{RQ}
  Pour $f\in L^p$, $a>0$ et $A_a:=\ens{x\in X\ :\ \abs{f(x)}}$, alors $\chi_{A_a}\in L^p$, puisque 
  $$a^p\mu(A_a)\leqslant\int_{A_a}\abs{f}^p\mathrm d\mu\leqslant \int_{X}\abs{f}^p\mathrm d\mu<\infty.$$
\end{RQ}


\begin{Proof}
  En séparant 
  $$f=\Re f+i\Im f$$
  on peut supposer que $f$ est réelle et en séparant 
  $$f=f⁺-f⁻$$
  on peut supposer que $f$ est positive.

  D'après le lemme, il existe $(s_n)$ une suite croissante de fonctions étagées positives convergeant simplement vers $f$. D'après la remaque et l'expression explicite de $s_n$, 
  $$\forall n\in\N,\ s_n\in \mathcal E\cap L^p$$
  et $f_n(x)\longrightarrow f(x)$ sur $X$. 
  La limite étant croissante, on a 
  $$\forall x\in X,\ 0\leqslant f(x)-s_n(x)\leqslant f(x),$$
  $t\mapsto t^p$ étant croissante,
  $$\forall x\in X,\ \abs{f(x)-s_n(x)}^p\leqslant \abs{f(x)}^p$$
  on peut donc appliquer le théorème de convergence dominée et 
  $$\int_X\abs{f-s_n}^p\mathrm d\mu=\norm{f-s_n}_p^p\tend 0.$$
\end{Proof}


Dans la suite du document, $X=\R$, $m=\mathcal B(\R)$ et $\mu$ est la mesure de Lebesgue, donc $L^p=L^p(\R,\mathcal B(\R),\lambda)$.


\begin{Def}
  Pour $f\in L^p$, on appelle \emph{support de }$f$ le complémentaire du plus grand ouvert $U$ tel que 
  $$f\big|_U=0\ \lambda\text{-p.p.}$$
  Ainsi, si $f$ est continue, on a 
  $$\mathrm{supp}\ f=\overline{\ens{x\in X\ :\ f(x)\neq 0}}.$$
\end{Def}


On note $C_{cc}$ l'ensemble des fonctions continues à support compact, 
$$C_{cc}:=\ens{f:\R\longrightarrow\C\text{ continue à support compact}}$$
donc 
$$f\in C_{cc}\Leftrightarrow\begin{cases}
  f\text{ est continue }\\
  \exists a>0\ :\ \abs{x}\geqslant a\Rightarrow f(x)=0.
\end{cases}$$


\begin{Th}
  Soit $1\leqslant p< \infty$,\\

  alors $C_{cc}$ est dense dans $L^p$.
\end{Th}


La preuve de ce théorème requiert le lemme suivant, qui découle de la construction de la mesure de Lebesgue, donc que l'on ne démontrera pas.


\begin{lemme}
  La mesure de Lebesgue est \emph{régulière}, c'est à dire que pour $A\in\mathcal B(\R)$ avec $\lambda(A)<\infty$, pour $\varepsilon>0$, il existe un compact $K$ et un ouvert $\O$ tel que 
  $$K\subset A\subset \O\text{ et }\lambda(\O\backslash K)\leqslant \varepsilon.$$
\end{lemme}


\begin{Proof}
  On a déjà montré que $\mathcal E\cap L^p$ est dense dans $L^p$, donc il suffit de montrer que pour $A\in \mathcal B(\R)$ de mesure finie, $\chi_A\in\overline{C_{cc}}$ (au sens de la norme $L^p$).

  Soit $A\in \mathcal B(\R)$ et $\varepsilon>0$, on considère la propriété de régularité de la mesure de Lebesgue avec $\Par{\Nfrac{2}}^p$. Le $K$ est un compact de $\R$ donc il existe $N>0$ tel que 
  $$K\subset [-N,N[,$$
  on pose 
  $$F:=\Par{\O\cap]-N,N[}^c$$
  $F$ est un fermé d'intersection nulle avec $K$, donc on peut poser 
  $$\varphi (t):=\frac{\mathrm d(t,F)}{\mathrm d(t,F)+\mathrm d(t,K)}\text{ pour }t\in \R$$
  $\varphi$ est continue, et pour $t\in F,\ \varphi(t)=0$, donc 
  $$\mathrm{supp}\varphi\subset \O\cap ]-N,N[\subset ]-N,N[$$
  c'est un fermé (car pré-image du fermé $\ens{0}$ par une application continue) et borné, donc c'est un compact et $\varphi$ est continue à support compact.
  On a 
  $$\int_X\abs{\chi_A-\varphi}^p\mathrm d\lambda=\int_K\abs{\chi_A-\varphi}^p\mathrm d\lambda+\int_{\O\backslash K}\abs{\chi_A-\varphi}^p\mathrm d\lambda+\int_{\R\backslash \O}\abs{\chi_A-\varphi}^p\mathrm d\lambda.$$ 
  Pour $t\in K,\ \varphi(t)=1=\chi_A(t)$ donc le premier terme de l'intégrale est nul.

  Pour $t\in \R\backslash\O,\ \varphi(t)=0=\chi_A$ et le dernier terme est aussi nul, donc 
  \begin{align*}
    \int_X\abs{\chi_A-\varphi}^p\mathrm d\lambda&=\int_{\R\backslash \O}\abs{\chi_A-\varphi}^p\mathrm d\lambda\\
    &\leqslant 2^p\lambda(\R\backslash\O)\\
    \int_X\abs{\chi_A-\varphi}^p\mathrm d\lambda&\leqslant \varepsilon^p
  \end{align*}
  donc 
  $$\norm{\chi_A-\varphi}\leqslant \varepsilon.$$
\end{Proof}


\begin{RQ}
  Ce théorème reste vrai dans $L^p(U)$ où $U\subset\R^n$ est un ouvert.
\end{RQ}


\begin{RQ}
  Pour $1\leqslant p< \infty$, $L^p$ est séparable (exercice).
\end{RQ}


\begin{Th}
  Soit $1\leqslant p< \infty$, pour $f\in L^p$ on pose 
  $$f_x:t\longmapsto f(t-x)$$
  l'opérateur de translation,\\

  alors $x\longmapsto f_x$ est uniformément continu.
\end{Th}

La preuve de ce théorème requiert le lemme suivant,


\begin{lemme}
  Soit $f:\R\longrightarrow\C$ continue telle que 
  $$f(x)\underset{\abs{x}\to+\infty}{\longrightarrow}0,$$
  
  alors $f$ est uniformément continue.
\end{lemme}


\begin{ProofC}{Démonstration du lemme}
  Soit $\varepsilon>0$, il existe $a>0$ tel que 
  $$\abs x>a\Rightarrow \abs{f(x)}<\Nfrac{2}\qquad (i)$$
  $f$ est continue sur $[-2a,2a]$ qui est un compact donc (théorème de Heine) $f$ y est uniformément continue et il existe $0<\delta<a$ tel que 
  $$\forall (x,y)\in [-2a,2a],\ \abs{x-y}<\delta\Rightarrow \abs{f(x)-f(y)}<\varepsilon$$
  Soient $(x,y)\in \R$ tels que $\abs{x-y}<\delta$, 

  $\dagger$ si $x$ et $y$ sont dans $[-2a,2a]$ c'est bon,
  
  $\dagger$ si ce n'est pas le cas, supposons que $\abs x>2a$, alors par inégalité triangulaire,
  $$\abs y=\abs{x-(x-y)}\geqslant \abs x-\abs{x-y}\geqslant 2a-\delta>a$$
  donc, d'après $(i)$, on a 
  $$\abs{f(x)-f(y)}\leqslant \abs{f(x)}+\abs{f(y)}\leqslant \varepsilon$$
  et $f$ est effectivement uniformément continue.
\end{ProofC}


\begin{ProofC}{Démonstration du théorème}
  Vérifions d'abord que l'opérateur translation est bien définie, \emph{i.e.} que pour tout $x\in\R,\ f_x\in L^p$. Soit $x\in\R$, on a 
  $$\norm{f_x}_p^p=\int_{-\infty}^{+\infty}\abs{f(x-t)}^p\dt=\int_{-\infty}^{+\infty}\abs{f(u)}^p\mathrm du=\norm{f}_p^p<\infty$$
  donc $f_n$ est bien dans $L^p$.

  $\star$ Si $f$ est continue à support compact (on suppose même que ce support est inclus dans $]-N,N[$), alors son support est borné donc $f(x)\longrightarrow 0$ quand $\abs x\to 0$, ainsi d'après le lemme, $f$ est uniformément continue et il existe $0<\delta<N$ tel que 
  $$\abs{x-y}<\delta\Rightarrow \abs{f(x)-f(y)}<\varepsilon.$$
  On a, par changement de variable,
  $$\intI \abs{f_x(t)-f_y(t)}^p\dt=\intI\abs{f(u)-f(u+(x-y))}^p\mathrm du.$$
  Soient $\abs{x-y}<\delta$, alors

  $\dagger$ si $u\leqslant -N-\delta$, alors 
  $$u+(x-y)\leqslant  -N-\delta+(x-y)\leqslant -N$$
  et $f(u)=f(u+(x-y))=0$.

  $\dagger$ si $u\geqslant N+\delta$, alors 
  $$u+(x-y)\geqslant N+\delta+(x-y)\geqslant N$$
  et $f(u)=f(u+(x-y))=0$.

  On peut donc réduire les bornes d'intégration et 
  \begin{align*}
    \intI\abs{f_x-f_y}^p\mathrm d\lambda&=\int_{-N-\delta}^{N+\delta}\abs{f(u)-f(u+(x-y))}^p\mathrm du\\
    &\leqslant \varepsilon^p\Par(2N+2\delta)\\
    \norm{f_x-f_y}_p^p&\leqslant 4N\varepsilon^p
  \end{align*}
  $N$ étant fixe, on peut revenir au début de la preuve et choisir un $\varepsilon$ adapté, ce qui conclut.\\

  $\star$ Dans le cas général, par densité de $C_{cc}$, il existe $g\in C_{cc}$ telle que 
  $$\norm{f-g}_p<\varepsilon.$$
  De plus, d'après le cas précédent, il existe $\delta>0$ tel que 
  $$\abs{x-y}<\delta\Rightarrow\norm{g_x-g_y}\leqslant \varepsilon$$
  donc, par inégalité triangulaire et pour $\abs{x-y}<\delta$,
  \begin{align*}
    \norm{f_x-f_y}_p&\leqslant \norm{f_x-g_x}_p+\norm{g_x-g_y}_p+\norm{g_y-f_y}_p\\
    &\leqslant \norm{f-g}_p+\norm{g_x-g_y}_p+\norm{g-f}_p\\
    \norm{f_x-f_y}_p&\leqslant 3\varepsilon
  \end{align*}
\end{ProofC}
\NouvellePart{Convolution et tranformée de Fourier}


\PremiereNouvelleSection{Produit de convolution}

\begin{Def}
  Pour $(f,g):\R\longrightarrow\C$, si pour tout $x\in \R,\ t\longmapsto f(x-t)g(t)\in L^1$, on définit la \emph{convolée de }$f$\emph{ par }$g$ par 
  $$\forall x\in\R,\ (f*g)(x):=\int_{-\infty}^{+\infty}f(x-t)g(t)\dt=\int_{-\infty}^{+\infty}f(u)(x-u)\mathrm du.$$
  (on passe de la première intégrale à la seconde via le changement de variable $u=x-t$)
\end{Def}


\begin{RQ}
  On remarque que la loi de composition $*$ est commutative.
\end{RQ}


\begin{Th}
  Soit $1\leqslant p\leqslant \infty$, soit $q$ l'exposant conjugué de $p$,\\

  $(\mathit 1)$ si $f\in L^p,\ g\in L^p$, alors $f*g$ est définie sur $\R$ et $f*g\in C_b(\R)$.\\
  $(\mathit 2)$ si $1<p< \infty$, alors 
  $$(f*g)(x)\longrightarrow 0\text{ quand }\abs x\to 0.$$ 
\end{Th}


\begin{Proof}{Démonstration du théorème}
  À FAIRE
\end{Proof}
\end{document}
