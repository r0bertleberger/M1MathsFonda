\documentclass[a4paper,12pt,reqno]{amsart}
\usepackage{lille}
%==============================================================
\lilleset{
  formation=Master 1\iere année,
  parcours=parcours Mathématiques,
  annee=2024--2025,
  module=Analyse,
  couleur=teal
}
%==============================================================
% -------------- Autres bibliothèques
\usepackage{dsfont} % pour \mathds{1}

% -------------- Les abréviations pour M1AN
% les ensembles standard
\def\R{\mathbb{R}}
\def\C{\mathbb{C}}
\def\N{\mathbb{N}}
\def\Z{\mathbb{Z}}
\def\Q{\mathbb{Q}}
% les parenthèses et leurs dérivées
\newcommand{\p}[1]{\left(#1\right)}% les parenthèses automatiques
\newcommand{\bp}[1]{\bigl(#1\bigr)}% les parenthèses 'big'
\newcommand{\Bp}[1]{\Bigl(#1\Bigr)}% les parenthèses 'Big'
\newcommand{\Bbp}[1]{\biggl(#1\biggr)}% les parenthèses 'bigg'
\newcommand{\BBp}[1]{\Biggl(#1\Biggr)}% les parenthèses 'Bigg'
\newcommand*{\suite}[2][n]{\left( #2 \right)_{#1}}% exemple \suite{u_n}, \suite[k>0]{u_k}
\newcommand*{\suiteN}[1]{\suite[n\in\mathbb{N}]{#1}}% exemple \suiteN{u_n}
\newcommand*{\abs}[1]{\left\lvert{\ifx\hfuzz#1\hfuzz \,\cdot\,\else#1\fi}\right\rvert}% |.|
\newcommand*{\norm}[1]{\left\lVert{\ifx\hfuzz#1\hfuzz \,\cdot\,\else#1\fi}\right\rVert}% norme
\newcommand*{\norminf}[1]{\norm{#1}_{\infty}}% la norme sup
\newcommand*{\vertiii}[1]{{\left\vert\kern-0.25ex\left\vert\kern-0.25ex\left\vert
  {\ifx\hfuzz#1\hfuzz \,\cdot\,\else#1\fi}
    \right\vert\kern-0.25ex\right\vert\kern-0.25ex\right\vert}}
\let\normop\vertiii% exemple \normop{T}
\newcommand*{\scalprod}[3][]{#1\langle{#2}\kern1pt #1|{#3}#1\rangle}% exemple \scalprod[\big]{A}{B}
\newcommand*{\ensemble}[3][]{#1\{ #2 \;#1|\; #3 #1\}}% exemple \ensemble[\big]{x^2}{x \in \R}
% indication et remarque
\newcommand{\indication}[1]{\emph{Indication : #1}}
\newcommand{\remarque}[1]{\emph{Remarque : #1}}
% la fonction indicatrice
\usepackage{dsfont}
\newcommand*{\ind}{\mathds{1}}% exemple \ind_{[0,1]}
% autre symboles
\DeclareMathOperator{\im}{Im}% exemple \im A
\let\ker\relax\DeclareMathOperator{\ker}{Ker}% exemple \ker A
\DeclareMathOperator{\vect}{vect}% exemple \vect\ensemble{x_n}{n\in\N}
\DeclareMathOperator{\dist}{dist}% exemple \dist(x,M)
\DeclareMathOperator{\id}{Id}% exemple T-\lambda\id
\DeclareMathOperator{\supp}{supp}% le support, exemple \supp \ind_{A} = A
\renewcommand{\Re}{\mathfrak{Re}}% la partie réelle
\newcommand{\dd}{\kern1pt\mathrm{d}}% le «d» du dx, dt, ...

% pour surligner
\sisolutions{
  \usepackage{soul}
  \colorlet{hl}{yellow!35!white}
  \sethlcolor{hl}
  \renewcommand{\hl}[1]{\relax\ifmmode\colorbox{hl}{\ensuremath{#1}}\else\texthl{#1}\fi}
}


\lilleset{titre={DM 1, Raphaël Casanova}}

\begin{document}


\begin{exo}
	\begin{enumerate}
	  \item Soit $f,g\in\mathcal{C}([0,1])$. Montrer que
	  \[
		\forall n\in\mathbb{N},\ \int_0^1f(t)t^n\,\mathrm{d}t=\int_0^1g(t)t^n\,\mathrm{d}t \Longleftrightarrow f=g
	  \]
	  \item Soit $f:[0,+\infty[\to\mathbb{R}$ une fonction continue et bornée.
	  \begin{enumerate}
		\item Montrer qu'il existe $g\in\mathcal{C}([0,1])$ telle que pour tout $n\geq 2$
		\[
		  \int_0^{+\infty}f(x)e^{-nx}\mathrm{d}x=\int_0^1g(t)t^{n-2}\,\mathrm{d}t.
		\]
		\item En déduire que si, pour tout $n\geq 2$, $\int_0^{+\infty} f(x)e^{-nx}dx=0$ alors $f=0$.
	  \end{enumerate}
	\end{enumerate}
  \end{exo}

\hrule

\vspace{1cm}
a) si $f=g$, il est clair que 
$$\forall n\in\mathbf N,\ \int_0^1f(t)t^n\mathrm dt=\int_0^1g(t)t^n\mathrm dt.$$
Réciproquement, si cette condition est remplie, on a (par linéarité de l'intégrale)
$$\forall P\in\mathbf R[X],\ \int_0^1(f-g)(t)P(t)\mathrm dt=0\qquad (i)$$
$f-g$ est continue sur le segment $[0,1]$, donc d'après le théorème de Weierstra\ss, il existe une suite $(P_n)$ de polynômes convergeant uniformément vers $f-g$, donc
$$\forall n\in\mathbf N,\ \int_0^1(f-g)(t)P_n(t)\mathrm dt=0$$
par passage à la limite uniforme,
$$\int_0^1(f-g)^2(t)\mathrm dt=0$$
et comme $(f-g)^2\geqslant 0$, $f-g=0$ et comme $f$ et $g$ sont continues, $f=g$.\\[1em]


b)i)On commence par ré-écrire l'intégrale sous la forme suivante :
$$\int_0^{+\infty}f(x)\mathrm e^{-nx}\mathrm dx=\int_0^{+\infty}f(t)\left(\mathrm e^{-x}\right)^n\mathrm dt.$$
Soit le changement de variable bijectif strictement décroissant de classe $C^1$
$$\begin{array}{rclcrcl}
	u&=&\mathrm e^{-x}\qquad&\Leftrightarrow&x&=&-\ln u\\
	\mathrm du&=&-\mathrm e^{-x}\mathrm dx&\Leftrightarrow&\mathrm dx&=&-\frac{1}{u}\mathrm du
\end{array}$$
donc (il n'y a pas de problème de définition de $f$ puisque $-\ln]0,1]=[0,+\infty[$)
$$\int_0^{+\infty}f(t)\mathrm e^{-nx}\mathrm dx=\int_0^1f(-\ln u)u^{n-1}\mathrm du$$
ce qui se ré-écrit
$$\int_0^{+\infty}f(t)\mathrm e^{-nx}\mathrm dx=\int_0^1\underbrace{f(-\ln u)u}_{:=g(u)}u^{n-2}\mathrm du.$$
$g$ est continue sur $]0,1[$ et, puisque $f$ est bornée, 
$$\forall t>0,\ |f(-\ln t)t|\leqslant ||f||_{\infty}t\underset{t\to 0^+}{\longrightarrow}0$$
ainsi $g$ est continue sur $[0,1]$ et c'est bon.


b)ii)Si $f$ vérifie
$$\forall n\geqslant 2,\ \int_0^{+\infty}f(t)\mathrm e^{-nx}\mathrm dx=0$$
alors en gardant les notations précédentes, on a 
$$\forall n\geqslant 2,\ \int_0^1g(t)t^{n-2}\mathrm dt=0$$
autrement dit, en prenant $m:=n-2$,
$$\forall m\in\mathbf N,\ \int_0^1g(t)t^{m}\mathrm dt=0=\int_0^1 0_{C([0,1]}t^n\mathrm dt.$$
D'après la question a), $g=0$ donc
$$\forall t\in]0,1],\ tf(-\ln t)=0$$
donc
$$\forall t>0,\ f(-\ln t)=0$$
et comme $t\mapsto -\ln t$ parcours $[0,+\infty[$, alors 
$$\forall x\in]0,+\infty[,\ f(x)=0$$
autrement dit $f=0$.
\end{document}
