\documentclass[a4paper,12pt,reqno]{amsart}
\usepackage{lille}
%==============================================================
\lilleset{
  formation=Master 1\iere année,
  parcours=parcours Mathématiques,
  annee=2024--2025,
  module=Analyse,
  couleur=teal
}
%==============================================================
% -------------- Autres bibliothèques
\usepackage{dsfont} % pour \mathds{1}

% -------------- Les abréviations pour M1AN
% les ensembles standard
\def\R{\mathbb{R}}
\def\C{\mathbb{C}}
\def\N{\mathbb{N}}
\def\Z{\mathbb{Z}}
\def\Q{\mathbb{Q}}
% les parenthèses et leurs dérivées
\newcommand{\p}[1]{\left(#1\right)}% les parenthèses automatiques
\newcommand{\bp}[1]{\bigl(#1\bigr)}% les parenthèses 'big'
\newcommand{\Bp}[1]{\Bigl(#1\Bigr)}% les parenthèses 'Big'
\newcommand{\Bbp}[1]{\biggl(#1\biggr)}% les parenthèses 'bigg'
\newcommand{\BBp}[1]{\Biggl(#1\Biggr)}% les parenthèses 'Bigg'
\newcommand*{\suite}[2][n]{\left( #2 \right)_{#1}}% exemple \suite{u_n}, \suite[k>0]{u_k}
\newcommand*{\suiteN}[1]{\suite[n\in\mathbb{N}]{#1}}% exemple \suiteN{u_n}
\newcommand*{\abs}[1]{\left\lvert{\ifx\hfuzz#1\hfuzz \,\cdot\,\else#1\fi}\right\rvert}% |.|
\newcommand*{\norm}[1]{\left\lVert{\ifx\hfuzz#1\hfuzz \,\cdot\,\else#1\fi}\right\rVert}% norme
\newcommand*{\norminf}[1]{\norm{#1}_{\infty}}% la norme sup
\newcommand*{\vertiii}[1]{{\left\vert\kern-0.25ex\left\vert\kern-0.25ex\left\vert
  {\ifx\hfuzz#1\hfuzz \,\cdot\,\else#1\fi}
    \right\vert\kern-0.25ex\right\vert\kern-0.25ex\right\vert}}
\let\normop\vertiii% exemple \normop{T}
\newcommand*{\scalprod}[3][]{#1\langle{#2}\kern1pt #1|{#3}#1\rangle}% exemple \scalprod[\big]{A}{B}
\newcommand*{\ensemble}[3][]{#1\{ #2 \;#1|\; #3 #1\}}% exemple \ensemble[\big]{x^2}{x \in \R}
% indication et remarque
\newcommand{\indication}[1]{\emph{Indication : #1}}
\newcommand{\remarque}[1]{\emph{Remarque : #1}}
% la fonction indicatrice
\usepackage{dsfont}
\newcommand*{\ind}{\mathds{1}}% exemple \ind_{[0,1]}
% autre symboles
\DeclareMathOperator{\im}{Im}% exemple \im A
\let\ker\relax\DeclareMathOperator{\ker}{Ker}% exemple \ker A
\DeclareMathOperator{\vect}{vect}% exemple \vect\ensemble{x_n}{n\in\N}
\DeclareMathOperator{\dist}{dist}% exemple \dist(x,M)
\DeclareMathOperator{\id}{Id}% exemple T-\lambda\id
\DeclareMathOperator{\supp}{supp}% le support, exemple \supp \ind_{A} = A
\renewcommand{\Re}{\mathfrak{Re}}% la partie réelle
\newcommand{\dd}{\kern1pt\mathrm{d}}% le «d» du dx, dt, ...

% pour surligner
\sisolutions{
  \usepackage{soul}
  \colorlet{hl}{yellow!35!white}
  \sethlcolor{hl}
  \renewcommand{\hl}[1]{\relax\ifmmode\colorbox{hl}{\ensuremath{#1}}\else\texthl{#1}\fi}
}


\usepackage{packageTest}

\lilleset{titre={DM 2, Raphaël Casanova}}

\begin{document}


\begin{exo}(Théorème de Lax-Milgram)

  Soit $H$ un espace de Hilbert réel. On considère une forme bilinéaire $a$ sur $H$
  et on suppose qu'il existe deux constantes $C > 0$ et $\alpha > 0$ telles que
  \[
    |a(x,y)| \leq C\norm{x}\norm{y} \quad \mbox{ et } \quad a(x,x) \geq \alpha \norm{x}^2 \quad (x,y\in H).
  \]
  \begin{enumerate}
    \item Montrer l'existence d'un opérateur $T \in \mathcal L(H)$ tel que $a(x,y) = \scalprod{Tx}{y}$ pour $x,y \in H$.
    \item Montrer que $T(H)$ est dense dans $H$.
    \item Montrer que $\norm{Tx} \geq \alpha\norm{x}$ pour tout $x\in H$. En déduire que $T$ est injectif à image fermée.
    \item En déduire que $T$ est un isomorphisme de $H$ sur lui-même.
  \end{enumerate}
  Soit $L$ une forme linéaire continue sur $H$.
  \begin{enumerate}[resume]
    \item Montrer qu'il existe un unique $u\in H$ tel que $L(y) = a(u,y)$ pour tout $y\in H$.
    \item On suppose que $a$ est symétrique. Soit $\Phi(x) = \frac{1}{2}a(x,x) - L(x)$. Montrer que le point $u$ vérifie $\Phi(u) = \min_{x\in H}\Phi(x)$.
  \end{enumerate}
\end{exo}

\hrule


\definecolor{darkWhite}{rgb}{0.94,0.94,0.94}
\definecolor{python}{rgb}{0.173,0.255,0.47}
\newcommand*{\norm}[1]{\left\lVert{\ifx\hfuzz#1\hfuzz \,\cdot\,\else#1\fi}\right\rVert} % la norme
\newcommand*{\scalprod}[3][]{#1\langle{#2}\kern1pt #1|{#3}#1\rangle}
\newcommand*{\equivalent}{\Leftrightarrow}
\newcommand*{\Image}[1]{\text{Im} (#1) }

\newcommand{\vertiii}[1]{{\left\vert\kern-0.25ex\left\vert\kern-0.25ex\left\vert #1 
    \right\vert\kern-0.25ex\right\vert\kern-0.25ex\right\vert}}











\begin{document}



\begin{enumerate}
    \item A $x\in H$ fixé, $y \xrightarrow[]{} a(x,y)$ est une forme linéaire sur H (cf. $a$ est bilinéaire donc à $x$ fixé, $y \xrightarrow[]{} a(x,y)$ est linéaire en $y$) . Elle est continue car : $|a(x,y)| \le C \norm{x} \norm{y}$ (par propriété de $a$) donc $\sup_{\norm{y}=1} |a(x,y)| \le C \norm{x}$. \\ Donc d'après le théorème de représentation de Riesz, on a qu'il existe un unique $Tx \in H$ tel que : $a(x,y) = \scalprod{Tx}{y} ~ \forall y \in H $. \\
    On remarque que : $\norm{Tx} = \sup_{\norm{y}=1} |a(x,y)| \le C \norm{x}$. \\
    On pose : $T : x \xrightarrow[]{} Tx$. \\
    Montrons que $T$ est linéaire. Soit $(x_1,x_2) \in H^2$, soit $(\lambda_1, \lambda_2) \in \mathbb{R}^2 $. On a que $\forall y \in H$,
    $$\underbrace{a(\lambda_1x_1 + \lambda_2x_2,y)}_{= \lambda_1 a(x_1,y) + \lambda_2 a(x_2,y) ~ \text{par bilinéarité de $a$}} = \scalprod{T(\lambda_1x_1+\lambda_2x_2}{y} \\   $$
    Or : 
        $$ \lambda_1 a(x_1,y) + \lambda_2 a(x_2,y) = \lambda_1 \scalprod{Tx_1}{y} + \lambda_2 \scalprod{Tx_2}{y}$$
    Donc : 
    $$\scalprod{T(\lambda_1 x_1 + \lambda_2 x_2}{y} = \lambda_1 \scalprod{Tx_1}{y} + \lambda_2 \scalprod{Tx_2}{y}$$
    Ainsi : 
    $$T(\lambda_1 x_1 + \lambda_2 x_2) = \lambda_1 T(x_1) + \lambda_2 T(x_2)$$
    Et $T$ est bien linéaire. \\
    Montrons désormais que $T$ est continue. Soit $x \in H$,
    $$\norm{Tx} \le C \norm{x} \quad \text{d'après ce qui précède}$$
    Donc $T$ est continue et on a que : $\vertiii{T} \le C$. Ainsi $T \in \mathcal{L}(H)$ et on a bien que $\forall x,y \in H, a(x,y) = \scalprod{Tx}{y}$
    \item 
    \begin{align*}
        y \in T(H)^\perp &\equivalent \scalprod{y}{Tx} = 0 ~ \forall x \in H \\
        &\equivalent a(x,y) = 0 ~ \forall x \in H ~ \text{d'après 1)}\\
        &\Rightarrow a(y,y) = 0
    \end{align*}
    Or : $a(y,y) \ge \alpha \norm{y}^2$ d'après les propriétés de $a$. Donc $\norm{y} = 0 $ et $y= 0$ par définition d'une norme donc $T(H)^\perp = \{0\}$. Ainsi $T(H)$ est dense dans $H$ car $T(H)$ est un sous-espace vectoriel. 
    \item On a que pour $x \in H$, 
    \begin{align*}
        \norm{Tx} \norm{x} &\ge |\scalprod{Tx}{x}| ~ \text{par inégalité de Cauchy-Scwharz} \\
        &\ge |a(x,x)| ~ \text{d'après la question 1)} \\
        &\ge \alpha \norm{x}^2 ~ \text{par propriété de $a$}
    \end{align*}
    Si $x=0$, on a bien que $\norm{Tx} = 0 \ge \alpha \norm{x} = 0$ \\
    Si $x \neq 0$, d'après ce qui précède : $\norm{Tx} \ge \alpha \norm{x}$ \\
    Donc pour tout $x\in H$, $\norm{Tx} \ge \alpha \norm{x}$ et donc par corollaire du théorème d'isomorphisme de Banach, $T \in \mathcal{L}(H)$ est injectif à image fermée. \\
    \item On a que : 
    \begin{align*}
        \overline{\Image{T}} &= \Image{T} ~ \text{car $T$ est à image fermée} \\
        &= H ~ \text{car \Image{T} est dense dans $H$}
    \end{align*}
    Or $T$ est surjective sur $\Image{T} = H$. \\
    Et $T$ est injective d'après 3). \\
    Donc $T$ est bijective sur $H$ et est une application linéaire continue (d'après 1)). \\
    D'après le théorème d'isomorphisme de Banach, on alors que T est un isomorphisme de $H$ sur lui-même. 
    \item D'après le théorème de représentation de Riesz, il existe un unique $y_0 \in H$ tel que : $L(y) = \scalprod{y_0}{y} ~ \forall y \in H$ \\
    Comme $T$ est un isomorphisme, il existe un unique $u = T^{-1}(y_ 0) \in H$. Et on a que : 
    \begin{align*}
        a(u,y) &= \scalprod{T(T^{-1}(y_0))}{y} ~\forall y \in H \\
        &= \scalprod{y_0}{y} ~ \forall y \in H \\
        &= L(y) ~ \forall y \in H
    \end{align*}
    D'où l'existence d'un unique $u \in H$ tel que $a(u,y) = L(y)$
    \item Soit $w \in H$, 
    \begin{align*}
        \phi(u+w) &= \dfrac{1}{2}a(u+w, u+w) - L(u+w) \\
        &= \dfrac{1}{2}a(u,u) + a(u,w) + \dfrac{1}{2}a(w,w) - L(u) - \underbrace{L(w)}_{=a(u,w)} ~ \text{cf. $a$ est bilinéaire et symétrique d'après 5)} \\
        &= \underbrace{\dfrac{1}{2}a(u,u) - L(u)}_{=\phi(u)}+ \underbrace{\dfrac{1}{2}a(w,w)}_{\ge 0} \\
        &\ge \phi(u)
    \end{align*}
    On a donc que $\forall w \in H, \phi(u+w) \ge \phi(u)$, ainsi on a donc que : $\phi(u) = \min_{x_\in H} \phi(x)$
\end{enumerate}


\end{document}
